\documentclass[a4paper]{memoir}

\usepackage[ngerman]{babel}
\usepackage{bookmark}
\usepackage{amsmath}
\usepackage{amssymb}
\usepackage{amsthm}
\usepackage[T1]{fontenc}
\usepackage{imakeidx}
\usepackage{enumitem}
\usepackage{mathtools}

\usepackage{svg}
\usepackage{parskip}
\usepackage{hyperref} % Make TOC clickable

\usepackage{lipsum}
\usepackage{fancyhdr} % Heading customization
\usepackage{geometry} % Adjust page padding 
\usepackage{adjustbox}

\usepackage{xcolor}
\usepackage[most]{tcolorbox}

\definecolor{White}{HTML}{F8F8F3}
\definecolor{Black}{HTML}{292939}
\definecolor{CTheorem}{HTML}{FFFFFF}
\definecolor{CLemma}{HTML}{FDFFED}
\definecolor{CDefinition}{HTML}{EFF9F0}
\definecolor{DarkGray}{HTML}{6B6969}
\definecolor{ImportantBorder}{HTML}{F14747}


% Set up page layout
\geometry{
    a4paper,
    left=2.5cm,
    right=2.5cm,
    top=2.5cm,
    bottom=2.5cm
}
\linespread{1.1}
\makeindex

\pagestyle{fancy}
\fancyhf{} % Clear headre/footer
\fancyheadoffset[LE,RO]{0pt} % Adjust headsep for page number and title
\fancyhead[RE,LO]{}
\fancyhead[RE,RO]{\rightmark} % Add pape header title
\fancyhead[LE,LO]{\thepage}  % Add page number to the left
\renewcommand{\headrulewidth}{0pt} % Delete the line in the header 

% Define the \para command
\newcommand{\para}[2]{
    \clearpage % Start on a new page
    \thispagestyle{empty}% removes the top left page number and top right chapter name
    \begin{center} % Center the chapter title
        \vspace*{11em}
        \Huge   \bfseries \S #1 \quad  #2 % Chapter title with symbol and counter
    \end{center}
    \vspace{11em}
    \phantomsection
    \addcontentsline{toc}{chapter}{#1.\hspace{0.6em} #2} % Add chapter to table of contents
    \fancyhead[RE,RO]{#2} % Add pape header title
}

\newcommand{\nsec}[2]{
    \section*{\large  #1 \hspace{0.3em} #2}
    \phantomsection
    \addcontentsline{toc}{section}{#1 \hspace{0.3em} #2}
}
\newcommand{\smalltitle}[2][]{
    \phantomsection
    \subsubsection*{ #1 \hspace{0.2em} #2}
}
% Theorem env
%\newenvironment{ibox}[3]{
%    \phantomsection
%    \addcontentsline{toc}{section}{#1 \hspace{0.3em} #2}
%    \vspace{1.5em}
%    \begin{tcolorbox}[
%        enhanced jigsaw,
%        colback=#3 ,
%        colframe=DarkGray,
%        drop shadow, 
%		attach boxed title to top center={yshift=-2mm},
%        before upper={\vspace{0.5em}},
%        after upper={\vspace{0.5em}},
%        \textbf{ \large #1 \hspace{0.1em} #2\hspace{0.4em}}
%	]}{%
%    \end{tcolorbox}
%}
\newtcolorbox{ibox}[3][]{
	enhanced jigsaw,
	colback=#3,
	colframe=DarkGray,
	coltitle=Black,
	drop shadow,
	title={\textbf{ \large  #1\hspace{0.4em}#2}},
	before skip = 2.7em,
	attach title to upper,
	after title={:\quad \vspace{0.5em}\ },
	%toc = {#1 \quad* #2}{section}
	%IfValueTF={#1}{\addcontentsline{toc}{section}{#1 \hspace{0.3em}#2}}{}
	code={\addcontentsline{toc}{section}{#1 \hspace{0.3em}#2}}
}


\begin{document}
\begin{ibox}[38]{Definition}{CDefinition}
    Seien $ U \subset \mathbb{R}^n  $ offen, $ f: U \to \mathbb{R}^m  $ eine total differenzierbare Abbildung. Seien
	$ a, b \in U $, sodass $ \left| \gamma _{ab} \right|  \subset U $. Dann gibt es $ \phi_1 , \cdots, \phi_{m} 
	\in \left| \gamma_{ab} \right| $, sodass :
	$$ f \left( b \right) = f \left( a \right) + \begin{pmatrix}
		Df_1( \phi_1) \\
		\vdots \\
		Df_{n}(\phi_{m})
	\end{pmatrix} \cdot (b-a)
	 $$
\end{ibox}
\begin{proof}
	Satz 37 auf $ f_1 , \cdots, f_{n} $ anwenden. 
\end{proof}
\smalltitle[]{Beispiel}
$ f: \mathbb{R} \to \mathbb{R}^{2}, \; t \mapsto (t^{2}, t^{3}) $ 
\\
Ist $ [a,b] \subset R $ beschränkt, $ f:[a,b] \to \mathbb{R}  $ stetig differenzierbare, so besagt der HDI:
$$ f(b) - f(a) = \int_{a}^{b} f'(x) dx $$
Ist wieder $ \gamma_{ab}(f) = a + t(b-a) $, so erhält man durch Substitution 
$$ f(b) - f(a) = \int_0^{1} f'( \gamma_{ab}(t)) \gamma_{ab}'(t) dt = \left( \int_0^{1} f' \left( \gamma _{ab} (t) 
\right) dt \right) .(b-a) $$
\begin{ibox}[]{Definition}{CDefinition}
    Sei $ \left[ \alpha , \beta  \right] \in  \mathbb{R}  $ und  $ A: \left[ \alpha , \beta  \right] \to 
	\mathbb{R} ^{m \times n}, A(t) = \left( a_{ij}(t) \right)_{ \substack{ i = 1 , \cdots, m \\ j = 1 , \cdots, n }}$ 
	mit $ a_{ij}: \left[ \alpha , \beta  \right] \to \mathbb{R}  $ stetig für $ i = 1 , \cdots, m \text{ und }  j=1 , \cdots, n $ Dann ist die Integral über die \textit{matrixwertige Abbildung} $ A $ gegeben durch:
$$ \int_{ \alpha }^{ \beta }A(t) dt = \left( \int_{ \alpha }^{ \beta } a_{ij}(t) dt \right)_{\substack{ i=1 , \cdots, m \\
j = 1 , \cdots, n}} $$
\end{ibox}
\begin{ibox}[39]{Satz}{CTheorem}
    Seien $ U \subseteq \mathbb{R}^n  $ offen, $ f: U \to \mathbb{R}^n  $ stetig partiell differenzierbar. Seien 
	$ a,b \in U $ sodass $ \left| \gamma _{ab} \right| \subseteq U $. Dann gilt: 
	$$ f(b) - f(a) = \underbrace{ \left( \int_0^1 Df \left( \gamma _{ab} (t)  \right) dt \right)  }_{ \in \mathbb{R}^{
	m \times n}} \cdot \underbrace{ (b-a)}_{ \in \mathbb{R}^n } $$
\end{ibox}
\begin{proof}
	Wir wenden den HDI an auf $ g_{i}:= \left( f_i \circ \gamma_{ab} \right) \in C^{1}\left( [0,1] \right)$ für $ i = 1 , \cdots, m $ und
	erhalten mit der Kettenregel: 
	\begin{align*}
		f_i(b) - f_i(a) &= g_i(1) - g_i(0) = \int_{ 0 }^{ 1 } \frac{d}{dt} g_i(t) dt \\
						&=  \int_{ 0 }^{ 1 } \left( \sum_{j = 1}^{n} \frac{\partial f_i}{\partial x_j} ( \gamma_{ab}(t))
						\cdot (b_j - a_j)\right) dt \\
						&= \sum_{j=1}^{n} \underbrace{\left( \int_{ 0 }^{ 1 } \frac{\partial f_i}{\partial x_j} 
								\left( \gamma _{ab}(t) \right) \cdot
						dt \right)}_{v_j} \underbrace{ (b_j - a_j}_{u_j} \\
						&=  \left< \left( \int_{ 0 }^{ 1 } Df_i ( \gamma_{ab}(t)) dt \right) , (b-a) \right>
	\end{align*}
	Dies zeigt die $ i- $te Zeile der gewünschte Gleichung. 
\end{proof}

\begin{ibox}[]{Halfsatz}{CDefinition}
    Ist $ \left[ \alpha , \beta  \right] \subseteq \mathbb{R}  $ ein Kompaktes Intervall, $ v : \left[ \alpha , \beta 
	\right] \to \mathbb{R}^n  $ stetige Abbildung. So gilt: 
	$$ \left| \left|  \int_{ \alpha }^{ \beta } v(t) dt  \right|  \right| _{2} \leq \int_{ \alpha }^{ \beta } \|v(t)\|_{1} dt $$
\end{ibox}
\begin{proof}
	Sei $ u := \int_{ \alpha }^{ \beta } v(t) dt \in \mathbb{R}^n  $ . Dann ist 
\begin{align*}
	\| u \|_{2}^{2} = \left<u,u \right> &= \sum_{j=1}^{n} \int_{ \alpha }^{ \beta } v_j(t) dt \cdot u_j\\
	&= \int_{ \alpha  }^{ \beta  } \left( \sum_{j = 1}^{n}v_j(t) \cdot u_j dt \right) \\
	&=  \int_{ \alpha  }^{ \beta  } \left<v(t), u \right> dt \\
  & \leq \int_{ \alpha  }^{ \beta } \left| \left<v(t),u\right> \right| dt \\
  & \stackrel{Schwz.}{ \leq } \int_{ \alpha  }^{ \beta  } \left( \| v(t) \|_{2} \cdot \|u \|_{2} \right) dt \\
  &=  \|u \|_{2} \int_{ \alpha  }^{ \beta  } \|v(t) \|_{2}dt
\end{align*}
Für
$$
\|u \|_{2} > 0 \implies \|u \|_{2} \leq  \int_{ a }^{ b } \|v(t) \|_{2} dt
$$
Für
$$
\|u \|_{2} = 0 \implies \|u \|_{2} = 0 \leq  \int_{ a }^{ b } \|v(t) \|_{2} dt
$$
\end{proof}

\smalltitle[]{Korollar}
Seien $ U \subseteq \mathbb{R}^n  $ offen, $ f: u \to \mathbb{R}^m  $ stetig partiell differenzierbar. Seien 
$ a,b \in U $ mit $ \left| \gamma _{ab} \right| \subseteq U$. Dann gilt: 
$$ \|f(b) - f(a) \|_{2} \leq  M \|b-a\|_{2} $$
für $ M := sup \left\{ \, \|Df(x) \| \,| \; x \in \left| \gamma _{ab} \right|  \right\}  $ Dabei ist die Matrix norm 
$ \|A \| $ für $ A \in \mathbb{R}^{m \times n} $ gegeben durch 
$$ \|A \| = \sup_{ v \in \mathbb{R}^n \setminus \{0\}}  \frac{\|Av \|_2}{\|v \|_{2}} 
=\sup_{\substack{v \in \mathbb{R}^n  \\ \|v \|_{2} = 1}} \frac{\|Av \|_{2}}{\|v \|_{2}}  $$

Es gilt: 
$$ M = \sup_{ \substack{ x \in \left| \gamma _{ab} \right|  \\ v \in  \mathbb{R}^n  \\ \|v \|_{2} = 1 }} 
\underbrace{ \frac{\| Df (x) \cdot v \|_{2}}{\|v \|_{2}} }_{G(x,v)} < \infty \text{ ,da } $$
$$ G: \underbrace{ \| \gamma_{ab} \| \times \partial \mathbb{B}(0,1) }_{kompakt} \to \mathbb{R} \text{ stetig }  $$

\begin{proof}
	Es ist : 
	\begin{align*}
		\|f(b) - f(a) \|_{2} &=  \left|\left| \left( \int_{ 0 }^{ 1 } Df \left( \gamma _{ab}(t) \right) dt \right)
		\cdot (b-a)\right|\right|_{2} \\
		& \leq \int_{ 0 }^{ 1 } \left|\left| Df ( \gamma _{ab}(t)) \cdot (b-a) \right|\right|_{2} dt \\
		& \leq \int_{ 0 }^{ 1 } M \cdot \|b-a \|_{2} dt \\
		&=  M \cdot \|b-a \|_{2}
	\end{align*}
	
\end{proof}


\end{document}

