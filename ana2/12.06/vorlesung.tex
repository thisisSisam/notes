\documentclass[11pt, a4paper]{memoir}


\usepackage[ngerman]{babel}
\usepackage{bookmark}
\usepackage{amsmath}
\usepackage{amssymb}
\usepackage{amsthm}
\usepackage[T1]{fontenc}
\usepackage{imakeidx}
\usepackage{enumitem}
\usepackage{mathtools}

\usepackage{svg}
\usepackage{parskip}
\usepackage{hyperref} % Make TOC clickable

\usepackage{lipsum}
\usepackage{fancyhdr} % Heading customization
\usepackage{geometry} % Adjust page padding 
\usepackage{adjustbox}

\usepackage{xcolor}
\usepackage[most]{tcolorbox}

\definecolor{White}{HTML}{F8F8F3}
\definecolor{Black}{HTML}{292939}
\definecolor{CTheorem}{HTML}{FFFFFF}
\definecolor{CLemma}{HTML}{FDFFED}
\definecolor{CDefinition}{HTML}{EFF9F0}
\definecolor{DarkGray}{HTML}{6B6969}
\definecolor{ImportantBorder}{HTML}{F14747}


% Set up page layout
\geometry{
    a4paper,
    left=2.5cm,
    right=2.5cm,
    top=2.5cm,
    bottom=2.5cm
}
\linespread{1.1}
\makeindex

\pagestyle{fancy}
\fancyhf{} % Clear headre/footer
\fancyheadoffset[LE,RO]{0pt} % Adjust headsep for page number and title
\fancyhead[RE,LO]{}
\fancyhead[RE,RO]{\rightmark} % Add pape header title
\fancyhead[LE,LO]{\thepage}  % Add page number to the left
\renewcommand{\headrulewidth}{0pt} % Delete the line in the header 

% Define the \para command
\newcommand{\para}[2]{
    \clearpage % Start on a new page
    \thispagestyle{empty}% removes the top left page number and top right chapter name
    \begin{center} % Center the chapter title
        \vspace*{11em}
        \Huge   \bfseries \S #1 \quad  #2 % Chapter title with symbol and counter
    \end{center}
    \vspace{11em}
    \phantomsection
    \addcontentsline{toc}{chapter}{#1.\hspace{0.6em} #2} % Add chapter to table of contents
    \fancyhead[RE,RO]{#2} % Add pape header title
}

\newcommand{\nsec}[2]{
    \section*{\large  #1 \hspace{0.3em} #2}
    \phantomsection
    \addcontentsline{toc}{section}{#1 \hspace{0.3em} #2}
}
\newcommand{\smalltitle}[2][]{
    \phantomsection
    \subsubsection*{ #1 \hspace{0.2em} #2}
}
% Theorem env
%\newenvironment{ibox}[3]{
%    \phantomsection
%    \addcontentsline{toc}{section}{#1 \hspace{0.3em} #2}
%    \vspace{1.5em}
%    \begin{tcolorbox}[
%        enhanced jigsaw,
%        colback=#3 ,
%        colframe=DarkGray,
%        drop shadow, 
%		attach boxed title to top center={yshift=-2mm},
%        before upper={\vspace{0.5em}},
%        after upper={\vspace{0.5em}},
%        \textbf{ \large #1 \hspace{0.1em} #2\hspace{0.4em}}
%	]}{%
%    \end{tcolorbox}
%}
\newtcolorbox{ibox}[3][]{
	enhanced jigsaw,
	colback=#3,
	colframe=DarkGray,
	coltitle=Black,
	drop shadow,
	title={\textbf{ \large  #1\hspace{0.4em}#2}},
	before skip = 2.7em,
	attach title to upper,
	after title={:\quad \vspace{0.5em}\ },
	%toc = {#1 \quad* #2}{section}
	%IfValueTF={#1}{\addcontentsline{toc}{section}{#1 \hspace{0.3em}#2}}{}
	code={\addcontentsline{toc}{section}{#1 \hspace{0.3em}#2}}
}



\begin{document}
\begin{ibox}[41]{Satz}{CTheorem}
    Seien $ U \in \mathbb{R}^n  $ offen $ x_0 \in U $ und $ f: U \to \mathbb{R}  $ eine $ k $-mal stetig partiell differenzierbare 
	Funktion. Für $ r>0 $ gelte $ B(x_0,r) \subset U$  Dann gibt es eine Funktion $ \eta : B \left( x_0, r \right) \to \mathbb{R}  $ 
	mit $ \eta \left(x_0\right)  = 0 $ und $ \lim_{ x \to x_0} \frac{\eta \left(x\right) }{|x-x_0|^{k}}  = 0 $. Sodass $ \forall  x \in 
	B \left( x_0, r \right) $ 
	$$ f(x) = \sum_{l=0}^{n} \left(\, \sum_{\substack{ |\alpha| = l \\ \alpha \in \mathbb{N}_{0}^{n} }} \frac{D^{\alpha }f
	\left(x_0\right) }{\alpha !} \cdot \left( x - x_0 \right) ^{\alpha }  \right) \eta \left(x\right)   $$
\end{ibox}
\smalltitle[]{Spezialfall für $ k =2 $ }
Sei $ U \in \mathbb{R}^n  $ offen, $ x_0 \in U $ und $ f \in  C^{2} \left(U\right)  $ Dann gibt es $ c \in \mathbb{R} , a \in \mathbb{R}^n $
$ A \in \mathbb{R}^{n \times n}  $ symmetrisch, sodass die durch 
$$ f \left(x\right)  = c + \left< a, \left( x - x_0 \right)  \right> + \frac{1}{2} \left<(x-x_0), A \cdot (x-x_0) \right> + 
\eta \left(x\right) \; \forall x \in U	$$ 
definierte Funktion $ \eta : U \to \mathbb{R}  $ die Eigenschaften 
$$ \eta \left(x_0\right)  = 0 , \; \lim_{ \substack{ x \to x_0 \\x \neq x_0 } } \frac{\eta \left(x\right) }{ |x - x_0|^{2}} = 0  $$
hat.
\begin{ibox}[]{Hesse Matrix}{CDefinition}
    Sind $ U \subset \mathbb{R}^n  $ offen, $ x_0 \in U $ und $ f: U \to \mathbb{R}  $ ist 2-mal stetig partiell differenzierbare, so
	heißt die symmetrische Matrix : 
	$$ \text{ Hess}f \left(x_0\right) := \begin{pmatrix}
		\frac{\partial ^{2} f}{\partial x_0 \partial x_1} \left(x_0\right) & \cdots    
	 	&\frac{\partial ^{2} f}{\partial x_n \partial x_1} \left(x_0\right) \\
		\vdots & & \vdots \\
		\frac{\partial ^{2} f}{\partial x_n \partial x_1} \left(x_0\right) & \cdots    
	 	&\frac{\partial ^{2} f}{\partial x_n \partial x_n} \left(x_0\right) \\
	\end{pmatrix}
	  $$
\textbf{Hessesche Matrix von} $ f \text{ in } x_0 $ 	
\end{ibox}
\para{9}{Lokale Extrema}
\begin{ibox}[]{Definition}{CDefinition}
    Seien $ U \subset \mathbb{R}^n  $ offen, $ x_0 \in U $ und $ f: U \to \mathbb{R}  $ eine Funktion 
	\begin{enumerate}[label=\alph*)]
		\item f hat in $ x_0 $ eine \textit{lokales Minimum (bzw lokales Maximum)}, wenn es eine offene Umgebung $ x_0 \in V \subset U $ 
			gibt mit $ \forall x \in V : f \left(x\right) \geq  f \left(x_0\right) \; \left( \text{ bzw. } \leq  \right)  $ 
			Falls man sogar $ V $ so wählen dann, dass 
			$$ \forall x \in V : \; f \left(x\right)  > f \left(x_0\right) \; \left( \text{ bzw. } <  \right)  $$
			so spricht man von einem \textit{isoliertes lokal Minimum ( bzw. Maximum)} von $ f $ 
		\item Die Bezeichnung \textit{ (isoliertes) lokales Extremum } bezeichnet sowohl Minima als auch Maxima 
	\end{enumerate}
\end{ibox}
				
\begin{ibox}[42]{Satz}{CTheorem}
    Sei $ U \subset \mathbb{R}^n  $ offen. Die Funktion $ f: U \to \mathbb{R}  $ habe in $ x_0 \in U$ ein lokales Extremum und sei 
	partiell differenzierbar in $ x_0 $ . Dann gilt grad $ f \left(x_0\right) = \left( \frac{\partial f}{\partial x_1} \left(x_0\right)
	, \cdots, \frac{\partial f}{\partial x_n} \left(x_0\right)  \right) = 0 $ 
\end{ibox}
\begin{proof}
	Sei $ r>0 $ sodass $ B \left( x_0, r \right) \subset U $ gilt. Sei ferner $ \left\{ e_1 , \cdots, e_n \right\}  $  die 
	Standardbasis. Dann sit für jedes $ j \in  \left\{ 1 , \cdots,  n \right\}  $ die Funktion $ g_{i j}: (-r, r) \to \mathbb{R} , \;$
	$ t \mapsto f \left( x_0 + t \cdot e_j \right)  $ in der Stelle $ t = 0 $ durch $ g_j' \left(0\right)  = 
	\frac{\partial f}{\partial x_j} \left(x_0\right)  $ differenzierbar und nimmt dort ein Extremum an. Aus Analysis 1 : $ g_{j}' 
	\left(0\right)  = 0$  
\end{proof}
\begin{ibox}[]{Definition}{CDefinition}
    Ist $ A \in \mathbb{R} ^{n \times n} $ eine symmetrische Matrix und $ Q_A $ die durch $ A $  gegebene quadratische Form auf 
	$ \mathbb{R}^n  $ so nennt man $ A $: 
	\begin{enumerate}[label=\alph*)]
		\item \textbf{positiv definit} ( $ A \gg 0 $ ), falls 
			$$ Q_{A} \left(\zeta\right) : = \left< \zeta, A \cdot \zeta \right> > 0 \; \forall \zeta \in \mathbb{R}^n \ \{0\} $$
			\item \textbf{positiv semidefinit},  falls 
			$$ Q_{A} \left(\zeta\right) : = \left< \zeta, A \cdot \zeta \right> \geq  0 \; \forall \zeta \in \mathbb{R}^n \ \{0\} $$
	\item \textbf{negativ definit} ( $ A \ll 0 $ ), falls 
			$$ Q_{A} \left(\zeta\right) : = \left< \zeta, A \cdot \zeta \right> < 0 \; \forall \zeta \in \mathbb{R}^n \ \{0\} $$
			\item \textbf{negativ semidefinit},  falls 
			$$ Q_{A} \left(\zeta\right) : = \left< \zeta, A \cdot \zeta \right> \leq  0 \; \forall \zeta \in \mathbb{R}^n \ \{0\} $$
		\item \textbf{indefinit}, falls Vektoren $ \zeta, \overline{ \zeta} \in \mathbb{R}^n \ \{0\} $ existieren mit: 
			$$ Q_{A} \left( \zeta \right) < 0 \, ; \; Q_{A} \left( \overline{ \zeta}  \right) > 0 $$
			
	\end{enumerate}
Zu jeder symmetrischen Matrix $ A \in \mathbb{R}^{n \times n} $ gibt es $ \lambda_1 , \cdots,  \lambda_n \in \mathbb{R}  $ und 
$ v_1 , \cdots,  v_n \in \mathbb{R}^n  $ mit 
\begin{enumerate}[label=\alph*)]
	\item $ A \cdot v_j = \lambda_j \cdot v_j \; \forall j = 1 , \cdots,  n $ 
	\item $ \forall  i,j = 1 , \cdots,  n: \; \left<v_i, v_j \right> = \delta_{i j} $ 
\end{enumerate}
\end{ibox}
Die Zahlen $ \lambda_j $ heißen \textit{Eigenwerte von $ A $ }, die $ x_i $ sind \textit{ Eigenvektoren von $ A $ } zum Eigenwert 
$ \lambda_j $. Bezüglich einer Orthonormalbasis hat, die durch $ A $ gegebene quadratische Form die Gestalt:
\begin{align*} 
	\zeta = \sum_{j=1}^{n} \alpha_j v_j \implies Q_{A} \left(\zeta\right)  &= \left< \sum_{i = 1}^{n} \alpha_i v_i, A \left(
\sum_{j=1}^{n} \alpha_j v_j\right)  \right>\\
	&=  \sum_{i,j = 1}^{n} \alpha_i \alpha_j \left<v_i, \underbrace{ A v_j}_{= \lambda_j v_j} \right> \\
	&= \sum_{i,j = 1}^{n} \alpha_i \alpha_j \lambda_{j} \delta_{ij} \\
	&= \sum_{i=1}^{n} \lambda_i \alpha_{i}^{n}
\end{align*}
 \smalltitle[]{Lemma}
 Sei $ A \in  \mathbb{R}^{n \times n} $ eine symmetrische Matrix mit Eigenwerten $ \lambda_{1} , \cdots,  \lambda_{n} $. Dann gilt:
 \begin{enumerate}[label=\alph*)]
 	\item A ist positiv definit genau dann, wenn $ \lambda_j > 0 \; \forall j = 1 , \cdots,  n $ 
 	\item A ist positiv semidefinit genau dann, wenn $ \lambda_j \geq 0 \; \forall j = 1 , \cdots,  n $ 
 	\item A ist negativ definit genau dann, wenn $ \lambda_j < 0 \; \forall j = 1 , \ldots,  n $ 
 	\item A ist negativ semidefinit genau dann, wenn $ \lambda_j \leq  0 \; \forall j = 1 , \cdots,  n $ 
 	\item A ist  indefinit genau dann, wenn es $ i,j \in  \left\{ 1, \cdots,  n \right\}  $ gibt mit $ \lambda_i , \cdots, \lambda_j < 0 $ 
 \end{enumerate}
\end{document}
