%\documentclass[a4paper]{memoir}
%
\usepackage[ngerman]{babel}
\usepackage{bookmark}
\usepackage{amsmath}
\usepackage{amssymb}
\usepackage{amsthm}
\usepackage[T1]{fontenc}
\usepackage{imakeidx}
\usepackage{enumitem}
\usepackage{mathtools}

\usepackage{svg}
\usepackage{parskip}
\usepackage{hyperref} % Make TOC clickable

\usepackage{lipsum}
\usepackage{fancyhdr} % Heading customization
\usepackage{geometry} % Adjust page padding 
\usepackage{adjustbox}

\usepackage{xcolor}
\usepackage[most]{tcolorbox}

\definecolor{White}{HTML}{F8F8F3}
\definecolor{Black}{HTML}{292939}
\definecolor{CTheorem}{HTML}{FFFFFF}
\definecolor{CLemma}{HTML}{FDFFED}
\definecolor{CDefinition}{HTML}{EFF9F0}
\definecolor{DarkGray}{HTML}{6B6969}
\definecolor{ImportantBorder}{HTML}{F14747}


% Set up page layout
\geometry{
    a4paper,
    left=2.5cm,
    right=2.5cm,
    top=2.5cm,
    bottom=2.5cm
}
\linespread{1.1}
\makeindex

\pagestyle{fancy}
\fancyhf{} % Clear headre/footer
\fancyheadoffset[LE,RO]{0pt} % Adjust headsep for page number and title
\fancyhead[RE,LO]{}
\fancyhead[RE,RO]{\rightmark} % Add pape header title
\fancyhead[LE,LO]{\thepage}  % Add page number to the left
\renewcommand{\headrulewidth}{0pt} % Delete the line in the header 

% Define the \para command
\newcommand{\para}[2]{
    \clearpage % Start on a new page
    \thispagestyle{empty}% removes the top left page number and top right chapter name
    \begin{center} % Center the chapter title
        \vspace*{11em}
        \Huge   \bfseries \S #1 \quad  #2 % Chapter title with symbol and counter
    \end{center}
    \vspace{11em}
    \phantomsection
    \addcontentsline{toc}{chapter}{#1.\hspace{0.6em} #2} % Add chapter to table of contents
    \fancyhead[RE,RO]{#2} % Add pape header title
}

\newcommand{\nsec}[2]{
    \section*{\large  #1 \hspace{0.3em} #2}
    \phantomsection
    \addcontentsline{toc}{section}{#1 \hspace{0.3em} #2}
}
\newcommand{\smalltitle}[2][]{
    \phantomsection
    \subsubsection*{ #1 \hspace{0.2em} #2}
}
% Theorem env
%\newenvironment{ibox}[3]{
%    \phantomsection
%    \addcontentsline{toc}{section}{#1 \hspace{0.3em} #2}
%    \vspace{1.5em}
%    \begin{tcolorbox}[
%        enhanced jigsaw,
%        colback=#3 ,
%        colframe=DarkGray,
%        drop shadow, 
%		attach boxed title to top center={yshift=-2mm},
%        before upper={\vspace{0.5em}},
%        after upper={\vspace{0.5em}},
%        \textbf{ \large #1 \hspace{0.1em} #2\hspace{0.4em}}
%	]}{%
%    \end{tcolorbox}
%}
\newtcolorbox{ibox}[3][]{
	enhanced jigsaw,
	colback=#3,
	colframe=DarkGray,
	coltitle=Black,
	drop shadow,
	title={\textbf{ \large  #1\hspace{0.4em}#2}},
	before skip = 2.7em,
	attach title to upper,
	after title={:\quad \vspace{0.5em}\ },
	%toc = {#1 \quad* #2}{section}
	%IfValueTF={#1}{\addcontentsline{toc}{section}{#1 \hspace{0.3em}#2}}{}
	code={\addcontentsline{toc}{section}{#1 \hspace{0.3em}#2}}
}


%\begin{document}
\begin{ibox}[46]{Satz}{CTheorem}
    $ F: U_1 \times U_2 \in \mathbb{R}^{k} \times \mathbb{R}  \to \mathbb{R}  $ stetig differenzierbar. $ F \left( a,b \right) = 0, 
		\frac{\partial F}{\partial y} (a,b) \neq 0 \implies\\  \exists!  g  : V_1 \in U_1 \to V_2 \in U_2$. 
		$ \forall (x,y) \in V_1 \times V_2$:
		$$ F (x,y) = 0 \iff y = g (x)  $$
\end{ibox}
\begin{ibox}[47]{Satz}{CTheorem}
	Seien $ U_1 \in \mathbb{R}^k, U_2 \in \mathbb{R}^m  $ offen, $ a \in U_1, b \in  U_2 \text{ und }  F = \left( F_1 , \cdots,  F_m \right) 
	: U_1 \times U_2 \to \mathbb{R}^m $ differenzierbar mit 
	$$ F (x,y) = \begin{pmatrix}
		F_1 (x,y) \\
		\vdots \\
		F_m (x,y) 
	\end{pmatrix}$$
	eine Abbildung mit $ F (a,b) = 0 \in \mathbb{R}^m  $ 
	 und sie $ \text{det} \left( \frac{\partial F}{\partial y } (a,b) \right) \neq 0 $ und sei $ g  = \left( g_1 , \cdots,  g_{m} \right): 
	 U_1 \to U_2$ eine Abbildung d.h $ g (U_1) = U_2 $ mit $ g(a) = b \text{ und } f \left( x, g (x)  \right) = 0 \; \forall x \in U_1 $ 
	wie in (B) differenzierbar. Dann gilt:
	$$
	\text{Dg} (a) = - \left( \frac{\partial F}{\partial y} (a,b)  \right)^{-1} \cdot \left( \frac{\partial F}{\partial x} (a,b)  \right) 
	$$
\end{ibox}
\begin{ibox}[48]{Satz}{CTheorem}
	Sei $ F $ wie in Satz 47 definiert und  differenzierbar in $ (a,b)  $ mit $ U_1 = B (a, \tau_1) \subseteq \mathbb{R}^{k}, U_2 = B (a, \tau
	_2) \subseteq \mathbb{R}^{m} $ und $ \text{det} \left( \frac{\partial F}{\partial y} (a,b)  \right) \neq 0 $. Sei $ g $ wie in Satz 47 
	stetig. Dann ist 
	$ g $ differenzierbar in $ a $ mit 
	$$ \text{Dg} (a) = - \left( \frac{\partial F}{\partial y} (a,b)  \right) ^{-1} \cdot \left( \frac{\partial F}{\partial x}(a,b)\right)$$
\end{ibox}
\begin{ibox}[49]{Satz über implizite Abbildungen}{CTheorem}
    Sei $ F $ wie in Satz 47 stetig differenzierbar mit $  U_1 = B (a, \tau_1) \subseteq \mathbb{R}^{k}, U_2 = B (a, \tau_2) 
		\subseteq \mathbb{R}^{m} $ und $ \text{det} \left( \frac{\partial F}{\partial y } (a,b)  \right) \neq 0  $  
		Dann gibt es offene Umgebungen $ V_1 \subseteq U_1 $ von $ a $, $ V_2 \subseteq U_2 $ von $ b $ und $ g: V_1 \to V_2 $ stetige 
		Abbildung, sodass: $ \forall (x,y) \in V_1 \times V_2 $ 
		$$ F (x,y) = 0 \iff y = g (x)  $$
\end{ibox}
\begin{ibox}[50]{Umkehrsatz}{CTheorem}
	Seien $ U_1, U_2 \subset \mathbb{R}^n $ offen und $ f : U_1 \to U_2$, stetig differenzierbar $ a \in U_1 $ mit $ \text{det} Df (a) 
	\neq 0, \; b := f (a) \in U_2 $ Dann gibt es offene Umgebungen $ W_1 \subset U_1 $ von $ a $, $ W_2 \subseteq U_2 $ von $ b $ 
	und eine stetig differenzierbar Abbildung $ g: W_2 \to W_1 $ mit $ g \circ \left( f | _{W_1} \right) = \text{id}_{W_1}  $ 
 $  \left( f | _{W_2} \right) \circ g = \text{id}_{W_2} $ 
\end{ibox}

\para{11}{Methode der Langrange'schen Multiplikatoren}
\begin{ibox}[]{Parametergebiet}{CDefinition}
	Ein beschränktes Gebiet $ P \subseteq \mathbb{R}^n  $ heißt \textit{Parametergebiet}, wenn $ \partial P = \partial ( \overline{P}) $ 
\end{ibox}

\begin{ibox}[]{Parametrisiertes Flächenstück}{CDefinition}
    Sei $ P \subseteq \mathbb{R}^p $ ein Parametergebiet. Ein parametrisiertes Flächenstück über $ P $ ist eine stetig 
		differenzierbar Abbildung $ \varphi : P \to \mathbb{R}^n  $, sodass: 
		\begin{itemize}
			\item $ \varphi  $ injektiv
			\item rang $ D \varphi (x) = p \; \forall x \in  P$ 
			\item Ist $ x_0 \in P \text{ und }  \left( x_{y} \right)_{y} \text{ s.d. } \lim_{ y \to 0} \varphi_{y} = \varphi (x_0) $.
				so ist $ \lim_{ y \to \infty x_{y} = x_0}  $ 
		\end{itemize}
	Die Zahl $ p $ heißt die Dimension des Flächenstücks. Für $ p = 1 $ heißt $ \varphi  $ auch \textit{glatter Weg} 	
\end{ibox}

\begin{ibox}[]{Glatte Fläche, Untermannigfaltigkeit}{CDefinition}
    Eine Menge $ M \subset \mathbb{R}^n  $ heißt $ p- $dimensionale glatte Fläche (Untermannigfaltigkeit), falls es zu jedem 
		$ x_0 \in M $ eine Umgebung $ U = U (x_0) \subseteq \mathbb{R}^n  $ und ein glattes parametrisiertes Flächenstück 
		$ \varphi P \to \mathbb{R}^n  $ mit $ \varphi (\zeta) = x_0 $ für ein $ \zeta_0 $ und $ \varphi (P) = U \cap M $ $ \varphi  $ heißt
		dann \textit{lokale Parametrisierung} von $ M $ in $ x_0 $. Für $ p = n-1 $ nennen wir $ M $ eine \textit{Hyperfläche}.
\end{ibox}

\begin{ibox}[51]{Satz}{CTheorem}
    Sei $ C \subseteq \mathbb{R}^n  $ offen, $ M \subseteq B, 0 \leq q \leq n $. Es gebe stetige differenzierbar Funktion 
		$ f_1 , \cdots,  f_{q}: B \to \mathbb{R}  $, sodass 
		\begin{enumerate}
			\item $ M = \left\{ x \in B | f_1 (x)  = \cdots = f_{q} (x)  = 0 \right\}  $ 
			\item Die Vektoren grad $ \left( f_1 (x)  \right) , , \cdots, \text{ grad} \left( f_{q} (x)  \right)   $ sind $ \forall x \in M $ 
				linear unabhängig.
		\end{enumerate}
	Dann ist $ M $ eine $ p$-dimensionale Untermannigfaltigkeit mit $ p = n -q $ 	
\end{ibox}
\begin{ibox}[]{Definition}{CDefinition}
    Sei $ M $ eine $ U $-dimensionale Untermannigfaltigkeit in $ \mathbb{R}^n $. Eine Funktion $ h : M \to \mathbb{R} $ heißt 
		differenzierbar, falls für jede Parametrisierung $ \varphi : P \to \mathbb{R}^n $ von $ M $ $ h \circ \varphi : P \to \mathbb{R} $ 
		differenzierbar ist
\end{ibox}

\begin{ibox}[]{Definition}{CDefinition}
    Sei $ B \subseteq \mathbb{R}^n $ offen, $ g = \left( g_1 , \cdots,  g_{m} \right) B : \mathbb{R}^m $ stetig differenzierbar mit 
		$ \operatorname{rang} \left( Dg (x)  \right) \; \forall x \in B $. Weiter sei $ H \left\{ x \in B | g (x)  = 0\right\}, a \in M  
		U = U (a) \subseteq S$ offene Umgebung, $ f: U \to \mathbb{R} $ stetig differenzierbar. $ f $ hat eine \textit{relatives Maximum (bzw.
		Minimum)} in $ a $ unter der Nebenbedingungen $ g_1 (x)  = \cdots = g_{m} (x)  = 0 	$, falls $ f (x) \leq f (a) \; \forall x 
		\in M \cap U  (\text{ bzw. } f (x) \geq f (a) \; \forall x \in U \cap M ) $ 

\end{ibox}

\begin{ibox}[52]{Methode der Langrange'schen Multiplikatoren}{CTheorem}
   Hat $ f $ in $ a $ ein relatives Extremum unter den Nebenbedingungen $ g_1 (x)  = \cdots = g_{m} (x)  = 0 $, so gibt es 
	 $ \lambda_1 , \cdots,  \lambda_m \in \mathbb{R}  $ sodass 
	 \begin{equation}
		 \text{grad} f (a) = \lambda_1 \text{grad} (g_1 (a) ) + \cdots + \lambda_m \text{grad} (g_m (a) )
		 \tag{$\star$}
	 \end{equation}
	Die Zahlen $ \lambda_1 , \cdots,  \lambda_{m} $  
\end{ibox}

%\end{document}
