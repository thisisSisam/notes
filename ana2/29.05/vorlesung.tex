%\documentclass[11pt, a4paper]{memoir}

%
\usepackage[ngerman]{babel}
\usepackage{bookmark}
\usepackage{amsmath}
\usepackage{amssymb}
\usepackage{amsthm}
\usepackage[T1]{fontenc}
\usepackage{imakeidx}
\usepackage{enumitem}
\usepackage{mathtools}

\usepackage{svg}
\usepackage{parskip}
\usepackage{hyperref} % Make TOC clickable

\usepackage{lipsum}
\usepackage{fancyhdr} % Heading customization
\usepackage{geometry} % Adjust page padding 
\usepackage{adjustbox}

\usepackage{xcolor}
\usepackage[most]{tcolorbox}

\definecolor{White}{HTML}{F8F8F3}
\definecolor{Black}{HTML}{292939}
\definecolor{CTheorem}{HTML}{FFFFFF}
\definecolor{CLemma}{HTML}{FDFFED}
\definecolor{CDefinition}{HTML}{EFF9F0}
\definecolor{DarkGray}{HTML}{6B6969}
\definecolor{ImportantBorder}{HTML}{F14747}


% Set up page layout
\geometry{
    a4paper,
    left=2.5cm,
    right=2.5cm,
    top=2.5cm,
    bottom=2.5cm
}
\linespread{1.1}
\makeindex

\pagestyle{fancy}
\fancyhf{} % Clear headre/footer
\fancyheadoffset[LE,RO]{0pt} % Adjust headsep for page number and title
\fancyhead[RE,LO]{}
\fancyhead[RE,RO]{\rightmark} % Add pape header title
\fancyhead[LE,LO]{\thepage}  % Add page number to the left
\renewcommand{\headrulewidth}{0pt} % Delete the line in the header 

% Define the \para command
\newcommand{\para}[2]{
    \clearpage % Start on a new page
    \thispagestyle{empty}% removes the top left page number and top right chapter name
    \begin{center} % Center the chapter title
        \vspace*{11em}
        \Huge   \bfseries \S #1 \quad  #2 % Chapter title with symbol and counter
    \end{center}
    \vspace{11em}
    \phantomsection
    \addcontentsline{toc}{chapter}{#1.\hspace{0.6em} #2} % Add chapter to table of contents
    \fancyhead[RE,RO]{#2} % Add pape header title
}

\newcommand{\nsec}[2]{
    \section*{\large  #1 \hspace{0.3em} #2}
    \phantomsection
    \addcontentsline{toc}{section}{#1 \hspace{0.3em} #2}
}
\newcommand{\smalltitle}[2][]{
    \phantomsection
    \subsubsection*{ #1 \hspace{0.2em} #2}
}
% Theorem env
%\newenvironment{ibox}[3]{
%    \phantomsection
%    \addcontentsline{toc}{section}{#1 \hspace{0.3em} #2}
%    \vspace{1.5em}
%    \begin{tcolorbox}[
%        enhanced jigsaw,
%        colback=#3 ,
%        colframe=DarkGray,
%        drop shadow, 
%		attach boxed title to top center={yshift=-2mm},
%        before upper={\vspace{0.5em}},
%        after upper={\vspace{0.5em}},
%        \textbf{ \large #1 \hspace{0.1em} #2\hspace{0.4em}}
%	]}{%
%    \end{tcolorbox}
%}
\newtcolorbox{ibox}[3][]{
	enhanced jigsaw,
	colback=#3,
	colframe=DarkGray,
	coltitle=Black,
	drop shadow,
	title={\textbf{ \large  #1\hspace{0.4em}#2}},
	before skip = 2.7em,
	attach title to upper,
	after title={:\quad \vspace{0.5em}\ },
	%toc = {#1 \quad* #2}{section}
	%IfValueTF={#1}{\addcontentsline{toc}{section}{#1 \hspace{0.3em}#2}}{}
	code={\addcontentsline{toc}{section}{#1 \hspace{0.3em}#2}}
}



%\begin{document}
Daran kann man sofort ablesen, dass folgende Satz richtig ist.
	\begin{ibox}[32]{Satz}{CTheorem}
	   Seien $ U \subset \mathbb{R}^n $  offen, $ x_0 \in U $ und $ f = \left( f_1, \cdots, f_{m} \right)  $  
	   eine Abbildung $ f:U \to \mathbb{R}^m $. Dann gilt:
	   \begin{enumerate}[label=\alph*)]
	   	\item $ f $ ist total differenzierbar in $ x_0 $ genau dann, wenn jede der Komponentenfunktion $ f_{i}, 1 \leq i \leq m	 $
			total differenzierbar in $ x_0 $ ist.
		\item Wenn $ f $ in $ x_0 $ total differenzierbar ist, so ist $ f $ an der Stelle $ x_0 $ stetig.
		\item Wenn $ f $ in $ x_0 $ total differenzierbar und die $ \left( m \times n \right)  $ Matrix $ A $ wie Definition
			von oben gewählt ist, dann ist alle Komponentenfunktion $f_{i}, 1 \leq i \leq m$ in $x_0 $ partiell differenzierbar
			mit $ D_{\nu} f_{i}(x_0) = \frac{\partial f_{i}}{\partial x_{\nu}} (x_0) = a_{i \nu}; \; \forall \nu = 1, \cdots n 
			; \; \forall  i = 1, \cdots, m$ . 
		\item Wenn alle Komponentenfunktion $ f_{i} $ auf $ U $ partiell differenzierbar sind und alle partiell Ableitungen
			$ D_{\nu}f_{i} $ an der Stelle $ x_0 $ stetig sind, dann ist $ f $ in $ x_0 $ total differenzierbar.
	   \end{enumerate}
	\end{ibox}
	 
	\begin{ibox}[]{Definition}{CDefinition}
	    Ist $ U \subset  \mathbb{R}^n  $ offen und $ f = \left( f_1, \cdots, f_{m} \right) : U \to \mathbb{R}^m $ total 
		differenzierbar in $ x_0 \in  U $, so nennt man die Matrix $ Df(x_0) := J_{f}(x_0) := 
		\left( \frac{\partial f_{i}}{\partial x_{j}} (x_0) \right)_{\substack{i = 1 , \cdots, m \\ j = 1, \cdots, n}}  $
		$$ := \begin{pmatrix}
			\frac{\partial f_1}{\partial x_1} (x_0) & \cdots & \frac{\partial f_1}{\partial x_{n}} (x_{0}) \\
			\vdots & &\vdots\\
			\frac{\partial f_{m}}{\partial x_{1}} (x_0) & \cdots & \frac{\partial f_{m}}{\partial x_{n}} (x_0)
		\end{pmatrix}
		 $$
	das \textit{Differential} oder die \textit{Jacobi-Matrix } oder die \textit{Funktionalmatrix } von $ f  $ in $ x_0 $  	
	\end{ibox}
	
	\begin{ibox}[33]{Satz}{CTheorem}
	    Seien $ U \subset \mathbb{R}^n  $ offen, $ x_0 \subset U, c \in \mathbb{R}  $, die Abbildung $ f:U \to \mathbb{R}^m $ 
		und $ g:U \to \mathbb{R}^m $ seien total differenzierbar in $ x_0 $. Dann gilt:
		\begin{enumerate}[label=\alph*)]
			\item Die Abbildung $ \left( f+g \right) : U \to \mathbb{R}^m  $ ist total differenzierbar in $ x_0 $ mit 
				$ D \left( f+g \right) (x_0) = Df\left( x_0 \right) + Dg\left( x_0 \right)  $ 
			\item Die Abbildung $ \left( c.f \right) : U \to \mathbb{R}^m $ ist total differenzierbar in $ x_0 $ mit 
				$ D\left( c f \right) (x_0) = c Df(x_0) $ 
		\end{enumerate}
	\end{ibox}
	\begin{proof}
		Wie in Analysis 1 für reellwertige Funktionen. 
	\end{proof}
Eine \textit{Productregel} gilt in folgender Form:
\begin{ibox}[34]{Definition}{CDefinition}
    Seien $ U \subset \mathbb{R}^n  $ offen, $ x_0 \subset U $ die Abbildung $ f = \left( f_1, \cdots, f_{m} \right) 
	:U \to \mathbb{R}^m$ und die Funktion $ g: U \to \mathbb{R}  $ siene total differenzierbar in $ x_0 $ . Dann gilt:
	Die Abbildung $ \left( f \circ g \right) : U \to \mathbb{R}^m \; x \mapsto \left( f_1(g_1(x)), \cdots, f_{m}(g_{x}(x_m)) \right) $ 
	ist in $ x_0 $ total differenzierbar mit
	$$
	\underbrace{ D \left( f \circ g \right) (x_0)}_{(m\times n)-Matrix} =
	\underbrace{f\left( x_0 \right)}_{(m\times 1)} \cdot \underbrace{Dg(x_0)}_{(1\times n)} +
	g(x_0)\underbrace{ Df(x_0)}_{(m\times n )- Matrix}
	$$
\end{ibox}

\begin{proof}
	Die totale  
\end{proof}

\begin{ibox}[35]{Kettenregel}{CDefinition}
    Seien $ U \subset \mathbb{R}^n \text{ und } V \subset \mathbb{R}^m  $ offen Menge, ferner $ g: U \to \mathbb{R}^m  \text{ und } 
	f:V \to \mathbb{R}^l $ Abbildungen mit $ g(U) \subset V  $. $ g $ sie total differenzierbar an der Stelle $ x_0 \in U $ 
	, $ f $ sie total differenzierbar in $ y_0 = g\left( x_0 \right)  $. Dann ist die Abbildung $ \left( f \cdot g \right) 
	:U \to \mathbb{R}^l  $ total differenzierbar in $ x_0 $ mit $ \underbrace{D \left( f \circ g \right) (x_0) }_{l\times n}= 
	\underbrace{Df(g(x_0))}_{l \times m} \cdot \underbrace{Dg(x_0)}_{m \times n}$ 
\end{ibox}

\smalltitle[]{Korollar}
$U\subset \mathbb{R} ^n$ und $V\subset \mathbb{R} ^m$ offene Mengen, die Abbildung
$ \left( g_1 , \cdots, g_{m} \right) : U \to \mathbb{R}^m  $ sei total differenzierbar in $ x_0 \in U $,
es gelte $ g(U) \subset U $ und die Funktion $ f: U \to \mathbb{R}  $ sei total differenzierbar in 
$ y_0 = g(x_0) $ . Dann ist die Funktion $ h:= (f \circ g): U \to \mathbb{R}  $ total differenzierbar in 
$ x_0 $ mit partiellen Ableitungen 
$$ \frac{\partial h}{\partial x_{i}} (x_0) = \sum_{j=1}^{m} \frac{\partial f }{\partial y_{i}} (y_0)
\cdot \frac{\partial g_{i}}{\partial x_{i}} (x_0)  \;  \forall 1 \leq j \leq n$$ 


\begin{ibox}[]{Definition}{CDefinition}
    Sei $ U \subset  \mathbb{R}^n  $ offen, ferner $ x_0 \in U $ und $ f: U \to \mathbb{R}  $ eine 
	Funktion auf $ U $. Dann bezeichnet man für $ \upsilon \in  \mathbb{R}^n \ \{0\} $ den Grenzwert 
	$$ D_{\upsilon}f(x_0) := \lim_{t \to 0} \frac{f(x_0 + t \cdot \upsilon)-f(x_0)}{t}  $$
	(im Falle siner Existenz) als \textit{Richtungsableitung von $ f $ an der Stell $ x_0 $ in Richtug $ \upsilon $ } 
	und nennt $ f $ \textit{ an der Stell $ x_0 $ in Richtug $ \upsilon $ differenzierbar} 
\end{ibox}

\begin{ibox}[36]{Satz}{CTheorem}
    Sei $ U \in \mathbb{R}^n  $ offen und sei $ f: U \to \mathbb{R}  $ eine in $ x_0 \in U $ total differenzierbar
	Funktion. Dann existiert für jedes $ \upsilon =  \left( \upsilon_1 , \cdots, \upsilon_2 \right) \in \mathbb{R}^n \ \{0\} $ 
	die Richtungsableitung $ D_{\upsilon}f(x_0) $ von $ f $ an der Stelle $ x_0 $ in Richtug $ \upsilon$ und es gilt:
	$$ D_{\upsilon}f(x_0) = \sum_{i=1}^{n} \frac{\partial f}{\partial x_{i}} (x_0) \cdot \upsilon_1  =
	\left<Df(x_0), \upsilon \right>$$
	Wobei
	$$ grad f(x_0) := \left( \frac{\partial f}{\partial x_1} (x_0)  , \cdots, \frac{\partial f}{\partial x_n} (x_0) \right)$$
\end{ibox}
\begin{proof}
	Sei ein beliebiger Vektor $ \upsilon \in \mathbb{R}^n \ \{0\} $ gegeben. Die Abbildung $ g: \mathbb{R}  \to \mathbb{R}^n 
	 \; t \mapsto g(t) = x_0 + t \upsilon$  ist differenzierbar auf $ \mathbb{R}  $, und es ist $ g(0) = x_0 $. 
	 Aufgrund der Stetigkeit von $ g $ an der stelle $ 0 $ existiert ein Intervall $ I_{\varepsilon} = \left( - \varepsilon 
	 , \varepsilon \right) \in \mathbb{R}  $ so dass $ g \left( I_{ \varepsilon } \right) \subset  B(x_0,t) $ gilbt, hierbei
	 sie $ r > 0 $ so gewahlt, dass $ B(x_0, r) \subset U $ erfüllt ist, was aufgrund der Offenheit von $ U $ möglich ist.
	 Dann ist die Funktion $ h:=(f \circ g): I_{ \varepsilon } \to \mathbb{R} \; t \mapsto h(t) := f(x_0+t \upsilon) $ 
	 differenzierbaran der Stelle $ t = 0 $, und nach dem Korollar zur Kettenregel gilt:
	 $$ D_{\epsilon}f(x_0) = \lim_{t \to 0} \frac{f(x_0+t \upsilon) - f(x_0)}{t} = h'(0) = 
	 \sum_{i = 1}^{n} \frac{\partial f}{\partial x_{i}} (x_0) \frac{dg_{i}}{t}(0) 
	 $$
	  $$ = \sum_{i=1}^{n} \frac{\partial f}{\partial x_{i}} (x_0) \cdot \upsilon_{i} = \left<Df(x_0), \upsilon \right> $$
	   
\end{proof}
\para{7}{Mittelwertsatz}
\begin{ibox}[37]{Satz}{CTheorem}
    Seien $ U \subset  \mathbb{R}^n  $ offen und $ f: U \to \mathbb{R}  $ eine total differenzierbar Funktion. Seien $ a,b \in U $
	zwei Punkte, deren Verbindungsstrecke $ \left| \gamma_{ab} \right| := \left\{ \gamma_{ab} := a + t(b-a); \; 
	t \in [1,0] \right\} \subset  U \text{ in } U $ enthalten ist. Dann gibt es ein $ \xi \in \left| \gamma_{ab} \right|  $ 
	derart dass $ f(b) = f(a)+ Df(\xi) \cdot (b-a)  $ gibt.
\end{ibox}
\begin{proof}
	Die Abbildung $ \gamma_{ab}: [0,1] \to U \; t \mapsto \gamma_{ab}(t) = a + t(b-a) $ ist differenzierbar auf $ \mathbb{R}  $ 
	mit $ D \gamma_{ab}(t) = (b-a) \; \forall t \in \mathbb{R}  $. Da $ \left| \gamma_{ab} \right| = \gamma_{ab}([0,1]) \subset  U $
	gilt und $ f $ auf $ U $ differenzierbar ist, ist die Komposition $ \left( f \cdot \gamma_{ab} \right) : [0,1] \to \mathbb{R}  $ 
	eine auf $ [0,1]  $ differenzierbare Funktion, auf die dier Mittelwertsatz für funktionen eine Variabel anwendbar ist. 
	Daher $ \exists t_0 \in (0,1) $ so dass
	\begin{equation}
	\left( f \circ \gamma_{ab} \right) (1) = \left( f \circ \gamma_{ab} \right) (0) + 
	\left( f \circ \gamma_{ab} \right) (t_0) \cdot (1-0)
	\end{equation}
	gilt. Da aber $ \gamma_{ab}(1) = b $ und $ \gamma_{ab}(0) = a $ ist,
	ferner für $ s := \gamma_{ab}(t_0) \in \left| \gamma_{ab} \right| $ mit der Kettenregel $ \left( f \circ \gamma_{ab} \right)'
	(t_0) = Df \left( \gamma _{ab}(t_0) \right) \cdot D \gamma _{ab}(t_0) = Df( \xi) \cdot (b-a)$ folgt, ist Gleichung (1) 
	äquivalent zu $ f(b) = f(a) + Df(\xi)\cdot(b-a) $ 
\end{proof}


%\end{document}
