\documentclass[11.5 pt, a4paper]{memoir}


\usepackage[ngerman]{babel}
\usepackage{bookmark}
\usepackage{amsmath}
\usepackage{amssymb}
\usepackage{amsthm}
\usepackage[T1]{fontenc}
\usepackage{imakeidx}
\usepackage{enumitem}
\usepackage{mathtools}

\usepackage{svg}
\usepackage{parskip}
\usepackage{hyperref} % Make TOC clickable

\usepackage{lipsum}
\usepackage{fancyhdr} % Heading customization
\usepackage{geometry} % Adjust page padding 
\usepackage{adjustbox}

\usepackage{xcolor}
\usepackage[most]{tcolorbox}

\definecolor{White}{HTML}{F8F8F3}
\definecolor{Black}{HTML}{292939}
\definecolor{CTheorem}{HTML}{FFFFFF}
\definecolor{CLemma}{HTML}{FDFFED}
\definecolor{CDefinition}{HTML}{EFF9F0}
\definecolor{DarkGray}{HTML}{6B6969}
\definecolor{ImportantBorder}{HTML}{F14747}


% Set up page layout
\geometry{
    a4paper,
    left=2.5cm,
    right=2.5cm,
    top=2.5cm,
    bottom=2.5cm
}
\linespread{1.1}
\makeindex

\pagestyle{fancy}
\fancyhf{} % Clear headre/footer
\fancyheadoffset[LE,RO]{0pt} % Adjust headsep for page number and title
\fancyhead[RE,LO]{}
\fancyhead[RE,RO]{\rightmark} % Add pape header title
\fancyhead[LE,LO]{\thepage}  % Add page number to the left
\renewcommand{\headrulewidth}{0pt} % Delete the line in the header 

% Define the \para command
\newcommand{\para}[2]{
    \clearpage % Start on a new page
    \thispagestyle{empty}% removes the top left page number and top right chapter name
    \begin{center} % Center the chapter title
        \vspace*{11em}
        \Huge   \bfseries \S #1 \quad  #2 % Chapter title with symbol and counter
    \end{center}
    \vspace{11em}
    \phantomsection
    \addcontentsline{toc}{chapter}{#1.\hspace{0.6em} #2} % Add chapter to table of contents
    \fancyhead[RE,RO]{#2} % Add pape header title
}

\newcommand{\nsec}[2]{
    \section*{\large  #1 \hspace{0.3em} #2}
    \phantomsection
    \addcontentsline{toc}{section}{#1 \hspace{0.3em} #2}
}
\newcommand{\smalltitle}[2][]{
    \phantomsection
    \subsubsection*{ #1 \hspace{0.2em} #2}
}
% Theorem env
%\newenvironment{ibox}[3]{
%    \phantomsection
%    \addcontentsline{toc}{section}{#1 \hspace{0.3em} #2}
%    \vspace{1.5em}
%    \begin{tcolorbox}[
%        enhanced jigsaw,
%        colback=#3 ,
%        colframe=DarkGray,
%        drop shadow, 
%		attach boxed title to top center={yshift=-2mm},
%        before upper={\vspace{0.5em}},
%        after upper={\vspace{0.5em}},
%        \textbf{ \large #1 \hspace{0.1em} #2\hspace{0.4em}}
%	]}{%
%    \end{tcolorbox}
%}
\newtcolorbox{ibox}[3][]{
	enhanced jigsaw,
	colback=#3,
	colframe=DarkGray,
	coltitle=Black,
	drop shadow,
	title={\textbf{ \large  #1\hspace{0.4em}#2}},
	before skip = 2.7em,
	attach title to upper,
	after title={:\quad \vspace{0.5em}\ },
	%toc = {#1 \quad* #2}{section}
	%IfValueTF={#1}{\addcontentsline{toc}{section}{#1 \hspace{0.3em}#2}}{}
	code={\addcontentsline{toc}{section}{#1 \hspace{0.3em}#2}}
}




\title{Analysis 2 (Prof. Scherbina) Notizen}
\author{Sisam Khanal}
\date{\today}
\begin{document}
\color{Black}
\maketitle
\vfill
{\centering \large Aktuelle Notizen unter: \\}
\vspace{0.3cm} 
\centerline{\href{https://thisissisam.github.io/notes/ana2/main.pdf}{https://thisissisam.github.io/notes/ana2/main.pdf}}
\thispagestyle{empty}

\para{1}{Der $ \mathbb{R}^{n}$ und seine Topologie  }

Erklärt man auf der Menge $ R^{n} := \left\{ \alpha = \left( x_1,x_2,x_3 \dots  \right) | x_{i} \in \mathbb{R} \forall i = 1, \dots n \right\}  $ 
\index{Vektorraum}
aller geordneten $ n $-Tupel reeller Zahlen eine Addition komponentenweise 
durch  $ x+y = x_1+y_1, x_2+y_2, \dots x_{n} + y_{n}$  für 
$ x = \left( x_1,x_2, \dots, x_{n} \right), y = \left( y_1,y_2\dots y_{n} \right)
\in \mathbb{R}^{n} $ und eine Multiplikation mit reellen Skalaren. So erhält
$ \mathbb{R}^{n} $ die Structure eines $ n $-dimensionalen Vektorraums über
$ \mathbb{R} $; eine Basis des $ \mathbb{R}^{n} $ ist durch die Vektoren
$ e_1 = \left( 1, 0, \dots , 0 \right) \dots e_{n} = \left( 0, \dots, 0, 1\right)
$ gegeben. Man bezeichnet die Familie $ \left\{ e_1, e_2, \dots e_{n} \right\} $
als \textit{Standardbasis} oder auch kanonische Basis des $ \mathbb{R}^{n} $ 
   
\begin{ibox}{euklidisches Skalarprodukt}{CDefinition}
    Das euklidisches Skalarprodukt auf $ \mathbb{R}^{n} $ ist die Abbildung
    $ \left< \circ , \circ \right> : \mathbb{R}^{n} \times \mathbb{R}^{n}
    \to \mathbb{R} \left( x, y \right) \mapsto \left<x, y \right> = 
    \sum_{1=1}^n x_{i}y_{i}$ die je zwei Vektoren $ x, y \in \mathbb{R}^{n} $
    die reellen Zahl $ \left<x,y \right> $zugeordnet, welsche man 
    \textit{euklidisches} Skalarprodukt  von $ x \text{ und } y $ nennt. 
\end{ibox}
    \index{Skalarprodukt}

Eine weitere gebräuchliche Schreibweise für das euklidische Skalarprodukt zwei
Vektor ist $ \left<x,y \right> =: x \cdot y $ \\
Die wichtigsten Eigenschaft des euklidische Skalarprodukt nennt der folgende
Satz, auf dessen trivialen Beweis wir verzichten werden.

\begin{ibox}{Satz}{CTheorem}
    Für alle $ x, y, z \in \mathbb{R}^{n} $ und jedes $ \alpha \in  \mathbb{R} $ gilt:
    \begin{enumerate}[label=\alph*)]
        \item $ \left< \left( x + z \right) , y \right> = \left< x,y \right>
            + \left<z, y \right>$ 
        \item $ \left< \left( \alpha \cdot x \right), y \right> = \alpha
            \left< x, y \right>$ 
        \item $ \left< x, y \right> = \left< y,x \right> $ 
        \item $ \left<x,x \right>  $ und $ \left<x,x \right> = 0 \iff x = 0 $
    \end{enumerate}
\end{ibox}

\begin{ibox}{Euklidische Norm}{CDefinition}
   Sei $ z \in \mathbb{R}^n $, dann nennt man die Zahl $ \left| x \right| :=
   \sqrt{ \left<x,x \right>}$, die euklidische Norm von $ x $ . Es folgt 
   \begin{enumerate}[label=\alph*)]
       \item $ \left| x \right| \geq 0  $ und $ \left| x \right| = 0 \iff x = 0$
       \item $ \left| \alpha \cdot x \right|  = \left| \alpha \right| \cdot
           \left| x \right| $ 
       \item $ \left| x + y \right| \leq \left| x \right| + \left| y \right|  $ 
   \end{enumerate}
\end{ibox}

\begin{ibox}{Schwarzsehe Ungleichung}{CTheorem}
    Für alle $ x, y \in \mathbb{R}^{n} $ gilt: $ \left< \left| x,y \right| \right>
    \leq \left| x \right| \left| y \right| $  
    \\ (wurde in LA2 bewiesen)
\end{ibox}
\smalltitle[]{Eigenschaft von Norm auf Vektorraum}
Allgemein nennt man jede Abbildung $ \| \cdot \| : V \to \mathbb{R} $ auf einem $
R $-Vektorraum $ V $ eine Norm auf $ V $, wenn sie die folgende Eigenschaften
haben
\index{Norm}
\index{Dreiecksgleichung}
\begin{enumerate}[label=\alph*)]
    \item $ \forall x \in V; \| x \| \geq 0 \text{ und } \| x \|=0 \iff x=0 $ 
    \item $ \forall x \in V, \forall a \in \mathbb{R} \| \alpha x  \| =
        \left| \alpha \right| \|x \| $ 
	\item $ \forall x, y \in V : \| x+y \| \leq \|x \| + \|y \| $ 
\end{enumerate}
Man nennt dann das Tupel $ \left( V, \| \cdot \| \right)  $ einem nominierten
Vektorraum \\
Dies wurde auch in LA2 bewiesen
\index{nominierter Vektorraum}

\smalltitle{Beispiel}

\begin{enumerate}[label=\alph*)]
    \item $ V := \mathbb{R}^{n} \text{ und sei } \|. \|_{\infty}:$
        $\mathbb{R}^{n} \to \mathbb{R}_{0}^+$ . $ x = \left( x_1,x_2 \dots
        x_{n}\right) \to \|x \|_{\infty}:= max \left\{ \left| x_1 \right| , \left| x_2 \right|, \left| x_{n} \right| \right\}$ . Dann ist $ \|. \|_{
    \infty}$ eine Norm auf $ \mathbb{R}_{n} $ nennt.
    \item Der Vektorraum $ V $ aller linearen Abbildungen $ L : \mathbb{R}^n
        \to \mathbb{R}^m$ wird durch die Definition $ \|L \|:= sum \left\{ 
        \left| L(x)\right| : x \in \mathbb{R}^{n}, \left| x \right| 
    \leq 1 \right\}  $ 
    \item Fehlt
\end{enumerate}

\begin{ibox}{euklidische Metrik}{CDefinition}
    Die Funktion $ d: \mathbb{R}^{n}\times \mathbb{R}^{n} \to \mathbb{R}
, (x,y) \mapsto d(x,y) := |x-y|$ heißt euklidische Metrik auf
$ \mathbb{R}^{n}$
\end{ibox}
Eigenschaften der euklidischen Metrik auf $ \mathbb{R}^{n} $ sind:
\begin{ibox}{Eigenschaften der euklidischen Metrik}{CTheorem}
    $ \forall x,y,z \in \mathbb{R}^{n} $ gibt:
    \begin{enumerate}[label=\alph*)]
        \item $ d(x,y) \geq 0 \text{ und } d(x,y) = 0 \iff x = y $ 
        \item $ d(x,y) = d(y,x) $ (Symmetrie)
        \item $ d(x,y) \geq d(x,y) + d(y,z) $ (Dreiecksgleichung)
    \end{enumerate}
    
\end{ibox}

\begin{ibox}{Metrik auf eine Menge}{CDefinition}
    Eine Metrik auf eine Menge $ A $ ist eine Abbildung $ f: A \times A 
    \to \mathbb{R}_{0}^{+}$ mit der Eigenschaft a)- c) von Satz 1.16 
\end{ibox}
\smalltitle[]{Beispiel}
Sei $ A $ beliebige Menge und  $ f: A \times A \to \mathbb{R}_{0}^{+}$
definiert durch $$ p(x,y) := \begin{cases}
    1, &\text{falls} x\neq y \\
    0, &\text{falls} x = y
\end{cases}
 $$
 \begin{ibox}{Kugel}{CDefinition}
     Sei $ a \in \mathbb{R}^{n} $ und sei $ r>0 $ . Dann heißt die Menge $ 
     B(a,r) := \left\{ x \in \mathbb{R}^{n} :\left| x-a \right|< r\right\} $ als die \textit{Kugel um a mit dem Radius } $ r $ .
 \end{ibox}
\begin{ibox}{Offene und abgeschlossene Menge}{CDefinition}
    Sei $ \mathcal{S} \subset \mathbb{R}^{n}$ eine Menge
    \begin{enumerate}[label=\alph*)]
        \item $ \mathcal{S}  $ heißt  \textit{offen}, wenn $ \forall a \in
            \mathcal{S}\; \exists r > 0 $ sodass $ B(a,r) < \mathcal{S}  $ 
        \item $ \mathcal{S} $ heißt \textit{abgeschlossen }, wenn $ 
            \mathbb{R}^n \ \mathcal{S}  $ offen ist. 
      \end{enumerate}
\end{ibox}
\begin{ibox}{Sätze über offene Menge}{CTheorem}
    \begin{enumerate}[label=\alph*)]
        \item $ \mathbb{R}^n \text{ und } $ die leere Menge $ \emptyset $ 
            sind offen.
        \item Sei $ J \neq \emptyset $ eine beliebige Indexmenge und sei 
            $ \left\{ U_{i} : i \in J \right\}  $ eine Familie offener Menge
            $ U_{i} \subset \mathbb{R}^n $  Dann ist die Menge $ V := 
            \bigcup\limits_{i \in J}^{} U_i $ ebenfalls offen. 
        \item Sei $ m \in \mathbb{N} \text{ und } seine U_1, U_2, \dots 
            , U_{m}$ offene Menge in $ \mathbb{R}^n $ Dann ist die Menge $ 
            W:= \bigcap\limits_{i =1}^{m} U_{i}  $ ebenfalls offen
    \end{enumerate}
\end{ibox}
 \begin{proof}
     \begin{enumerate}[label=\alph*)]
         \item Trivial
         \item Sei $ a \in V \implies \exists i \in J$ sodass $ a \in U_i$
            $ U_{i} $ offen $ \implies r>0 $ sodass $ B(a,r) \subset U_{i}
            \subset V \implies V-offen$ 
         \item Sei $ a \in W \implies a \in U_{i}, i = 1,2 \dots , m $ . $ 
             U_{i}-offen \implies r_{i}>0  $ sodass $ B(a,r_i) \subset U_{i}
            $. Sei $ r:= min\left\{ r_1,r_2,\dots, r_{m} \right\} > 0 $  
            Dann $ \forall i = 1,2, \dots , m $ gilt $ B(a,r) \subset 
            B(a,r_i) \subset V_{i} \implies B(a,r) \subset 
            \bigcap\limits_{i=1}^{m} U_{i} = w \implies w-offen  $ 
     \end{enumerate}
 \end{proof}
 Aus Satz 1.10 und aus der Definition von abgeschlossen Mengen folgt direkt
 \begin{ibox}{Weitere Eigenschaften von Menge}{CDefinition}
     \begin{enumerate}[label=\alph*)]
         \item $ \mathbb{R}^n $ und die leere Menge $ \emptyset $ sind 
             abgeschlossen
         \item Sei $ J \neq \emptyset $ eine beliebige Indexmenge und sei 
             $ \left\{ A_{i} : i \in J \right\}  $ eine Familie
             abgeschlossener Menge $ A_{i} \subset \mathbb{R}^n $ .Dann ist
             die Menge $ A := \bigcap\limits_{i \in J}^{} A_{i} \subset 
             \mathbb{R}^n$ ebenfalls abgeschlossen.
         \item Sei $ m \in \mathbb{N}  $ und seien $ A_1, A_2, \dots A_{m}$
             abgeschlossene Mengen in $ \mathbb{R}^n $. Dann ist die Menge 
             $ L:= \bigcup\limits_{i=1}^{m} A_{i} \subset \mathbb{R}^n  $ 
             ebenfalls abgeschlossen. $ \bigcap\limits_{i \in J }^{}  \left( \mathbb{R}^n \ A_{i} \right) = \mathbb{R}^n
             \bigcup\limits_{i \in J}^{} A_{i} \text{ und } \bigcup\limits_{i \in J}^{} \left( \mathbb{R}^n A_{i} \right) = \mathbb{R}^n
             \bigcap\limits_{i \in J}^{} A_{i} $ 
     \end{enumerate}
 \end{ibox}
 
\begin{ibox}{Umgebung}{CDefinition}
    Ist $ a \in \mathbb{R}^n $ gegeben, so nennt man jede offene Menge $ U \in \mathbb{R}^n $ mit $ a \in U $ eine 
    \textit{offene Umgebung von a}. Für $ \varepsilon > 0 $ bezeichnet man die Kugel $ B(a,\varepsilon) $ 
    auch als \textit{ $\varepsilon$ -Umgebung }von $ a $ 
\end{ibox}

\begin{ibox}{Definition}{CDefinition}
    \begin{enumerate}[label=\alph*)]
        \item Man bezeichnet die Menge $ \overline{A}:= \bigcap B $ mit $ A \subset B \subset \mathbb{R}^n$ , $ B-abgeschlossen $ als den
            \textit{Abschluss} oder \textit{abgeschlossene Hülle von A}
        \item Die Menge $ \mathring{A} \stackrel{\text{u}}{=} uU $ mit $ u \in U$ und $ U-offen $ heißt \textit{offener Kern} von $ A $
        \item Der Rand von $ A $ ist gegeben durch $ \partial A := \overline{A} \setminus \mathring{A} $ 
    \end{enumerate}
\end{ibox}

\smalltitle[]{Bemerkung}
Nach Satz 1.10 und 1.11 ist $ \overline{A} $ stets abgeschlossen und $ \mathring{A} $ stets offen. $ \mathring{A} $ ist die kleinste
abgeschlossene Menge, die $ \overline{A} $ enthält, $ \mathring{A} $ ist die größte in $ A $ enthaltene offene Teilmenge von $ A $ .
Insbesondere gilt $ \mathring{A} \subset A \subset \overline{A} $ 

\smalltitle[]{Beisiele}
\begin{enumerate}[label=\alph*)]
    \item Sei $ A = [0,1) \times [0,1] \subset \mathbb{R}^{2}$. Dann ist $ \overline{A} = (0,1) \times (0,1), \overline{A} = [0,1]$
       $ \times [0,1], \partial A = {0,1} \times [0,1]$
       $  \\ \text{ und } \partial A = {0,1} \times [0,1] \cup [0,1] \times {0,1}$  
    \item $ A = B(0,1) \cup \left( B(0,2) \cup \left( \mathbb{Q} \times \mathbb{Q} \right)  \right) \text{ und } \overline{A} =$
        $\overline{B(0,2)}, \mathring{A} = B(0,1), \partial A = B(0,2) \setminus B(0,1)$
\end{enumerate}
\index{abgeschlossene Hülle}
\begin{ibox}{Eigenschaften der abgeschlossenen Hülle und offener Kern}{CDefinition}
    \newline
    Sei $ A \subset \mathbb{R}^n $ und sei $ x \in \mathbb{R}^n $. Dann gilt:
    \begin{enumerate}[label=\alph*)]
        \item $ x \in \partial A \iff $ jede offene Umgebung des Punktes $ x $ sowohl $ A $ als auch $ \mathbb{R}^n \setminus A $ trifft.
            (Das heißt sowohl mit A, als auch mit $ \mathbb{R}^n \setminus A $ einen nicht leeren Durchschnitt hat.)
        \item $ A \setminus \partial A = \mathring{A}$
        \item $ A \bigcup \partial = \overline{A} $ 
        \item $ \partial A $ ist abgeschlossen.
    \end{enumerate}
\end{ibox}

\begin{ibox}{innerer Punkt}{CDefinition}
   Sei $ A \in \mathbb{R}^n $ .
   \begin{enumerate}[label=\alph*)]
   	\item Man nennt $ x \in A $ einen inneren Punkt der Menge $ A $ , wenn es eine offene Umgebung $ U = U(x) $ des Punktes $ x $ gibt,
		so dass $ U \subset A $ gilt.
    \item Mann nennt $ y \in \mathbb{R}^n $ einen Häufungspunkt der Menge $ A $, wenn in jeder offenen Umgebung $ U = U(y) $ des
		Punktes $ y $ ein von $ y $ verschiedener Punkt der Menge $ A $ liegt, dass heißt wenn gilt:
		$$ \forall U = U(y) \text{ offen } \exists x \in A \bigcup U : x \neq y $$
		
   \end{enumerate}
  Die Menge der Häufungspunkt von $ A $ wird $ HP(A) $ 
\end{ibox}
\index{innerer Punkt}
\index{Häufungsspunkt}

\begin{ibox}[8]{Satz}{CTheorem}
    Sei $ A \subset \mathbb{R}^n $ , dann gilt:
	\begin{enumerate}[label=\alph*)]
		\item $ \mathring{A} = \left\{ x \in A : x \text{ ist innerer Punkt con A} \right\}  $ 
		\item $ \overline{A} = A \bigcup HP(A) $ 
 	\end{enumerate}
\end{ibox}
\begin{proof}
	\begin{enumerate}[label=\alph*)]
		\item Ist $ x \in \mathring{A} $ so it definitions gemäß $ x \in U \bigcup : U \subset A, U-\text{offen} $ 
			also existiert mindestens eine offene Umgebung $ V = V(x) $ des Punktes $ x  $ mit $ V \subset A \implies x $ innerer Punkt
			von $ A $ ist. Ist anderseits $ x $ innerer Punkt von $ A $ , so existiert eine offene Umgebung $ V = V(x) $ des 
			Punktes x mit $ x \in V \subset A \implies $ insbesondere $ x  $ ist dann Element der Vereinigung. 
			$ \bigcup U : U \subset A, U- $ offen, das heißt $ x \in \overline{A} $ .
		\item Wegen $ \overline{A} A \bigcup \partial A = A \bigcup \left( \partial A \setminus A \right) \text{ und } HP(A) \bigcup 
			= A \bigcup (HA(A)\setminus A) $  folgt die Behauptung aus der einfachen Beobachtung, dass alle nicht in $ A $ gelegene
			Randpunkte von $ A $ zwangsläufig Häufungspunkt von $ A $ sind und umgekehrt alle nicht in $ A $ gelegenen Häufungspunkt 
			von $ A $ natürliche Randpunkte von $ A $ sind. 
	\end{enumerate}
	 
\end{proof}
\para{2}{Punktfolgen im $ \mathbb{R}^n $ }
\begin{ibox}{Definition}{CDefinition}
    Sei $ \left( x_k \right) \subset \mathbb{R}^n $ eine Folge von Punkten im $ \mathbb{R}^n $. $ (x_k) $ heißt \textit{konvergent}
	gegeben $ a \in \mathbb{R}^n $ (in Zeichnen: $ \lim_{h \to \infty} x_k = a$ ), wenn zu jeder offenen Umgebung $ U = U(a) $ des 
	Punktes $ a $ ein Index $ k_0 \in \mathbb{N}$ existiert, so dass $ x_n \in U \forall x \geq k_0$ gibt. 
\end{ibox}
\smalltitle[]{Bemerkung}
Definition $ \iff \forall \varepsilon > 0 \exists k_0 \in \mathbb{N} \text{ sodass } x_k \in B(a, \varepsilon)	\forall k \geq k_0 $ 
\begin{ibox}[9]{Satz}{CTheorem}
    Sei $ \left( x_k \right) \subset \mathbb{R}^n $ eine Folge und sei $ a \in \mathbb{R}^n $ ; es seien Komponentenschreibweise 
$ x_k = \left( x_{k_1}, x_{k_2} \cdots x_{k_n} \right) \forall k \in \mathbb{N}, a = (a_1,a_2 \cdots a_{n})  $ , Dann gilt 
$$ \lim_{k \to \infty} x_{k} = a \iff \lim_{k \to \infty}x_{k_{j}} a_{j} \forall j = 1,2 \cdots, n$$
\end{ibox}
\begin{proof}
	"$ \implies $" $\;\; \forall \varepsilon >0  $ sei $ k_0 \in \mathbb{N} $ so gezählt, dass $ x_{k} \in B(a, \varepsilon) \forall k \geq k_0$
	gilt. Für beliebiges $ j \in \left\{ 1, 2, \cdots n \right\}  $ ist dann
	\begin{align*}
		\left| x_{k_{j}} - a_{j} \right| &= \sqrt{\left( x_{k_{j}} - a_{j} \right)^2}\\ 
		& \leq	\sqrt{ \left(  x_{k_{1}} - a _{1} \right)^2 +  \left(  x_{k_{2}} - a _{2}\right)^2 \cdots \left(  x_{k_{n}} - a _{n}\right)^2} \\ 
		& = \left| x_{k} -a \right| \\
		 & < \varepsilon \implies \lim_{ k \to \infty} x_{j} = a  
	\end{align*}
	
$"\impliedby"$ $ \forall \varepsilon > 0 $ wähle man $ \forall j \in \left\{  1,2, \cdots n \right\}$ , so gilt $ \forall k \geq k_0$
\begin{align*}
	 \left| x_{k_{j}} - a_{j} \right| &= \sqrt{ \left(  x_{k_{1}} - a _{1} \right)^2 +  \left(  x_{k_{2}} - a _{2}\right)^2 \cdots 
\left(  x_{k_{n}} - a _{n}\right)^2} \\
&= \sqrt{\left( \frac{ \varepsilon}{\sqrt{n}} \right)^2 + \cdots +  \left( \frac{ \varepsilon}{\sqrt{n}} \right)^2}\\
& = \sqrt{ \varepsilon^2}\\
&= \varepsilon \implies \lim_{k \to \infty} x_{k} = a  
\end{align*}

 
\end{proof}
	
\begin{ibox}{Definition}{CDefinition}
     Sei $ \left( x_k \right) \subset \mathbb{R}^n $ eine Folge. Man nennt $ a \in \mathbb{R}^n $ einen Häufungspunkt der Folge
	 $ \left( x_{k} \right)  $, falls es eine Teilfolge $ \left( x_{k_{y}} \right) \subset \left( x_{k} \right)  $ mit 
	 $ \lim_{y \to \infty }x_{k_{y}} = a  $ gibt.
\end{ibox}
\index{Häufungspunkt}
\begin{ibox}[10]{Satz}{CTheorem}
    Für eine Mengen $ A \subset \mathbb{R}^n $ sind folgende Aussagen äquivalent:
	\begin{enumerate}[label=\alph*)]
		\item $ A $ ist abgeschlossen.	
		\item Der Grenzwert einer jeden Folge $ \left( x_{k} \right) \subset A $ , die als Punktfolge in $ \mathbb{R}^n $ konvergiert,
			liegt in A
	\end{enumerate}
\end{ibox}
\begin{proof}
	$ 1) \implies 2) $ Sei $ A $ abgeschlossen und sei $  \left( x_{k} \right) \subset A $ eine Folge mit $ \lim_{k \to \infty } x_k =
	a \in \mathbb{R}^n	$ . Falls $  x_n $ eine konstante Teilfolge $\left(  x_{k_{ \gamma} } \right) \subset \left( x_k \right)  $,
	so gilt $ x_{k_{ \gamma}} = a \forall \gamma \in \mathbb{N} $, und es folgt $ a \in A $ wegen $ \left( x_{k_{ \gamma}} \right) 
	\subset A  \implies a \in A $ \\
	Hat $ \left( x_k \right) $ keine konstante Teilfolge, so gibt es ein $ k_0 \in \mathbb{N} $ derart, dass $ x_k \neq a \; \forall
	k \geq k_0$ gilt, offenbar ist $ a $  ein Häufungspunkt der Menge $ \left\{ x_{k}: k \geq k_0 \right\} \subset A $ und deshalb auch
	$ a \in HP(A) $ . Nach Satz 8 ist $ HP(A) \subset \overline{A} $, aber $ \overline{A} = A  $, denn $ A  $ is abgeschlossen. 
	Deshalb $ a \in A $ \\
	$ 2) \implies 1) $ Sei $ x \in HP(A) $. Dann gibt es eine Folge $ \left( x_k \right) \subset A, x_k \neq x \; \forall x \in \mathbb{N} $
	, so dass $ x = \lim_{k \to \infty}x_k $ gilt. Nach Voraussetzung liegt der Grenzwert jeder convergente Folge von Punkten aus $ A $ 
	ebenfalls in $ A $, und es folgt $ x \in A $. Dann ist $ HP(A) \subset A $ gezeigt, also ist $ A = A \cup HP(A) = \overline{A} $,
	das heißt $ A-abgeschlossen $ 
\end{proof}

\begin{ibox}{Definition}{CDefinition}
    Eine Folge $ \left( x_k \right)\subset\mathbb{R}^n$ heißt \textit{Cauchy-Folge}, wenn es $ \forall\varepsilon>0 \; \exists\, k_0 
	\in \mathbb{N} \text{ ,s.d } \left|x_n - x_m \right| < \varepsilon \;\forall n,m \geq k_0$ 
\end{ibox}
\begin{ibox}[11]{Satz}{CTheorem}
    Jede konvergente Folge $\left(  x_k  \right) \subset \mathbb{R}^n$ ist eine \textit{Cauchy-Folge}
\end{ibox}

\begin{ibox}[12]{Satz}{CTheorem}
    Sei $ \left( x_k \right) \in \mathbb{R}^n $ eine Folge, es sei $ x_k = \left( x_{k_{1}}, x_{k_{2}},\cdots, x_{k_{n}} \right) \forall \;
	k \in \mathbb{N}$. Dann ist $ \left( x_k \right)  $ eine \textit{Cauchy-Folge} genau dann, wenn jede der Folgen $ x_{k_{j}}, \; j = 1,2
	, \cdots ,n$ eine \textit{Cauchy-Folge} in $ \mathbb{R} $ ist.
\end{ibox}
\begin{proof}
	$ " \implies" \left( x_k \right)  $ ist eine Cauchy-Folge $ \implies \forall \varepsilon > 0 \; \exists k_0 \in \mathbb{N} \text{ s.d } 
	\left| x_k - x_m \right| < \varepsilon \; \forall k,m \geq k_0  \implies \forall  1 \leq j \leq n $  \\
	$ " \impliedby" $ $ \left( x_{k_{j}} \right)  $-eine Cauchy-Folge	$ \forall 1  \leq j \leq \implies \; \forall \varepsilon > 0
	\exists k_{0_{j}} \in \mathbb{N} \text{ s.d } \left| x_{k_{j}} - s_{m_{j}} \right| < \varepsilon \; \forall k,m \geq k_{0_{j}}$.
	Sei $ k_0 := max \left\{ k_{0_{1}}, \cdots, k_{0_{n}} \right\} \in \mathbb{N} $. Dann ist $ \left| x_k - x_m \right| = \sqrt{
	\left( x_{k_1} - x_{m_1} \right)^{2} + \cdots + \left( x_{k_n} - x_{m_n} \right)^{2}} <  
	\sqrt{\left( \frac{ \varepsilon}{\sqrt{n}} \right)^2 + \cdots +  \left( \frac{ \varepsilon}{\sqrt{n}} \right)^2} = \varepsilon \;
	\forall k, m \geq k_0 \implies \left( x_k \right)$-eine Cauchy-Folge.
\end{proof}

\begin{ibox}[13]{Satz}{CTheorem}
    Jede Cauchy-Folge im $ \mathbb{R}^n  $ ist konvergent, das heißt $ \mathbb{R}^n  $ ist vollständig.
\end{ibox}

\para{3}{Funktionen, Abbildungen, Stetigkeit}

\begin{ibox}[]{Definition}{CDefinition}
    Eine Funktion $ f: U \to \mathbb{R}  $ heißt \textit{stetig} and der Stell $ a \in  U $, wenn gilt: $ \forall  \varepsilon >0 
	\text{ s.d } \left| x-a \right| < \delta, x \in  U \implies \left| f(x)-f(a) \right| < \varepsilon $\\
	$ f $ heißt \textit{stetig} $ \left( \text{ auf } U \right)  $, wenn $ f $ in jedem Punkt $ a \in  U $ stetig ist.
\end{ibox}
Wir verallgemeinern diese Stetigkeitsdefinition sogleich auf Abbildungen mit Werten in $ \mathbb{R}^m $. Eine Abbildung $ f : \mathbb{R}^n 
\subset U \to \mathbb{R}^m  $ ist gegeben durch ein $ m- $Tupel $ f = (f_1,f_2,\cdots, f_m) $ von Funktionen $ f_{j}: U \to \mathbb{R} 
, 1 \leq  j \leq m \; \forall  x \in  U, f(x) \in  \mathbb{R}^m, \text{ d.h } f(x) = \left( f(x_1), \cdots, f(x_m) \right) $ 

\begin{ibox}[]{Definition}{CDefinition}
    $ f: \mathbb{R}^n  \subset U \to \mathbb{R}^m $ ist \textit{an der stelle} $ a \in  U $ \textit{stetig}, wenn $ \forall \varepsilon 
	>0 \exists \delta >0 \text{ s.d } x \in  U, \left| x-a \right| < \delta \implies \left| f(x)-f(a) \right| < \varepsilon $ . $ f $ heißt
	\textit{stetig} (auf $ U $ ), wenn $ \forall  a \in U $, $ f \text{ in } a $ stetig ist. 
\end{ibox}

\begin{ibox}[14]{Satz}{CTheorem}
    Seien $ U \subset \mathbb{R}^n  $ offen, $ a \in  U $ und $ f=(f_1,f_2,\cdots, f_m): U \to \mathbb{R}^m  $ ein Abbildung. $ f $ ist 
	stetig in a, genau dann, wenn jede der Komponenten Funktionen $ f_{j} :U \to \mathbb{R} , j = 1,2, \cdots, m $ stetig in $ a $ ist.
\end{ibox}

\begin{proof}
	Der Beweis beruht wie der Beweis des Sates 9 auf der Äquivalenz der Maximumnorm auf $ \mathbb{R}^m $ zur euklidischen Norm auf 
	$ \mathbb{R}^m $, genauer auf der Beziehung
	$$ \|x \|_{\infty}:= max \left\{ \left| x_1 \right| ,\cdots, \left| x_m \right| \leq \sqrt{m} \|x \|_{\infty} \right\} \;  
	\forall  x \in  \mathbb{R}^m$$
	
\end{proof}
\begin{ibox}[15]{Satz}{CTheorem}
    Sei $ U \subset  \mathbb{R}^n  $ offen, sei $ a \in  U $. Eine Abbildung $ f: U \to \mathbb{R}^m $ ist genau dann in a stetig,
	wenn zu jeder offenen Umgebung $ F=V(f(a)) \subset \mathbb{R}^m $ des Punktes $ f(a) \in \mathbb{R}^m $ eine offene Umgebung
	$ W = W(a) \subset  \mathbb{R}^n  $ des Punktes $ a \in  V$ existiert, so dass $ f(W) = \left\{ f(x) |\, x \in U \cap W \right\} 
	\subset  V$ gilt. 
\end{ibox}

\begin{proof}
	"$  \implies$ " $ f $ ist stetig. sei $ V = V \left( f \left( a \right)  \right)  \subset  \mathbb{R}^m $ 
	eine offene Umgebung des Punktes a  $ \implies \exists 
	\varepsilon  >0 \text{ ,s.d } B(f(a), \varepsilon ) \subset  V(f(a))$. $ f $ ist stetig in a $ \implies \exists \delta > 0 
	\text{ ,s.d. } x \in  \underbrace{B(a, \delta) \cap U}_{:=W} $ \\ $\implies f(x) \in B(f(a), \varepsilon ) \subset  V(f(a))$\\ \\
	$ "\impliedby" $ Sei $ \forall \, V(f(a)) \text{-offen, } \exists W(a) \text{-offen s.d. } f(W) \subset  V $ gilt.
	$ \forall \varepsilon >0 \text{ sei } V = N(f(a), \varepsilon ) \\
	\text{ ,dann } \exists W(a) \text{-offen s.d. } f(W) \subset B(f(a), \varepsilon ) \implies 
	\exists  \delta >0 \text{ s.d. } B(a,\delta) \subset  W \\ 
	\implies  f(B(a, \varepsilon ) \cap U \subset B(f(a), \varepsilon ) \implies 
	f$ ist stetig in a   
\end{proof}

\begin{ibox}[16]{Satz}{CTheorem}
    Sei $ U \subset  \mathbb{R}^n  $ offen und sei $ f: U \to \mathbb{R}^m $ eine Abbildung. $ f $ ist genau dann auf $ U  $ stetig,
	wenn das Urbild $ f^{-1}(V) := \left\{ x \in U : f(x) \in V \right\}  $ einer jeden offene Teilmenge $ V \subset  \mathbb{R}^m $ unter
	$ f $ selbst wieder offen ist.
\end{ibox}

\begin{proof}
	$ " \implies" $ $ f $ ist stetig auf $ U $. Sei $ V  $ in $ \mathbb{R}^n  $ offen. Sei $ a \in  f^{-1}(V) \text{ ,d.h. } f(a) 
	\in  V  \xRightarrow{Satz 15} \exists W(a) \subset  \mathbb{R}^n $ offen, s.d $ f(W(a) \cap V) \subset  V \implies  \exists \delta
> 0 \text{ s.d. } B(a, \delta) \subset  W(a) \implies  W(a) \subset  f^{-1}(V) \implies  B(a, \delta) \subset  f^{-1}(V)$ \\
$ " \implies " $ Für $ a \in f^{-1}(V) $ sei $ W(a) := f^{-1}(V) \xRightarrow{Satz 15} f $ stetig in $ a $  
\end{proof}


Für stetige Abbildung des $ \mathbb{R}^n  $ mit Werten in $ \mathbb{R}  $ 
gelten Rechenregeln, die denen für das Rechnen mit stetige Funktionen $ f: \mathbb{R}  \to \mathbb{R}  $ entsprechen und die völlig
analog zu beweisen sind:

\begin{ibox}[17]{Satz}{CTheorem}
    Sei $ U \subset \mathbb{R}^n  $ offen. Die Funktionen $ f : U \to \mathbb{R} \text{ und } g : U \to \mathbb{R}  $ seien beide 
	stetig an der Stelle $ a \in  U $, dann gilt :
	\begin{enumerate}[label=\alph*)]
		\item Die Funktion $ (f+g):U \to \mathbb{R}  $ ist stetig in $ a $ 
		\item Die Funktion $ (f \cdot g) : U \to \mathbb{R}  $ ist stetig in $ a $ 
		\item Falls $ g(a) \neq 0 $ existiert eine offene Umgebung $ V = V(g) \text{ mit } g(x) \neq 0 \; \forall x \in  V $,
			die Funktion $ \left( \frac{f}{g}\right) : V \to \mathbb{R}    $ ist stetig in $ a $.  
	\end{enumerate}
	
\end{ibox}

\smalltitle[]{Bemerkung}
Die Übertragung des Sates 17 auf dem Fall $ \mathbb{R}^m  $-wertiger Abbildungen ist nicht uneingeschränkt möglich. Folgende Version auf 
deren Beweis wie ebenfalls verzichten können, ist aber gültig. 

\begin{ibox}[18]{Satz}{CTheorem}
	Sei $ U \subset \mathbb{R}^n  $ offen, die Abbildung $ f : U \to \mathbb{R}^m $ sei stetig in $ a $. Dann gilt:
	\begin{enumerate}[label=\alph*)]
		\item Ist die Abbildung $ g: V \to \mathbb{R}^m $ stetig in $ a $, so ist auch die Abbildung
			$ (f+g):U \to \mathbb{R}  $ ist stetig in $ a $ .
 		\item Ist die Funktion $ g: U \to \mathbb{R}  $ stetig in $ a $, so ist auch die Abbildung
			 $ (f \cdot g) : U \to \mathbb{R}  $ ist stetig in $ a $ .
		\item Ist die Funktion $ g: U \to \mathbb{R}  $ stetig in $ a  $ und  $ g(a) \neq 0 $, so existiert eine offene Umgebung 
			$ V = V(a) $ mit $ g(x) \neq 0 \; \forall x \in V $ ; die Abbildung 
			$ \left( \frac{f}{g}\right) : V \to \mathbb{R}    $ ist stetig in $ a $.  
	\end{enumerate}
\end{ibox}

Auch die Komposition stetiger Abbildungen liefert wieder eine stetige Abbildung:

\begin{ibox}[19]{Satz}{CTheorem}
    Seien $ U \subset \mathbb{R}^n , \; V \subset \mathbb{R}^m $ offen und seien $ f: U \to V \subset \mathbb{R}^m $ sowie
	$ g: V \to \mathbb{R}^k $ Abbildungen für die gilt: $ f $ ist stetig in $ x_0 \in V, g  $ ist stetig in $ y_0 = f \left( x_0 \right) 
	\in V$ .Dann ist die Abbildung $ \left( g \cdot f \right) : U \xrightarrow{f} V \xrightarrow{g} \mathbb{R}^k  $ stetig in $ x_0 $  
\end{ibox}

\begin{proof}
	Ist $ W = W \left( \left( g \cdot f \right) \left( x_0 \right)  \right)  $ eine offene Umgebung des Punktes $ \left| g \cdot f\right| 
	= g(y_0) \in \mathbb{R}^k $, so gibt es wegen der Stetigkeit von $ g $ an der Stell $ y_0 $ nach Satz 15 eine offene $ y_0 $ -Umgebung
	$ W_1 = W(y_0) = W \left( f \left( x_0 \right)  \right) \subset  \mathbb{R}^m $ mit $ g \left( w_1 \right) \subset W $. Ebenso 
	impliziert die Stetigkeit von $ f $ in $ x_0 $ die Existenz einer offenen Umgebung $ W_2 = W_2 \left( x_0 \right) $ mit
	$ f \left( W_2 \right) \subset W_1 $ . Offenbar ist dann $ \left( g \cdot f \right) \left( W_2 \right) \subset g \left( f 
	\left( W_2 \right) \right) \subset g \left( W_1 \right) \subset W $, und die Stetigkeit von $ \left( g \cdot f \right)  $ in
	$ x_0 $ sit beweisen
\end{proof}
\begin{ibox}[]{Definition}{CDefinition}
    Sei $ D \subset \mathbb{R}^n  $ und sei $ \left( f_{k} \right)  $ ein Folge von Abbildungen $ f_{k}: D \to \mathbb{R}^m, \; k \in 
	\mathbb{N} $ .
	\begin{enumerate}[label=\alph*)]
		\item $ \left( f_{k} \right)  $ heißt auf $ D $  \textit{gleichmäßig konvergent} gegen $ f: D \to \mathbb{R}^n $ , wenn gilt:
			$$ \forall \varepsilon > 0 \; \exists k_0 \in \mathbb{N} \text{ s.d. } \left| f(x) - f_{k}(x) \right| < \varepsilon \; 
			\forall x \in  D, \; \forall k \geq k_0$$
		\item $ \left( f_{k} \right)  $ heißt auf $ D $  \textit{lokal-gleichmäßig konvergent} gegen $ f: D \to \mathbb{R}^n $ , wenn es
			$ \forall a \in D \; \exists V = V(a) \subset  \mathbb{R}^n $-offen s.d. die Folge 
			$ \left( \left. f_{k} \right|_{D \cap V} \right)  $ auf $ D \cap V $ gleichmäßig gegen 
			$ \left. f \right|_{D \cap V}  $ konvergiert.
	\end{enumerate}
\end{ibox}

\smalltitle[]{Bemerkung}
b) $\not \implies$ a) . \textit{Gegenbeispiel}: Sei $ D = (0, + \infty) $ und $ f_{k}(x) := \frac{1}{kx} \text{ für } 1,2 \cdots  $ 
lokal-gleichmäßig konvergiert gegen $ 0 $, aber nicht gleichmäßig.

\begin{ibox}[20]{Satz}{CTheorem}
    Sei $ D \subset  \mathbb{R}^n  $ und sei $ \left( f_k \right)  $ eine Folge stetiger Abbildungen $ f_k : D \to \mathbb{R}^m $ 
	welsche auf $ D $ lokal-gleichmäßig gegen $ f: D \to \mathbb{R}^m $ konvergiert. Dann ist $ f $ stetig auf $ D $ 
\end{ibox}

\para{4}{Kompakte Mengen}

\begin{ibox}[]{Definition}{CDefinition}
    Sei $ \mathcal{S} \subset \mathbb{R}^n , \; \mathcal{S} \neq \emptyset. $ 
	\begin{enumerate}[label=\alph*)]
		\item Unter dem \textit{Durchmesser von $ \mathcal{S}  $ } versteht man die Zahl 
			$$ d \left( \mathcal{S}  \right) := sup \left\{ \left| x-y \right| : \; x,y \in \mathcal{S}  \right\} \leq \infty $$
		\item $ \mathcal{S}  $ heißt \textit{beschränkt}, falls $ d \left( \mathcal{S}  \right) < \infty $ gilt.
	\end{enumerate}
\end{ibox}

\smalltitle[]{Bemerkung}
\begin{enumerate}[label=\alph*)]
	\item Ist $ \mathcal{S} \subset  \mathbb{R}^n  $ beschränkt und ist $ x_0 \in \mathcal{S}  $   gilt $ \mathcal{S} \subset 
		B \left( 0, \left| x_0 \right| + d( \mathcal{S} ) \right)  $, denn $ \forall  y \in \mathcal{S}  $ ist $ |y| = 
		\left| y -x_0 + x_0 \right| \leq \left| y-x_0 \right| + \left| x_0 \right| \leq d \left| \mathcal{S}  \right| + |x|$. 
	\item Ist $ \mathcal{S} \subset B(0,r) $, so folgt $  d \left| \mathcal{S}  \right| \leq 2r $, denn $ \forall x,y \in \mathcal{S}  $ 
		gilt $ \left| x-y \right| \leq |x| + |y| \leq 2r \implies  d \left| \mathcal{S}  \right| \leq 2r $.
	\item Für $ \mathcal{S}_1 \subset \mathcal{S}_2 $ gilt $d \left(  \mathcal{S}_1 \right)  \subset \left(  \mathcal{S}_2 \right) $ 
\end{enumerate}

\begin{ibox}[]{Definition}{CDefinition}
    Sei $ K \subset  \mathbb{R}^n  $ und sie $ J $ eine beliebige (endlich oder unendliche) Indexmenge.
	\begin{enumerate}[label=\alph*)]
		\item Eine Familie $ \left( V_j \right)_{j \in J} $ von offenen Menge $ V_{j} \subset  \mathbb{R}^n  $ heißt (offene)
			Überdeckung von $ K $ wenn $ K \subset  \cup V_{j}$ gibt.
		\item $ K $ heißt \textit{kompakt}, wenn es $ zu jeder $ offenen Überdeckung $ \left( U_{i} \right)_{i \in  I} $ der Menge
			$ K $ endlich viele Indizes $ i_1, \cdots , i_{m} \in  I $ gibt, so dass bereits 
			$$ K \subset  \bigcup_{p = 1}^{m} U_{i_{p}}  $$gibt. Mann nennt ein solches endliches Mengensystem 
			$ \left\{ U_{i_{p}}; \; p = 1,2, \cdots m \right\}  $ offener Menge der Überdeckung $ \left( V_{i} \right)_{i \in  I} $ 
			von $ K $, welsches die Eigenschaft $  K \subset  \bigcup_{p = 1}^{m} U_{i_{p}} $ hat eine $ \left( U_{i}\right)_{i \in  I}   $ 
			zugehörige \textit{offene Teilüberdeckung} der Menge $ \mathbb{R}  $ 
	\end{enumerate}
\end{ibox}

\smalltitle[]{Beispiele}
\begin{enumerate}[label=\alph*)]
	\item Die Menge $ K_1 := \left\{ \frac{1}{n}, \; n \in  \mathbb{N}  \right\} \subset \mathbb{R}  $ nicht kompakt, weil sie nicht
		abgeschlossen ist.
	\item $ K_2 \cup \{0\} $ ist aber kompakt.  
 \end{enumerate}
\begin{ibox}[21]{Definition}{CDefinition}
    Sei $ K \subset \mathbb{R}^n  $. Dann gilt: Ist $ K $ kompakt, so ist $ K $ abgeschlossen und
	beschränkt.
	$$ K \subset  \mathbb{R}^n \text{ kompakt } \iff k \text{ abgeschlossen und beschränkt}  $$
\end{ibox}

\begin{proof}
\begin{enumerate}[label=\alph*)]
	\item \textit{ K ist beschränkt}. Wir betrachten folgende Überdeckung von $ \mathbb{R}^n  $ bei
		offene Menge $ U_{k} := B(0,k), \; k \in \mathbb{N}  $ Dann ist $ K \subset \mathbb{R}^n 
		= \bigcup\limits_{k = 1}^{\infty} U_{k} $. $ K $-kompakt,
		$ \left\{ U_{k} \right\}_{k=1}^{\infty} $ eine offene Überdeckung von $ K \implies \exists 
		k_1, k_2, \cdots k_{m} \text{ s.d. } K \subset  \bigcup\limits_{i=1}^{m} U_{k_{i}} $ .
		Sei $ k^{*} := max \left\{ k_1, k_2, \cdots k_{m} \right\}  $. Dann ist $ K \subset B(0,k^*)$ 
		$ \implies  d(K) < 2k^* \implies k- $beschränkt 
	\item $ " \impliedby" $ \textit{K ist abgeschlossen (Wiederspruchbeweis!)} Sei $ K $ ist nicht
		abgeschlossen. Dann gitb es ein Punkt $ x \in \mathbb{R}^n \setminus K $ eine Folge
		$ \left\{ x_{i} \right\}_{i=1}^\infty $, $ x_{i} \in K \; \forall i \in  \mathbb{N}
		, \lim_{i \to
		\infty} x_i = x$ . Wir betrachten folgende offen Überdeckungen K:
		$ U_i := \mathbb{R}^n  \setminus \overline{B \left( x, \frac{1}{i}  \right) ,
		i = 1, 2 , \cdots}   $. Denn $ \bigcup\limits_{i=1}^{\infty} U_i = \mathbb{R}^n \setminus
		{x} \supset K $ $ \implies \exists i_1,i_2, \cdots i_{m} $ s.d. $ K \subset
		\bigcup\limits_{k=1}^{m} U_{i_{k}} $. Sei $ i^* = max \left\{ i_1,i_2, \cdots i_{m} \right\}$
		Dann ist $ K \subset \mathbb{R}^n \setminus \overline{B \left( x, \frac{1}{i^*}  \right) } $ 
		$ \implies K \cap B \left( x, \frac{1}{i^*}  \right) = \emptyset  $ Wiederspruch.
\end{enumerate}
\end{proof}

\begin{ibox}[22]{Satz}{CTheorem}
    Sei $ K \subset  \mathbb{R}^n  $ kompakt. Dann ist jede abgeschlossene Menge $ A \subset K $ 
	ebenfalls Kompakt.
\end{ibox}
\begin{proof}
	Sei $ \left\{ U_i \right\}_{i \in  I} $ eine offene Überdeckung vom $ A $. Dann ist die Familie
	$ \left\{ \left\{ U_i \right\}_{i \in  I}, \mathbb{R}^n \setminus A \right\}  $- eine Überdeckung
	von $ \mathbb{R}^n \supset K, K-kompakt \implies \exists U_{i_{1}}, \cdots U_{i_{k}} $ 
	s.d. $ K \subset  \bigcup\limits_{k=1}^{m} U_{i_{k}} \cup \left( \mathbb{R}^n \setminus A \right)$
	$ \implies A \subset  \bigcup\limits_{k=1}^{m} U_{i_{k}} \cup \left( \mathbb{R}^n \setminus
	A \right) \implies A \subset \bigcup\limits_{k=1}^{m} U_{i_{k}}   $ (Weil $ A \cap 
	\left( \mathbb{R}^n \setminus A \right) = \emptyset $) $ \implies A- $kompakt. 
\end{proof}

\begin{ibox}[]{Quader}{CDefinition}
    Eine Menge in $ Q \subset \mathbb{R}^n  $ heißt \textit{abgeschlossene Quader}, wenn es 
	beschränkt, abgeschlossene Intervalle $ I_{k} = \left[ a_{k},b_{k} \right] \subset \mathbb{R}^n $ 
	$ , - \infty < a_{k} < b_{k}< + \infty , \; k = 1, 2, \cdots, n $ gibt, so dass   
	$ Q = I_{1} \times I_{2} \times \cdots \times I_{n} $ $ = \left\{ x \in \mathbb{R}^n : 
	a_{k} \leq  x_{k} \leq b_{k}, \; \forall  k = 1,2, \cdots n \right\}  $ gilt.
\end{ibox}

\begin{ibox}[23]{Satz}{CTheorem}
    Jede abgeschlossene Quader $ Q \subset  \mathbb{R}^n  $ ist kompakt.
\end{ibox}
Wir werden folgende Resultat brauchen: \textit{Cantorsche Schachtelungsprinzip} : 

\begin{ibox}[24]{Satz}{CTheorem}
    Sei $ \left( A_{k} \right) $ eine Folge nicht-leerer, abgeschlossene Menge $ A_{k} \subset
	\mathbb{R}^n  $ mit der folgende Eigenschaften:
	\begin{enumerate}[label=\alph*)]
		\item $ \left( A_{k} \right) $ ist absteigend, d.h es gilt $ A_{k+1} \subset A \; \forall 
			k \in  \mathbb{N} $ 
		\item Die Folge $ \left( d \left( A_{k} \right)  \right)  $ der Durchmesser der Menge $ A_k $ 
			ist eine Nullfolge ( d.h $ d \left( A_k \right) \to 0 $ 
	\end{enumerate}
	Dann gibt es genau einen Punkt $ x \in \mathbb{R}^n  $ der allen Menge $ A_{k} $ angehört:
	$$ \bigcap\limits_{k=1}^{\infty} A_k = \left\{ x \right\}   $$
\end{ibox}
\begin{proof}
	\begin{enumerate}[label=\alph*)]
		\item Wir zeigen dass $ \bigcap\limits_{k=1}^{\infty} A_k \neq \emptyset  $ . Sei 
			$ \left( x_{k} \right)_{k =1}^{\infty} $ eine Folge s.d. $ x_{k} \in A_k \; \forall 
			k \in  \mathbb{N} $. Wir wissen, dass $ d \left( A_k \right) \to 0 \text{ wenn } k \to
			\infty $ $ \implies \forall \varepsilon >0 \exists k_0 \in N \text{ s.d. } 
			d \left( A_k \right) < \varepsilon \; $ $ \forall k \geq k_0 $. Dann ist $ \forall 
			k,m \geq k_0 \; \; \; \;$  $  x_{k} \in A_k \subset A_{k_{0}}, \; x_{m} \in A_{m} A_{k_0}$
			$ \implies x_{k}, x_{m} \in A_{k_0} \implies  \|x_{k} - x_{m} \| \leq d 
			\left( A_{k_0} \right) < \varepsilon  \implies \left( x_{k} \right)  $-eine Cauchy-Folge.
			$ \implies \exists x \in \mathbb{R}^n  \text{ s.d. } \lim_{k \to \infty} x_{k} = x $ 
			$ \; \forall x \in \mathbb{N}  $ ist $ x_{m} \in A_{m} \subset A_{k}, \;  m \geq k,
			\; A_{k}$-abgeschlossen $ \implies x = \lim_{m \to \infty}x_{m} \in A_{k} \; \forall k
			\in \mathbb{N} $ $ \implies  x \in  \bigcap\limits_{k=1}^{\infty} A_{k}  $ 
		\item Wir zeigen, dass $ \bigcap\limits_{k=1}^{\infty} A_{k}  $ nur eine Punkt x. 
			(Wiederspruchbeweis). Sei $ x \neq y \in \bigcap\limits_{k=1}^{\infty} A_{k} $. Dann ist 
			$ x,y \in A_{k}, k \in \mathbb{N}  \implies \|x-y \| \leq d \left( A_k \right) \to 0  $ 
			wenn $ k \to \infty \implies  \|x-y \| = 0 \implies x = y  $-Wiederspruch
	\end{enumerate}
\end{proof}

Das Contorsche Schachtelungsprinzip werden wir zum Beweis des Satzes 23 auf eine geeignete absteigende
Folge abgeschl. Quader an: 
Es sei $ Q \subset \mathbb{R}^n  $ eine abgeschl. Quader und $ \left( U_{i} \right)  $ eine offene
Überdeckung von $ Q $. \\
\textit{Annahme: } Zu  $ \left( U_{i} \right)  $ gibt es keine endliche Teilüberdeckung von $ Q $. 
Wir konstruieren induktiv eine absteigende Folge $$ Q = Q_0 \subset Q_1 \subset Q_2 \subset \cdots $$ 
abgeschl. Quader $ Q_{m} \subset \mathbb{R}^n  $ mit der folgenden Eigenschaften:
\begin{enumerate}[label=\alph*)]
	\item $ \forall m \in  \mathbb{N} :  $ Es gibt keine $ \left( U_{i} \right)  $ zugehörige
		endliche Teilüberdeckung von $ Q_{m} $ 
	\item $ d \left( Q_{m} \right)  = \frac{1}{2^{m}} d \left( Q_{m} \right) \left( \implies 
		\lim_{n \to \infty} d \left( Q_{m} \right) = 0 \right) $ 
\end{enumerate}
	 
\begin{center}
	\includesvg[height=3in]{path2.svg}
\end{center}

%\documentclass[a4paper]{memoir}
%
\usepackage[ngerman]{babel}
\usepackage{bookmark}
\usepackage{amsmath}
\usepackage{amssymb}
\usepackage{amsthm}
\usepackage[T1]{fontenc}
\usepackage{imakeidx}
\usepackage{enumitem}
\usepackage{mathtools}

\usepackage{svg}
\usepackage{parskip}
\usepackage{hyperref} % Make TOC clickable

\usepackage{lipsum}
\usepackage{fancyhdr} % Heading customization
\usepackage{geometry} % Adjust page padding 
\usepackage{adjustbox}

\usepackage{xcolor}
\usepackage[most]{tcolorbox}

\definecolor{White}{HTML}{F8F8F3}
\definecolor{Black}{HTML}{292939}
\definecolor{CTheorem}{HTML}{FFFFFF}
\definecolor{CLemma}{HTML}{FDFFED}
\definecolor{CDefinition}{HTML}{EFF9F0}
\definecolor{DarkGray}{HTML}{6B6969}
\definecolor{ImportantBorder}{HTML}{F14747}


% Set up page layout
\geometry{
    a4paper,
    left=2.5cm,
    right=2.5cm,
    top=2.5cm,
    bottom=2.5cm
}
\linespread{1.1}
\makeindex

\pagestyle{fancy}
\fancyhf{} % Clear headre/footer
\fancyheadoffset[LE,RO]{0pt} % Adjust headsep for page number and title
\fancyhead[RE,LO]{}
\fancyhead[RE,RO]{\rightmark} % Add pape header title
\fancyhead[LE,LO]{\thepage}  % Add page number to the left
\renewcommand{\headrulewidth}{0pt} % Delete the line in the header 

% Define the \para command
\newcommand{\para}[2]{
    \clearpage % Start on a new page
    \thispagestyle{empty}% removes the top left page number and top right chapter name
    \begin{center} % Center the chapter title
        \vspace*{11em}
        \Huge   \bfseries \S #1 \quad  #2 % Chapter title with symbol and counter
    \end{center}
    \vspace{11em}
    \phantomsection
    \addcontentsline{toc}{chapter}{#1.\hspace{0.6em} #2} % Add chapter to table of contents
    \fancyhead[RE,RO]{#2} % Add pape header title
}

\newcommand{\nsec}[2]{
    \section*{\large  #1 \hspace{0.3em} #2}
    \phantomsection
    \addcontentsline{toc}{section}{#1 \hspace{0.3em} #2}
}
\newcommand{\smalltitle}[2][]{
    \phantomsection
    \subsubsection*{ #1 \hspace{0.2em} #2}
}
% Theorem env
%\newenvironment{ibox}[3]{
%    \phantomsection
%    \addcontentsline{toc}{section}{#1 \hspace{0.3em} #2}
%    \vspace{1.5em}
%    \begin{tcolorbox}[
%        enhanced jigsaw,
%        colback=#3 ,
%        colframe=DarkGray,
%        drop shadow, 
%		attach boxed title to top center={yshift=-2mm},
%        before upper={\vspace{0.5em}},
%        after upper={\vspace{0.5em}},
%        \textbf{ \large #1 \hspace{0.1em} #2\hspace{0.4em}}
%	]}{%
%    \end{tcolorbox}
%}
\newtcolorbox{ibox}[3][]{
	enhanced jigsaw,
	colback=#3,
	colframe=DarkGray,
	coltitle=Black,
	drop shadow,
	title={\textbf{ \large  #1\hspace{0.4em}#2}},
	before skip = 2.7em,
	attach title to upper,
	after title={:\quad \vspace{0.5em}\ },
	%toc = {#1 \quad* #2}{section}
	%IfValueTF={#1}{\addcontentsline{toc}{section}{#1 \hspace{0.3em}#2}}{}
	code={\addcontentsline{toc}{section}{#1 \hspace{0.3em}#2}}
}


%\begin{document}
Unsere Ausgangsquader $ Q_0 = Q $ hat die geforderten Eigenschaften. Seine 
$ Q = Q_0 \supset Q_1 \supset Q_2 \supset \cdots$ bereits konstruiert. Es gelte $ Q_{m} = I_1 
\times I_2 \times \cdots I_{n} \; , I_{k} = \left[ a_{k}, b_{k} \right]  \subset \mathbb{R} \; 
\forall k = 1,2,\cdots, n$ . Wir splitten jedes der Intervalle $ I_{k} $ in der Intervalllemitte 
auf und erhalten Unterteilung $ I_{k} = I_{k_{1}} \cup I_{k_{2}} $ mit \\
$ I_{k_{1}} = \left[ a_{k}, \frac{a_{k}+ b_{k}}{2} \right] \; , I_{k_{2}} 
= \left[ \frac{a_{k}+ b_{k}}{2}, b_{k} \right] \; , 1 \leq k \leq n $. Die kartesischen Produckte
$ I_{1_{ \gamma_{1}}} \times I_{2_{ \gamma_{2}}} \times \cdots  I_{n_{ \gamma_{n}}} \; ,
\gamma \in {1,2} \; \forall 1 \leq k \leq n $ aller Intervallhälften der $ I_{k} \; , k = 1,2, \cdots n $ unterteilen den  abgeschlossener Quader $ Q_{m} $  in 
in insgesamt $ 2^{n} $  vielen kleinere abgeschlos. Quader  
Nur können wir folgendes Kriterium formulieren:
\begin{ibox}[25]{Satz von Heine-Borel}{CTheorem}
    Für ein Menge $ K \subset  \mathbb{R}^n  $ sind folgende Ausagen äquivalent:
	\begin{enumerate}[label=\alph*)]
		\item $ K $ ist abgeschlossen und beschränkt
		\item $ K $ ist kompakt
	\end{enumerate}
\end{ibox}

\begin{proof}
	Siehe Satz 21.
\end{proof}

\smalltitle[]{Korollar}
Für jede kompakte Teilmenge $ \emptyset \neq A \subset \mathbb{R}  $ gilt:
$$ sup A \in A, \; inf A \in A $$
\begin{proof}
	Da $ A $ als kompakte Menge insbesondere beschränkt ist, existeren $ sup A, \; inf A \in R $.
	Weil sowohl $ sup A  $ als auch $ inf A $ Häufungspunkte geeigneter Folgen in $ A $ sind, 
	impliziert die Abgeschlossenheit von $ A $ in Verbindung mit Satz 8, dass $ sup A \in  A
	\text{ und } inf A \in A $ gilt.
\end{proof}

\begin{ibox}[26]{Satz}{CTheorem}
    Ist $ U \subset  \mathbb{R}^n  $ offen $ K \subset  U $-kompakt und $ f:U \to \mathbb{R}^m $
	-stetige Abbildung, so ist $ f(k) \subset \mathbb{R}^m $ ebenfalls kompakt.
\end{ibox}

\begin{proof}
	Sei $ \left( U_{i} \right)_{i \in  I} $   eine offene Überdeckung von $ f(k) $. Sei 
	$ \forall i \in  I, \; V_{i}:= f^{-1} \left( U_{i} \right) $, $ f- $stetig $ \implies i \in I$
	$ V_{i}- $offen. Die Familie $ \left( V_{i} \right)_{i \in  I} $ ist dine Überdeckungvon $ K $
	, $ K $-kompakt $ \implies  \exists V_{i_{1}}, V_{i_{2}} , \cdots V_{i_{n}} \text{ s.d. } K 
	\subset \bigcup\limits_{p = 1}^{k} U_{i_{p}} \implies f(k) \subset  
	\bigcup\limits_{p = 1}^{k} U_{i_{p}} \implies  f(k) $-kompakt
\end{proof}

\begin{ibox}[27]{Satz}{CTheorem}
    Ist $ U \subset \mathbb{R}^n  $ offen, $ k \subset  U $ kompakt und $ f: U \to \mathbb{R}  $
	eine stetige Funktion auf $ U $, so gibt es Punkte $ p,q \in K $ mit $ f(p) = sup \left\{ 
f(x): x \in  K \right\}, f(q) = inf \left\{ f(x): x \in K \right\}   $ 
\end{ibox}

\begin{proof}
	Laut Satz 26 ist $ A := f(k) \subset  \mathbb{R}  $ kompakt, also im Hinblick auf dem Korollar
	zu Satz 25: $ sup f(k) \in  f(k), \; inf f(k) \in  f(k) $ . Deshalb können wir $ p,q \in K $ 
	mit gewünschten Eigenschaften finden
\end{proof}

\begin{ibox}[]{Definition}{CDefinition}
    Sei $ U \subset  \mathbb{R}^n  $ eine beliebige Teilmenge und $ f: U \to \mathbb{R}^m  $ eine 
	Abbildung. Dann nennt man $ f $ \textit{gleichmäßig stetig} auf $ U $ wenn gilt:
	$$ \forall \varepsilon > 0 \; \exists \delta > 0 \text{ s.d. } \left| x - y \right| < \delta , 
	 \; x,y \in  U \implies  \left| f(x) - f(y) \right| < \varepsilon  $$
\end{ibox}

\begin{ibox}[28]{Satz}{CTheorem}
    Sei $ R \in  \mathbb{R}^n $ kompakt und $ f: K \to \mathbb{R}^m $ eine stetige Abbildung.
	Dann ist $ f $ gleichmäßig stetig auf $ K $ 
\end{ibox}

\begin{proof}
	Sei $ \varepsilon > 0 $ vergegeben. $ \forall a \in  K \; \exists r(a)>0$  mit 
	$ \left| f(x) - f(a) \right| < \frac{ \varepsilon }{2} \; \forall x \in  B(a, r(a)) \cap K $
	(wegen der Stetigkeit von $ f $  in a) Offenbar ist $ K \subset \bigcup\limits_{a \in K}^{} 
	B(a,\frac{r(a)}{2} ) $. $ K $-kompakt $ \implies a_1, a_2, \cdots , c_{k} $ s.d $   K \subset 
	\bigcup\limits_{p = 1}^{} B \left( a_{p}, \frac{r \left( a_{p} \right) }{2}  \right) $.
	Man setz nun $ \delta := \frac{1}{2} min \left\{ r \left( a_{1} \right), \cdots, 
	\left( a_{k} \right) \right\}   $ Seien dann $ x_{1}, x_{2} \in K $ mit 
	$ \left| x_1-x_2 \right|  < \delta $ beliebig gewählt. Wir fixieren ein
	$ i \in  \left\{1,2,\cdots , k  \right\}  $  mit $ x_1 \in  B \left( a_{i},
	\frac{r \left( a_{i} \right) }{2}  \right)  $; dann gilt auch $ \left| x_2 - a_{i} \right| 
	\leq \left| x_2 - x_1 \right| + \left| x_{1}  + a_{i}\right| 
	< \delta + \frac{r(a_i)}{2} < r(a_i) $
	also liegt mit $ x_1 \text{ auch }  x_2 $ in $ B(a_{i}, r \left( a_{i} \right))
	$, und wir erhalten $ \left| f \left( x_1 \right) - f \left( x_2 \right)  \right| \leq 
	\left| f \left( x_1 \right) - f \left( a_{i} \right)   \right| + \left| f \left( a_{i} \right) 
	- f \left( x_2 \right) \right| $ $ < \frac{ \varepsilon }{2} + \frac{ \varepsilon }{2}  $.
	Damit ist die gleichmäßig Stetigkeit von $ f $ auf $ K $ beweisen.
\end{proof}
\para{5}{Partielle Ableitung}
\begin{ibox}[]{Definition}{CDefinition}
    Seien $ U \subset \mathbb{R}^n  $ offen, ferner 
	$$ f:U \to \mathbb{R}  (x_1,x_2,\cdots,x_{n}) \mapsto f(x_1,x_2, \cdots, x_{n} $$
     eine Funktion auf $ U $ und $ a = (a_1,a_2, \cdots, a_{n}) \in U $ 
	 \begin{enumerate}[label=\alph*)]
	 	\item $ f $ heißt in $ a $ \textit{partiell differenzierbar} nach $ x_i $ oder auch \textit{partiell 
			differenzierbar bezüglich der $ i $-ten Koordinate}, wenn der Grenzwert 
			$$	
				 D_{i}f(a): \frac{\partial f}{\partial x_{i}} (a)_i := \\= 
				\lim_{\substack{t \to a_{i} \\ t \neq  a_i}} \frac{f(a_1,\cdots ,
			a_{i-1}, t, a_{i+1}, \cdots , a_n) - f(a_1,\cdots ,a_{i-1}, a_i, a_{i+1}, \cdots , a_n) }{t-a_i}  
			$$	
			dann $ \frac{\partial f}{\partial x_i} (a) $ die \textit{partille Ableitung von $ f $ nach $ x_i $ an der 
			Stelle $ a $ } 
		\item $ f $ heißt in $ a $ partiall differenzierbar, wenn $ f $ in $ a $ nach \textit{ allen} $ x_i, i = 1,\cdots,n$ 
			partiell differenzierbarist. In diesem Fall nennt man den Vektor $ grad(f(a)) = \left( 
			\frac{\partial f}{\partial x_1}(a), \frac{\partial f}{\partial x_2}(a), \cdots ,
			\frac{\partial f}{\partial x_n}(a) \right)  $ den \textit{Gradient von $ f $ in $ a $ } 
		\item $ f $ heißt auf $ U $ \textit{ partiell differenzierbar }, wenn $ f $ partiell differenzierbar in $ a $ 
			für jedes $ a \in  U $ ist.
		\item $ f $ heißt in $ a $ \textit{stetig partiell differenzierbar} , wenn $ f $ auf $ U $ partiell differenzierbar
			ist und zusätzlich alle Ableitungen $ \frac{\partial f}{\partial x_{i}}  = D_i f_{i} U \to \mathbb{R} , 
			1 \leq i \leq  n$, an der Stelle $ a $ stetig sind.
		\item $ f $ heißt \textit{stetig partiell differenzierbar auf $ U $ }, wenn $ f  $ in jedem $ a \in  U $ stetig 
			partiell differenzierbar ist. Für die Menge aller solcher Funktion führen wir die Bezeichnung 
			$$ C^{1}(U) : = \{f: U \to \mathbb{R} : f \text{ ist stetig partiell differenzierbar auf } U\} $$ ein.
	 \end{enumerate}
\end{ibox}
\smalltitle[]{Beispiel}
... 
\begin{ibox}[]{Definition}{CDefinition}
	Seien $ U \subset \mathbb{R}^n  $ offen, ferner 
	$ f:U \to \mathbb{R}  (x_1,x_2,\cdots,x_{n}) \mapsto f(x_1,x_2, \cdots, x_{n} $
	eine Funktion auf $ U $ und $ a = (a_1,a_2, \cdots, a_{n}) \in U $. Seien $ k \in \mathbb{N}  $ und $ i \in {1,\cdots,n }$
	\begin{enumerate}[label=\alph*)]
		\item $ f $ heißt in $ a $ \textit{ $ k $-mal partiell differenzierbar nach $ x_i $  },wenn $ f \; (k-1) $ -mal partiell
			differenzierbar auf $ U $ ist und alle partiell Ableitungen $ (k-1) $-ter Ordnung von $ f $ an der Stell $ a $ partiell
			differenzierbar nach $ x_i $ sind.
		\item f heißt in $ a $ \textit{$ k $-mal partiell differenzierbar}, wenn $ f $ in $ a \; k$-mal partiell differenzierbar ist.
			Nach $ x_j $ für alle $ j \in  {1,2, \cdots, n} $ ist. Die partielle Ableitungen der Ordnung $ k $ von $ f $ in $ a $ 
			sind dann gegeben durch 
			\begin{align*}
				D_{i_k}\left( D_{i_{k-1}} \left( \cdots D_{i_2} \left( D_{i_1} (f) \right) \cdots \right)  \right)(a) &:= 
			\frac{\partial^{k} f}{\partial x_{i_{k}} \partial x_{i_{k-1}} \cdots \partial x_{i_1}} (a) \\
			&:= D_{i_{k}} D_{i_{k-1}} \cdots D_{i_1} f(a) \\
			&:= f_{x_{i_{k}} x_{i_{k-1}} \cdots x_{i_1}}^{\left( k \right) } (a)
			\end{align*}
			
			 wobei die Indizes $ i_{1}, \cdots, i_{k} $ voneinandere unabhängig die Menge $ {1,2,\cdots,n}  $ durchlaufen. 
		\item $ f $ heißt $ k $- \textit{mal partiell differenzierbar auf $ U $ }, wenn $ f $ in jedem $ a \in U \; k-$mal partiell
			differenzierbar sit. Die Funktionen $ D_{i_{k}} \cdots D_{i_1} f:U \to \mathbb{R} , i_1,i_2, \cdots, i_{k} \in{1,\cdots,n}$
			heißen \textit{partiell Ableitung $ k $-te Ordnung von f}.
		\item $ f $ heißt \textit{$ k $-mal stetig partiell differenzierbar auf $ U $ } wenn $ f k- $mal partiell differenzierbar auf 
			$ U $ ist und alle partiellen Ableitungen der Ordnung $ \leq  k $ auf $ U $ stetig sind. Für die Menge aller solcher 
			Funktionen führen wir die Bezeichnung 
			$$ C^{k}(U) : = \{f: U \to \mathbb{R} : f \text{ ist $ k $-mal stetig partiell differenzierbar auf } U \} $$ ein.
	\end{enumerate}
\end{ibox}
\begin{ibox}[29]{Satz}{CTheorem}
    Sei $ U \subset  \mathbb{R}^n  $ offen und sei $ f: U \to \mathbb{R}  $ zweimal partiell differenzierbar auf $ U $; außerhalb 
	seien alle partiellen Ableitungen 2.Ordnung von $ f $ stetig auf $ U $. Dann gilt :
	$$ \frac{\partial^{2} f }{\partial x_{i}  \partial x_{j} } =  \frac{\partial^{2} f }{\partial x_{j}  \partial x_{i} } 
	\forall  i,j \in {1,\cdots, n} $$
\end{ibox}

\smalltitle[]{Korollar}
Seien $ U \subset  \mathbb{R}^n  $ offen, $ k \geq 2 $ und sei $ f \in  C^{k}(U) $. Dann gilt für jedes $ k $-Tupel von Indizes \\
$ \left( i_1,i_2,\cdots,i_{k} \in  {1,2,\cdots,n}^{k} \right)  $ und für jedes Permutation $ \pi: \{1,\cdots,k\} \to \{1,\cdots,k\}: $ 
$$ D_{i_{k}} D_{i_{k-1}} \cdots D_{i_{1}} f = D_{i_{\pi(k)}} D_{i_{\pi(k-1)}} \cdots D_{i_{\pi(1)}}f  $$
\begin{proof}
	Der Beweis erfolgt durch vollständige Induktion über $ k $. Den Induktions Anfang bildet Satz 29 im Induktionsschluss verwendet 
	man die Tatsache, dass sich jede Permutation als Hintereinanderausführung endliche vieler Vertauchungen benachbarter 
	Glider darstellen lässt.

\end{proof}

%\end{document}

%\documentclass[a4paper]{memoir}
%
\usepackage[ngerman]{babel}
\usepackage{bookmark}
\usepackage{amsmath}
\usepackage{amssymb}
\usepackage{amsthm}
\usepackage[T1]{fontenc}
\usepackage{imakeidx}
\usepackage{enumitem}
\usepackage{mathtools}

\usepackage{svg}
\usepackage{parskip}
\usepackage{hyperref} % Make TOC clickable

\usepackage{lipsum}
\usepackage{fancyhdr} % Heading customization
\usepackage{geometry} % Adjust page padding 
\usepackage{adjustbox}

\usepackage{xcolor}
\usepackage[most]{tcolorbox}

\definecolor{White}{HTML}{F8F8F3}
\definecolor{Black}{HTML}{292939}
\definecolor{CTheorem}{HTML}{FFFFFF}
\definecolor{CLemma}{HTML}{FDFFED}
\definecolor{CDefinition}{HTML}{EFF9F0}
\definecolor{DarkGray}{HTML}{6B6969}
\definecolor{ImportantBorder}{HTML}{F14747}


% Set up page layout
\geometry{
    a4paper,
    left=2.5cm,
    right=2.5cm,
    top=2.5cm,
    bottom=2.5cm
}
\linespread{1.1}
\makeindex

\pagestyle{fancy}
\fancyhf{} % Clear headre/footer
\fancyheadoffset[LE,RO]{0pt} % Adjust headsep for page number and title
\fancyhead[RE,LO]{}
\fancyhead[RE,RO]{\rightmark} % Add pape header title
\fancyhead[LE,LO]{\thepage}  % Add page number to the left
\renewcommand{\headrulewidth}{0pt} % Delete the line in the header 

% Define the \para command
\newcommand{\para}[2]{
    \clearpage % Start on a new page
    \thispagestyle{empty}% removes the top left page number and top right chapter name
    \begin{center} % Center the chapter title
        \vspace*{11em}
        \Huge   \bfseries \S #1 \quad  #2 % Chapter title with symbol and counter
    \end{center}
    \vspace{11em}
    \phantomsection
    \addcontentsline{toc}{chapter}{#1.\hspace{0.6em} #2} % Add chapter to table of contents
    \fancyhead[RE,RO]{#2} % Add pape header title
}

\newcommand{\nsec}[2]{
    \section*{\large  #1 \hspace{0.3em} #2}
    \phantomsection
    \addcontentsline{toc}{section}{#1 \hspace{0.3em} #2}
}
\newcommand{\smalltitle}[2][]{
    \phantomsection
    \subsubsection*{ #1 \hspace{0.2em} #2}
}
% Theorem env
%\newenvironment{ibox}[3]{
%    \phantomsection
%    \addcontentsline{toc}{section}{#1 \hspace{0.3em} #2}
%    \vspace{1.5em}
%    \begin{tcolorbox}[
%        enhanced jigsaw,
%        colback=#3 ,
%        colframe=DarkGray,
%        drop shadow, 
%		attach boxed title to top center={yshift=-2mm},
%        before upper={\vspace{0.5em}},
%        after upper={\vspace{0.5em}},
%        \textbf{ \large #1 \hspace{0.1em} #2\hspace{0.4em}}
%	]}{%
%    \end{tcolorbox}
%}
\newtcolorbox{ibox}[3][]{
	enhanced jigsaw,
	colback=#3,
	colframe=DarkGray,
	coltitle=Black,
	drop shadow,
	title={\textbf{ \large  #1\hspace{0.4em}#2}},
	before skip = 2.7em,
	attach title to upper,
	after title={:\quad \vspace{0.5em}\ },
	%toc = {#1 \quad* #2}{section}
	%IfValueTF={#1}{\addcontentsline{toc}{section}{#1 \hspace{0.3em}#2}}{}
	code={\addcontentsline{toc}{section}{#1 \hspace{0.3em}#2}}
}


%\begin{document}
Unsere Ausgangsquader $ Q_0 = Q $ hat die geforderten Eigenschaften. Seine 
$ Q = Q_0 \supset Q_1 \supset Q_2 \supset \cdots$ bereits konstruiert. Es gelte $ Q_{m} = I_1 
\times I_2 \times \cdots I_{n} \; , I_{k} = \left[ a_{k}, b_{k} \right]  \subset \mathbb{R} \; 
\forall k = 1,2,\cdots, n$ . Wir splitten jedes der Intervalle $ I_{k} $ in der Intervalllemitte 
auf und erhalten Unterteilung $ I_{k} = I_{k_{1}} \cup I_{k_{2}} $ mit \\
$ I_{k_{1}} = \left[ a_{k}, \frac{a_{k}+ b_{k}}{2} \right] \; , I_{k_{2}} 
= \left[ \frac{a_{k}+ b_{k}}{2}, b_{k} \right] \; , 1 \leq k \leq n $. Die kartesischen Produckte
$ I_{1_{ \gamma_{1}}} \times I_{2_{ \gamma_{2}}} \times \cdots  I_{n_{ \gamma_{n}}} \; ,
\gamma \in {1,2} \; \forall 1 \leq k \leq n $ aller Intervallhälften der $ I_{k} \; , k = 1,2, \cdots n $ unterteilen den  abgeschlossener Quader $ Q_{m} $  in 
in insgesamt $ 2^{n} $  vielen kleinere abgeschlos. Quader  
Nur können wir folgendes Kriterium formulieren:
\begin{ibox}[25]{Satz von Heine-Borel}{CTheorem}
    Für ein Menge $ K \subset  \mathbb{R}^n  $ sind folgende Ausagen äquivalent:
	\begin{enumerate}[label=\alph*)]
		\item $ K $ ist abgeschlossen und beschränkt
		\item $ K $ ist kompakt
	\end{enumerate}
\end{ibox}

\begin{proof}
	Siehe Satz 21.
\end{proof}

\smalltitle[]{Korollar}
Für jede kompakte Teilmenge $ \emptyset \neq A \subset \mathbb{R}  $ gilt:
$$ sup A \in A, \; inf A \in A $$
\begin{proof}
	Da $ A $ als kompakte Menge insbesondere beschränkt ist, existeren $ sup A, \; inf A \in R $.
	Weil sowohl $ sup A  $ als auch $ inf A $ Häufungspunkte geeigneter Folgen in $ A $ sind, 
	impliziert die Abgeschlossenheit von $ A $ in Verbindung mit Satz 8, dass $ sup A \in  A
	\text{ und } inf A \in A $ gilt.
\end{proof}

\begin{ibox}[26]{Satz}{CTheorem}
    Ist $ U \subset  \mathbb{R}^n  $ offen $ K \subset  U $-kompakt und $ f:U \to \mathbb{R}^m $
	-stetige Abbildung, so ist $ f(k) \subset \mathbb{R}^m $ ebenfalls kompakt.
\end{ibox}

\begin{proof}
	Sei $ \left( U_{i} \right)_{i \in  I} $   eine offene Überdeckung von $ f(k) $. Sei 
	$ \forall i \in  I, \; V_{i}:= f^{-1} \left( U_{i} \right) $, $ f- $stetig $ \implies i \in I$
	$ V_{i}- $offen. Die Familie $ \left( V_{i} \right)_{i \in  I} $ ist dine Überdeckungvon $ K $
	, $ K $-kompakt $ \implies  \exists V_{i_{1}}, V_{i_{2}} , \cdots V_{i_{n}} \text{ s.d. } K 
	\subset \bigcup\limits_{p = 1}^{k} U_{i_{p}} \implies f(k) \subset  
	\bigcup\limits_{p = 1}^{k} U_{i_{p}} \implies  f(k) $-kompakt
\end{proof}

\begin{ibox}[27]{Satz}{CTheorem}
    Ist $ U \subset \mathbb{R}^n  $ offen, $ k \subset  U $ kompakt und $ f: U \to \mathbb{R}  $
	eine stetige Funktion auf $ U $, so gibt es Punkte $ p,q \in K $ mit $ f(p) = sup \left\{ 
f(x): x \in  K \right\}, f(q) = inf \left\{ f(x): x \in K \right\}   $ 
\end{ibox}

\begin{proof}
	Laut Satz 26 ist $ A := f(k) \subset  \mathbb{R}  $ kompakt, also im Hinblick auf dem Korollar
	zu Satz 25: $ sup f(k) \in  f(k), \; inf f(k) \in  f(k) $ . Deshalb können wir $ p,q \in K $ 
	mit gewünschten Eigenschaften finden
\end{proof}

\begin{ibox}[]{Definition}{CDefinition}
    Sei $ U \subset  \mathbb{R}^n  $ eine beliebige Teilmenge und $ f: U \to \mathbb{R}^m  $ eine 
	Abbildung. Dann nennt man $ f $ \textit{gleichmäßig stetig} auf $ U $ wenn gilt:
	$$ \forall \varepsilon > 0 \; \exists \delta > 0 \text{ s.d. } \left| x - y \right| < \delta , 
	 \; x,y \in  U \implies  \left| f(x) - f(y) \right| < \varepsilon  $$
\end{ibox}

\begin{ibox}[28]{Satz}{CTheorem}
    Sei $ R \in  \mathbb{R}^n $ kompakt und $ f: K \to \mathbb{R}^m $ eine stetige Abbildung.
	Dann ist $ f $ gleichmäßig stetig auf $ K $ 
\end{ibox}

\begin{proof}
	Sei $ \varepsilon > 0 $ vergegeben. $ \forall a \in  K \; \exists r(a)>0$  mit 
	$ \left| f(x) - f(a) \right| < \frac{ \varepsilon }{2} \; \forall x \in  B(a, r(a)) \cap K $
	(wegen der Stetigkeit von $ f $  in a) Offenbar ist $ K \subset \bigcup\limits_{a \in K}^{} 
	B(a,\frac{r(a)}{2} ) $. $ K $-kompakt $ \implies a_1, a_2, \cdots , c_{k} $ s.d $   K \subset 
	\bigcup\limits_{p = 1}^{} B \left( a_{p}, \frac{r \left( a_{p} \right) }{2}  \right) $.
	Man setz nun $ \delta := \frac{1}{2} min \left\{ r \left( a_{1} \right), \cdots, 
	\left( a_{k} \right) \right\}   $ Seien dann $ x_{1}, x_{2} \in K $ mit 
	$ \left| x_1-x_2 \right|  < \delta $ beliebig gewählt. Wir fixieren ein
	$ i \in  \left\{1,2,\cdots , k  \right\}  $  mit $ x_1 \in  B \left( a_{i},
	\frac{r \left( a_{i} \right) }{2}  \right)  $; dann gilt auch $ \left| x_2 - a_{i} \right| 
	\leq \left| x_2 - x_1 \right| + \left| x_{1}  + a_{i}\right| 
	< \delta + \frac{r(a_i)}{2} < r(a_i) $
	also liegt mit $ x_1 \text{ auch }  x_2 $ in $ B(a_{i}, r \left( a_{i} \right))
	$, und wir erhalten $ \left| f \left( x_1 \right) - f \left( x_2 \right)  \right| \leq 
	\left| f \left( x_1 \right) - f \left( a_{i} \right)   \right| + \left| f \left( a_{i} \right) 
	- f \left( x_2 \right) \right| $ $ < \frac{ \varepsilon }{2} + \frac{ \varepsilon }{2}  $.
	Damit ist die gleichmäßig Stetigkeit von $ f $ auf $ K $ beweisen.
\end{proof}
\para{5}{Partielle Ableitung}
\begin{ibox}[]{Definition}{CDefinition}
    Seien $ U \subset \mathbb{R}^n  $ offen, ferner 
	$$ f:U \to \mathbb{R}  (x_1,x_2,\cdots,x_{n}) \mapsto f(x_1,x_2, \cdots, x_{n} $$
     eine Funktion auf $ U $ und $ a = (a_1,a_2, \cdots, a_{n}) \in U $ 
	 \begin{enumerate}[label=\alph*)]
	 	\item $ f $ heißt in $ a $ \textit{partiell differenzierbar} nach $ x_i $ oder auch \textit{partiell 
			differenzierbar bezüglich der $ i $-ten Koordinate}, wenn der Grenzwert 
			$$	
				 D_{i}f(a): \frac{\partial f}{\partial x_{i}} (a)_i := \\= 
				\lim_{\substack{t \to a_{i} \\ t \neq  a_i}} \frac{f(a_1,\cdots ,
			a_{i-1}, t, a_{i+1}, \cdots , a_n) - f(a_1,\cdots ,a_{i-1}, a_i, a_{i+1}, \cdots , a_n) }{t-a_i}  
			$$	
			dann $ \frac{\partial f}{\partial x_i} (a) $ die \textit{partille Ableitung von $ f $ nach $ x_i $ an der 
			Stelle $ a $ } 
		\item $ f $ heißt in $ a $ partiall differenzierbar, wenn $ f $ in $ a $ nach \textit{ allen} $ x_i, i = 1,\cdots,n$ 
			partiell differenzierbarist. In diesem Fall nennt man den Vektor $ grad(f(a)) = \left( 
			\frac{\partial f}{\partial x_1}(a), \frac{\partial f}{\partial x_2}(a), \cdots ,
			\frac{\partial f}{\partial x_n}(a) \right)  $ den \textit{Gradient von $ f $ in $ a $ } 
		\item $ f $ heißt auf $ U $ \textit{ partiell differenzierbar }, wenn $ f $ partiell differenzierbar in $ a $ 
			für jedes $ a \in  U $ ist.
		\item $ f $ heißt in $ a $ \textit{stetig partiell differenzierbar} , wenn $ f $ auf $ U $ partiell differenzierbar
			ist und zusätzlich alle Ableitungen $ \frac{\partial f}{\partial x_{i}}  = D_i f_{i} U \to \mathbb{R} , 
			1 \leq i \leq  n$, an der Stelle $ a $ stetig sind.
		\item $ f $ heißt \textit{stetig partiell differenzierbar auf $ U $ }, wenn $ f  $ in jedem $ a \in  U $ stetig 
			partiell differenzierbar ist. Für die Menge aller solcher Funktion führen wir die Bezeichnung 
			$$ C^{1}(U) : = \{f: U \to \mathbb{R} : f \text{ ist stetig partiell differenzierbar auf } U\} $$ ein.
	 \end{enumerate}
\end{ibox}
\smalltitle[]{Beispiel}
... 
\begin{ibox}[]{Definition}{CDefinition}
	Seien $ U \subset \mathbb{R}^n  $ offen, ferner 
	$ f:U \to \mathbb{R}  (x_1,x_2,\cdots,x_{n}) \mapsto f(x_1,x_2, \cdots, x_{n} $
	eine Funktion auf $ U $ und $ a = (a_1,a_2, \cdots, a_{n}) \in U $. Seien $ k \in \mathbb{N}  $ und $ i \in {1,\cdots,n }$
	\begin{enumerate}[label=\alph*)]
		\item $ f $ heißt in $ a $ \textit{ $ k $-mal partiell differenzierbar nach $ x_i $  },wenn $ f \; (k-1) $ -mal partiell
			differenzierbar auf $ U $ ist und alle partiell Ableitungen $ (k-1) $-ter Ordnung von $ f $ an der Stell $ a $ partiell
			differenzierbar nach $ x_i $ sind.
		\item f heißt in $ a $ \textit{$ k $-mal partiell differenzierbar}, wenn $ f $ in $ a \; k$-mal partiell differenzierbar ist.
			Nach $ x_j $ für alle $ j \in  {1,2, \cdots, n} $ ist. Die partielle Ableitungen der Ordnung $ k $ von $ f $ in $ a $ 
			sind dann gegeben durch 
			\begin{align*}
				D_{i_k}\left( D_{i_{k-1}} \left( \cdots D_{i_2} \left( D_{i_1} (f) \right) \cdots \right)  \right)(a) &:= 
			\frac{\partial^{k} f}{\partial x_{i_{k}} \partial x_{i_{k-1}} \cdots \partial x_{i_1}} (a) \\
			&:= D_{i_{k}} D_{i_{k-1}} \cdots D_{i_1} f(a) \\
			&:= f_{x_{i_{k}} x_{i_{k-1}} \cdots x_{i_1}}^{\left( k \right) } (a)
			\end{align*}
			
			 wobei die Indizes $ i_{1}, \cdots, i_{k} $ voneinandere unabhängig die Menge $ {1,2,\cdots,n}  $ durchlaufen. 
		\item $ f $ heißt $ k $- \textit{mal partiell differenzierbar auf $ U $ }, wenn $ f $ in jedem $ a \in U \; k-$mal partiell
			differenzierbar sit. Die Funktionen $ D_{i_{k}} \cdots D_{i_1} f:U \to \mathbb{R} , i_1,i_2, \cdots, i_{k} \in{1,\cdots,n}$
			heißen \textit{partiell Ableitung $ k $-te Ordnung von f}.
		\item $ f $ heißt \textit{$ k $-mal stetig partiell differenzierbar auf $ U $ } wenn $ f k- $mal partiell differenzierbar auf 
			$ U $ ist und alle partiellen Ableitungen der Ordnung $ \leq  k $ auf $ U $ stetig sind. Für die Menge aller solcher 
			Funktionen führen wir die Bezeichnung 
			$$ C^{k}(U) : = \{f: U \to \mathbb{R} : f \text{ ist $ k $-mal stetig partiell differenzierbar auf } U \} $$ ein.
	\end{enumerate}
\end{ibox}
\begin{ibox}[29]{Satz}{CTheorem}
    Sei $ U \subset  \mathbb{R}^n  $ offen und sei $ f: U \to \mathbb{R}  $ zweimal partiell differenzierbar auf $ U $; außerhalb 
	seien alle partiellen Ableitungen 2.Ordnung von $ f $ stetig auf $ U $. Dann gilt :
	$$ \frac{\partial^{2} f }{\partial x_{i}  \partial x_{j} } =  \frac{\partial^{2} f }{\partial x_{j}  \partial x_{i} } 
	\forall  i,j \in {1,\cdots, n} $$
\end{ibox}

\smalltitle[]{Korollar}
Seien $ U \subset  \mathbb{R}^n  $ offen, $ k \geq 2 $ und sei $ f \in  C^{k}(U) $. Dann gilt für jedes $ k $-Tupel von Indizes \\
$ \left( i_1,i_2,\cdots,i_{k} \in  {1,2,\cdots,n}^{k} \right)  $ und für jedes Permutation $ \pi: \{1,\cdots,k\} \to \{1,\cdots,k\}: $ 
$$ D_{i_{k}} D_{i_{k-1}} \cdots D_{i_{1}} f = D_{i_{\pi(k)}} D_{i_{\pi(k-1)}} \cdots D_{i_{\pi(1)}}f  $$
\begin{proof}
	Der Beweis erfolgt durch vollständige Induktion über $ k $. Den Induktions Anfang bildet Satz 29 im Induktionsschluss verwendet 
	man die Tatsache, dass sich jede Permutation als Hintereinanderausführung endliche vieler Vertauchungen benachbarter 
	Glider darstellen lässt.

\end{proof}

%\end{document}

%\documentclass[a4paper]{memoir}
%
\usepackage[ngerman]{babel}
\usepackage{bookmark}
\usepackage{amsmath}
\usepackage{amssymb}
\usepackage{amsthm}
\usepackage[T1]{fontenc}
\usepackage{imakeidx}
\usepackage{enumitem}
\usepackage{mathtools}

\usepackage{svg}
\usepackage{parskip}
\usepackage{hyperref} % Make TOC clickable

\usepackage{lipsum}
\usepackage{fancyhdr} % Heading customization
\usepackage{geometry} % Adjust page padding 
\usepackage{adjustbox}

\usepackage{xcolor}
\usepackage[most]{tcolorbox}

\definecolor{White}{HTML}{F8F8F3}
\definecolor{Black}{HTML}{292939}
\definecolor{CTheorem}{HTML}{FFFFFF}
\definecolor{CLemma}{HTML}{FDFFED}
\definecolor{CDefinition}{HTML}{EFF9F0}
\definecolor{DarkGray}{HTML}{6B6969}
\definecolor{ImportantBorder}{HTML}{F14747}


% Set up page layout
\geometry{
    a4paper,
    left=2.5cm,
    right=2.5cm,
    top=2.5cm,
    bottom=2.5cm
}
\linespread{1.1}
\makeindex

\pagestyle{fancy}
\fancyhf{} % Clear headre/footer
\fancyheadoffset[LE,RO]{0pt} % Adjust headsep for page number and title
\fancyhead[RE,LO]{}
\fancyhead[RE,RO]{\rightmark} % Add pape header title
\fancyhead[LE,LO]{\thepage}  % Add page number to the left
\renewcommand{\headrulewidth}{0pt} % Delete the line in the header 

% Define the \para command
\newcommand{\para}[2]{
    \clearpage % Start on a new page
    \thispagestyle{empty}% removes the top left page number and top right chapter name
    \begin{center} % Center the chapter title
        \vspace*{11em}
        \Huge   \bfseries \S #1 \quad  #2 % Chapter title with symbol and counter
    \end{center}
    \vspace{11em}
    \phantomsection
    \addcontentsline{toc}{chapter}{#1.\hspace{0.6em} #2} % Add chapter to table of contents
    \fancyhead[RE,RO]{#2} % Add pape header title
}

\newcommand{\nsec}[2]{
    \section*{\large  #1 \hspace{0.3em} #2}
    \phantomsection
    \addcontentsline{toc}{section}{#1 \hspace{0.3em} #2}
}
\newcommand{\smalltitle}[2][]{
    \phantomsection
    \subsubsection*{ #1 \hspace{0.2em} #2}
}
% Theorem env
%\newenvironment{ibox}[3]{
%    \phantomsection
%    \addcontentsline{toc}{section}{#1 \hspace{0.3em} #2}
%    \vspace{1.5em}
%    \begin{tcolorbox}[
%        enhanced jigsaw,
%        colback=#3 ,
%        colframe=DarkGray,
%        drop shadow, 
%		attach boxed title to top center={yshift=-2mm},
%        before upper={\vspace{0.5em}},
%        after upper={\vspace{0.5em}},
%        \textbf{ \large #1 \hspace{0.1em} #2\hspace{0.4em}}
%	]}{%
%    \end{tcolorbox}
%}
\newtcolorbox{ibox}[3][]{
	enhanced jigsaw,
	colback=#3,
	colframe=DarkGray,
	coltitle=Black,
	drop shadow,
	title={\textbf{ \large  #1\hspace{0.4em}#2}},
	before skip = 2.7em,
	attach title to upper,
	after title={:\quad \vspace{0.5em}\ },
	%toc = {#1 \quad* #2}{section}
	%IfValueTF={#1}{\addcontentsline{toc}{section}{#1 \hspace{0.3em}#2}}{}
	code={\addcontentsline{toc}{section}{#1 \hspace{0.3em}#2}}
}


%\begin{document}
Unsere Ausgangsquader $ Q_0 = Q $ hat die geforderten Eigenschaften. Seine 
$ Q = Q_0 \supset Q_1 \supset Q_2 \supset \cdots$ bereits konstruiert. Es gelte $ Q_{m} = I_1 
\times I_2 \times \cdots I_{n} \; , I_{k} = \left[ a_{k}, b_{k} \right]  \subset \mathbb{R} \; 
\forall k = 1,2,\cdots, n$ . Wir splitten jedes der Intervalle $ I_{k} $ in der Intervalllemitte 
auf und erhalten Unterteilung $ I_{k} = I_{k_{1}} \cup I_{k_{2}} $ mit \\
$ I_{k_{1}} = \left[ a_{k}, \frac{a_{k}+ b_{k}}{2} \right] \; , I_{k_{2}} 
= \left[ \frac{a_{k}+ b_{k}}{2}, b_{k} \right] \; , 1 \leq k \leq n $. Die kartesischen Produckte
$ I_{1_{ \gamma_{1}}} \times I_{2_{ \gamma_{2}}} \times \cdots  I_{n_{ \gamma_{n}}} \; ,
\gamma \in {1,2} \; \forall 1 \leq k \leq n $ aller Intervallhälften der $ I_{k} \; , k = 1,2, \cdots n $ unterteilen den  abgeschlossener Quader $ Q_{m} $  in 
in insgesamt $ 2^{n} $  vielen kleinere abgeschlos. Quader  
Nur können wir folgendes Kriterium formulieren:
\begin{ibox}[25]{Satz von Heine-Borel}{CTheorem}
    Für ein Menge $ K \subset  \mathbb{R}^n  $ sind folgende Ausagen äquivalent:
	\begin{enumerate}[label=\alph*)]
		\item $ K $ ist abgeschlossen und beschränkt
		\item $ K $ ist kompakt
	\end{enumerate}
\end{ibox}

\begin{proof}
	Siehe Satz 21.
\end{proof}

\smalltitle[]{Korollar}
Für jede kompakte Teilmenge $ \emptyset \neq A \subset \mathbb{R}  $ gilt:
$$ sup A \in A, \; inf A \in A $$
\begin{proof}
	Da $ A $ als kompakte Menge insbesondere beschränkt ist, existeren $ sup A, \; inf A \in R $.
	Weil sowohl $ sup A  $ als auch $ inf A $ Häufungspunkte geeigneter Folgen in $ A $ sind, 
	impliziert die Abgeschlossenheit von $ A $ in Verbindung mit Satz 8, dass $ sup A \in  A
	\text{ und } inf A \in A $ gilt.
\end{proof}

\begin{ibox}[26]{Satz}{CTheorem}
    Ist $ U \subset  \mathbb{R}^n  $ offen $ K \subset  U $-kompakt und $ f:U \to \mathbb{R}^m $
	-stetige Abbildung, so ist $ f(k) \subset \mathbb{R}^m $ ebenfalls kompakt.
\end{ibox}

\begin{proof}
	Sei $ \left( U_{i} \right)_{i \in  I} $   eine offene Überdeckung von $ f(k) $. Sei 
	$ \forall i \in  I, \; V_{i}:= f^{-1} \left( U_{i} \right) $, $ f- $stetig $ \implies i \in I$
	$ V_{i}- $offen. Die Familie $ \left( V_{i} \right)_{i \in  I} $ ist dine Überdeckungvon $ K $
	, $ K $-kompakt $ \implies  \exists V_{i_{1}}, V_{i_{2}} , \cdots V_{i_{n}} \text{ s.d. } K 
	\subset \bigcup\limits_{p = 1}^{k} U_{i_{p}} \implies f(k) \subset  
	\bigcup\limits_{p = 1}^{k} U_{i_{p}} \implies  f(k) $-kompakt
\end{proof}

\begin{ibox}[27]{Satz}{CTheorem}
    Ist $ U \subset \mathbb{R}^n  $ offen, $ k \subset  U $ kompakt und $ f: U \to \mathbb{R}  $
	eine stetige Funktion auf $ U $, so gibt es Punkte $ p,q \in K $ mit $ f(p) = sup \left\{ 
f(x): x \in  K \right\}, f(q) = inf \left\{ f(x): x \in K \right\}   $ 
\end{ibox}

\begin{proof}
	Laut Satz 26 ist $ A := f(k) \subset  \mathbb{R}  $ kompakt, also im Hinblick auf dem Korollar
	zu Satz 25: $ sup f(k) \in  f(k), \; inf f(k) \in  f(k) $ . Deshalb können wir $ p,q \in K $ 
	mit gewünschten Eigenschaften finden
\end{proof}

\begin{ibox}[]{Definition}{CDefinition}
    Sei $ U \subset  \mathbb{R}^n  $ eine beliebige Teilmenge und $ f: U \to \mathbb{R}^m  $ eine 
	Abbildung. Dann nennt man $ f $ \textit{gleichmäßig stetig} auf $ U $ wenn gilt:
	$$ \forall \varepsilon > 0 \; \exists \delta > 0 \text{ s.d. } \left| x - y \right| < \delta , 
	 \; x,y \in  U \implies  \left| f(x) - f(y) \right| < \varepsilon  $$
\end{ibox}

\begin{ibox}[28]{Satz}{CTheorem}
    Sei $ R \in  \mathbb{R}^n $ kompakt und $ f: K \to \mathbb{R}^m $ eine stetige Abbildung.
	Dann ist $ f $ gleichmäßig stetig auf $ K $ 
\end{ibox}

\begin{proof}
	Sei $ \varepsilon > 0 $ vergegeben. $ \forall a \in  K \; \exists r(a)>0$  mit 
	$ \left| f(x) - f(a) \right| < \frac{ \varepsilon }{2} \; \forall x \in  B(a, r(a)) \cap K $
	(wegen der Stetigkeit von $ f $  in a) Offenbar ist $ K \subset \bigcup\limits_{a \in K}^{} 
	B(a,\frac{r(a)}{2} ) $. $ K $-kompakt $ \implies a_1, a_2, \cdots , c_{k} $ s.d $   K \subset 
	\bigcup\limits_{p = 1}^{} B \left( a_{p}, \frac{r \left( a_{p} \right) }{2}  \right) $.
	Man setz nun $ \delta := \frac{1}{2} min \left\{ r \left( a_{1} \right), \cdots, 
	\left( a_{k} \right) \right\}   $ Seien dann $ x_{1}, x_{2} \in K $ mit 
	$ \left| x_1-x_2 \right|  < \delta $ beliebig gewählt. Wir fixieren ein
	$ i \in  \left\{1,2,\cdots , k  \right\}  $  mit $ x_1 \in  B \left( a_{i},
	\frac{r \left( a_{i} \right) }{2}  \right)  $; dann gilt auch $ \left| x_2 - a_{i} \right| 
	\leq \left| x_2 - x_1 \right| + \left| x_{1}  + a_{i}\right| 
	< \delta + \frac{r(a_i)}{2} < r(a_i) $
	also liegt mit $ x_1 \text{ auch }  x_2 $ in $ B(a_{i}, r \left( a_{i} \right))
	$, und wir erhalten $ \left| f \left( x_1 \right) - f \left( x_2 \right)  \right| \leq 
	\left| f \left( x_1 \right) - f \left( a_{i} \right)   \right| + \left| f \left( a_{i} \right) 
	- f \left( x_2 \right) \right| $ $ < \frac{ \varepsilon }{2} + \frac{ \varepsilon }{2}  $.
	Damit ist die gleichmäßig Stetigkeit von $ f $ auf $ K $ beweisen.
\end{proof}
\para{5}{Partielle Ableitung}
\begin{ibox}[]{Definition}{CDefinition}
    Seien $ U \subset \mathbb{R}^n  $ offen, ferner 
	$$ f:U \to \mathbb{R}  (x_1,x_2,\cdots,x_{n}) \mapsto f(x_1,x_2, \cdots, x_{n} $$
     eine Funktion auf $ U $ und $ a = (a_1,a_2, \cdots, a_{n}) \in U $ 
	 \begin{enumerate}[label=\alph*)]
	 	\item $ f $ heißt in $ a $ \textit{partiell differenzierbar} nach $ x_i $ oder auch \textit{partiell 
			differenzierbar bezüglich der $ i $-ten Koordinate}, wenn der Grenzwert 
			$$	
				 D_{i}f(a): \frac{\partial f}{\partial x_{i}} (a)_i := \\= 
				\lim_{\substack{t \to a_{i} \\ t \neq  a_i}} \frac{f(a_1,\cdots ,
			a_{i-1}, t, a_{i+1}, \cdots , a_n) - f(a_1,\cdots ,a_{i-1}, a_i, a_{i+1}, \cdots , a_n) }{t-a_i}  
			$$	
			dann $ \frac{\partial f}{\partial x_i} (a) $ die \textit{partille Ableitung von $ f $ nach $ x_i $ an der 
			Stelle $ a $ } 
		\item $ f $ heißt in $ a $ partiall differenzierbar, wenn $ f $ in $ a $ nach \textit{ allen} $ x_i, i = 1,\cdots,n$ 
			partiell differenzierbarist. In diesem Fall nennt man den Vektor $ grad(f(a)) = \left( 
			\frac{\partial f}{\partial x_1}(a), \frac{\partial f}{\partial x_2}(a), \cdots ,
			\frac{\partial f}{\partial x_n}(a) \right)  $ den \textit{Gradient von $ f $ in $ a $ } 
		\item $ f $ heißt auf $ U $ \textit{ partiell differenzierbar }, wenn $ f $ partiell differenzierbar in $ a $ 
			für jedes $ a \in  U $ ist.
		\item $ f $ heißt in $ a $ \textit{stetig partiell differenzierbar} , wenn $ f $ auf $ U $ partiell differenzierbar
			ist und zusätzlich alle Ableitungen $ \frac{\partial f}{\partial x_{i}}  = D_i f_{i} U \to \mathbb{R} , 
			1 \leq i \leq  n$, an der Stelle $ a $ stetig sind.
		\item $ f $ heißt \textit{stetig partiell differenzierbar auf $ U $ }, wenn $ f  $ in jedem $ a \in  U $ stetig 
			partiell differenzierbar ist. Für die Menge aller solcher Funktion führen wir die Bezeichnung 
			$$ C^{1}(U) : = \{f: U \to \mathbb{R} : f \text{ ist stetig partiell differenzierbar auf } U\} $$ ein.
	 \end{enumerate}
\end{ibox}
\smalltitle[]{Beispiel}
... 
\begin{ibox}[]{Definition}{CDefinition}
	Seien $ U \subset \mathbb{R}^n  $ offen, ferner 
	$ f:U \to \mathbb{R}  (x_1,x_2,\cdots,x_{n}) \mapsto f(x_1,x_2, \cdots, x_{n} $
	eine Funktion auf $ U $ und $ a = (a_1,a_2, \cdots, a_{n}) \in U $. Seien $ k \in \mathbb{N}  $ und $ i \in {1,\cdots,n }$
	\begin{enumerate}[label=\alph*)]
		\item $ f $ heißt in $ a $ \textit{ $ k $-mal partiell differenzierbar nach $ x_i $  },wenn $ f \; (k-1) $ -mal partiell
			differenzierbar auf $ U $ ist und alle partiell Ableitungen $ (k-1) $-ter Ordnung von $ f $ an der Stell $ a $ partiell
			differenzierbar nach $ x_i $ sind.
		\item f heißt in $ a $ \textit{$ k $-mal partiell differenzierbar}, wenn $ f $ in $ a \; k$-mal partiell differenzierbar ist.
			Nach $ x_j $ für alle $ j \in  {1,2, \cdots, n} $ ist. Die partielle Ableitungen der Ordnung $ k $ von $ f $ in $ a $ 
			sind dann gegeben durch 
			\begin{align*}
				D_{i_k}\left( D_{i_{k-1}} \left( \cdots D_{i_2} \left( D_{i_1} (f) \right) \cdots \right)  \right)(a) &:= 
			\frac{\partial^{k} f}{\partial x_{i_{k}} \partial x_{i_{k-1}} \cdots \partial x_{i_1}} (a) \\
			&:= D_{i_{k}} D_{i_{k-1}} \cdots D_{i_1} f(a) \\
			&:= f_{x_{i_{k}} x_{i_{k-1}} \cdots x_{i_1}}^{\left( k \right) } (a)
			\end{align*}
			
			 wobei die Indizes $ i_{1}, \cdots, i_{k} $ voneinandere unabhängig die Menge $ {1,2,\cdots,n}  $ durchlaufen. 
		\item $ f $ heißt $ k $- \textit{mal partiell differenzierbar auf $ U $ }, wenn $ f $ in jedem $ a \in U \; k-$mal partiell
			differenzierbar sit. Die Funktionen $ D_{i_{k}} \cdots D_{i_1} f:U \to \mathbb{R} , i_1,i_2, \cdots, i_{k} \in{1,\cdots,n}$
			heißen \textit{partiell Ableitung $ k $-te Ordnung von f}.
		\item $ f $ heißt \textit{$ k $-mal stetig partiell differenzierbar auf $ U $ } wenn $ f k- $mal partiell differenzierbar auf 
			$ U $ ist und alle partiellen Ableitungen der Ordnung $ \leq  k $ auf $ U $ stetig sind. Für die Menge aller solcher 
			Funktionen führen wir die Bezeichnung 
			$$ C^{k}(U) : = \{f: U \to \mathbb{R} : f \text{ ist $ k $-mal stetig partiell differenzierbar auf } U \} $$ ein.
	\end{enumerate}
\end{ibox}
\begin{ibox}[29]{Satz}{CTheorem}
    Sei $ U \subset  \mathbb{R}^n  $ offen und sei $ f: U \to \mathbb{R}  $ zweimal partiell differenzierbar auf $ U $; außerhalb 
	seien alle partiellen Ableitungen 2.Ordnung von $ f $ stetig auf $ U $. Dann gilt :
	$$ \frac{\partial^{2} f }{\partial x_{i}  \partial x_{j} } =  \frac{\partial^{2} f }{\partial x_{j}  \partial x_{i} } 
	\forall  i,j \in {1,\cdots, n} $$
\end{ibox}

\smalltitle[]{Korollar}
Seien $ U \subset  \mathbb{R}^n  $ offen, $ k \geq 2 $ und sei $ f \in  C^{k}(U) $. Dann gilt für jedes $ k $-Tupel von Indizes \\
$ \left( i_1,i_2,\cdots,i_{k} \in  {1,2,\cdots,n}^{k} \right)  $ und für jedes Permutation $ \pi: \{1,\cdots,k\} \to \{1,\cdots,k\}: $ 
$$ D_{i_{k}} D_{i_{k-1}} \cdots D_{i_{1}} f = D_{i_{\pi(k)}} D_{i_{\pi(k-1)}} \cdots D_{i_{\pi(1)}}f  $$
\begin{proof}
	Der Beweis erfolgt durch vollständige Induktion über $ k $. Den Induktions Anfang bildet Satz 29 im Induktionsschluss verwendet 
	man die Tatsache, dass sich jede Permutation als Hintereinanderausführung endliche vieler Vertauchungen benachbarter 
	Glider darstellen lässt.

\end{proof}

%\end{document}

%\documentclass[a4paper]{memoir}
%
\usepackage[ngerman]{babel}
\usepackage{bookmark}
\usepackage{amsmath}
\usepackage{amssymb}
\usepackage{amsthm}
\usepackage[T1]{fontenc}
\usepackage{imakeidx}
\usepackage{enumitem}
\usepackage{mathtools}

\usepackage{svg}
\usepackage{parskip}
\usepackage{hyperref} % Make TOC clickable

\usepackage{lipsum}
\usepackage{fancyhdr} % Heading customization
\usepackage{geometry} % Adjust page padding 
\usepackage{adjustbox}

\usepackage{xcolor}
\usepackage[most]{tcolorbox}

\definecolor{White}{HTML}{F8F8F3}
\definecolor{Black}{HTML}{292939}
\definecolor{CTheorem}{HTML}{FFFFFF}
\definecolor{CLemma}{HTML}{FDFFED}
\definecolor{CDefinition}{HTML}{EFF9F0}
\definecolor{DarkGray}{HTML}{6B6969}
\definecolor{ImportantBorder}{HTML}{F14747}


% Set up page layout
\geometry{
    a4paper,
    left=2.5cm,
    right=2.5cm,
    top=2.5cm,
    bottom=2.5cm
}
\linespread{1.1}
\makeindex

\pagestyle{fancy}
\fancyhf{} % Clear headre/footer
\fancyheadoffset[LE,RO]{0pt} % Adjust headsep for page number and title
\fancyhead[RE,LO]{}
\fancyhead[RE,RO]{\rightmark} % Add pape header title
\fancyhead[LE,LO]{\thepage}  % Add page number to the left
\renewcommand{\headrulewidth}{0pt} % Delete the line in the header 

% Define the \para command
\newcommand{\para}[2]{
    \clearpage % Start on a new page
    \thispagestyle{empty}% removes the top left page number and top right chapter name
    \begin{center} % Center the chapter title
        \vspace*{11em}
        \Huge   \bfseries \S #1 \quad  #2 % Chapter title with symbol and counter
    \end{center}
    \vspace{11em}
    \phantomsection
    \addcontentsline{toc}{chapter}{#1.\hspace{0.6em} #2} % Add chapter to table of contents
    \fancyhead[RE,RO]{#2} % Add pape header title
}

\newcommand{\nsec}[2]{
    \section*{\large  #1 \hspace{0.3em} #2}
    \phantomsection
    \addcontentsline{toc}{section}{#1 \hspace{0.3em} #2}
}
\newcommand{\smalltitle}[2][]{
    \phantomsection
    \subsubsection*{ #1 \hspace{0.2em} #2}
}
% Theorem env
%\newenvironment{ibox}[3]{
%    \phantomsection
%    \addcontentsline{toc}{section}{#1 \hspace{0.3em} #2}
%    \vspace{1.5em}
%    \begin{tcolorbox}[
%        enhanced jigsaw,
%        colback=#3 ,
%        colframe=DarkGray,
%        drop shadow, 
%		attach boxed title to top center={yshift=-2mm},
%        before upper={\vspace{0.5em}},
%        after upper={\vspace{0.5em}},
%        \textbf{ \large #1 \hspace{0.1em} #2\hspace{0.4em}}
%	]}{%
%    \end{tcolorbox}
%}
\newtcolorbox{ibox}[3][]{
	enhanced jigsaw,
	colback=#3,
	colframe=DarkGray,
	coltitle=Black,
	drop shadow,
	title={\textbf{ \large  #1\hspace{0.4em}#2}},
	before skip = 2.7em,
	attach title to upper,
	after title={:\quad \vspace{0.5em}\ },
	%toc = {#1 \quad* #2}{section}
	%IfValueTF={#1}{\addcontentsline{toc}{section}{#1 \hspace{0.3em}#2}}{}
	code={\addcontentsline{toc}{section}{#1 \hspace{0.3em}#2}}
}


%\begin{document}
\begin{ibox}[38]{Definition}{CDefinition}
    Seien $ U \subset \mathbb{R}^n  $ offen, $ f: U \to \mathbb{R}^m  $ eine total differenzierbare Abbildung. Seien
	$ a, b \in U $, sodass $ \left| \gamma _{ab} \right|  \subset U $. Dann gibt es $ \phi_1 , \cdots, \phi_{m} 
	\in \left| \gamma_{ab} \right| $, sodass :
	$$ f \left( b \right) = f \left( a \right) + \begin{pmatrix}
		Df_1( \phi_1) \\
		\vdots \\
		Df_{n}(\phi_{m})
	\end{pmatrix} \cdot (b-a)
	 $$
\end{ibox}
\begin{proof}
	Satz 37 auf $ f_1 , \cdots, f_{n} $ anwenden. 
\end{proof}
\smalltitle[]{Beispiel}
$ f: \mathbb{R} \to \mathbb{R}^{2}, \; t \mapsto (t^{2}, t^{3}) $ 
\\
Ist $ [a,b] \subset R $ beschränkt, $ f:[a,b] \to \mathbb{R}  $ stetig differenzierbare, so besagt der HDI:%
$$ f(b) - f(a) = \int_{a}^{b} f'(x) dx $$
Ist wieder $ \gamma_{ab}(f) = a + t(b-a) $, so erhält man durch Substitution 
$$ f(b) - f(a) = \int_0^{1} f'( \gamma_{ab}(t)) \gamma_{ab}'(t) dt = \left( \int_0^{1} f' \left( \gamma _{ab} (t) 
\right) dt \right) .(b-a) $$
\begin{ibox}[]{Definition}{CDefinition}
    Sei $ \left[ \alpha , \beta  \right] \in  \mathbb{R}  $ und  $ A: \left[ \alpha , \beta  \right] \to 
	\mathbb{R} ^{m \times n}, A(t) = \left( a_{ij}(t) \right)_{ \substack{ i = 1 , \cdots, m \\ j = 1 , \cdots, n }}$ 
	mit $ a_{ij}: \left[ \alpha , \beta  \right] \to \mathbb{R}  $ stetig für $ i = 1 , \cdots, m \text{ und }  j=1 , \cdots, n $ Dann ist die Integral über die \textit{matrixwertige Abbildung} $ A $ gegeben durch:
$$ \int_{ \alpha }^{ \beta }A(t) dt = \left( \int_{ \alpha }^{ \beta } a_{ij}(t) dt \right)_{\substack{ i=1 , \cdots, m \\
j = 1 , \cdots, n}} $$
\end{ibox}
\begin{ibox}[39]{Satz}{CTheorem}
    Seien $ U \subseteq \mathbb{R}^n  $ offen, $ f: U \to \mathbb{R}^n  $ stetig partiell differenzierbar. Seien 
	$ a,b \in U $ sodass $ \left| \gamma _{ab} \right| \subseteq U $. Dann gilt: 
	$$ f(b) - f(a) = \underbrace{ \left( \int_0^1 Df \left( \gamma _{ab} (t)  \right) dt \right)  }_{ \in \mathbb{R}^{
	m \times n}} \cdot \underbrace{ (b-a)}_{ \in \mathbb{R}^n } $$
\end{ibox}
\begin{proof}
	Wir wenden den HDI an auf $ g_{i}:= \left( f_i \circ \gamma_{ab} \right) \in C^{1}\left( [0,1] \right)$ für $ i = 1 , \cdots, m $ und
	erhalten mit der Kettenregel: 
	\begin{align*}
		f_i(b) - f_i(a) &= g_i(1) - g_i(0) = \int_{ 0 }^{ 1 } \frac{d}{dt} g_i(t) dt \\
						&=  \int_{ 0 }^{ 1 } \left( \sum_{j = 1}^{n} \frac{\partial f_i}{\partial x_j} ( \gamma_{ab}(t))
						\cdot (b_j - a_j)\right) dt \\
						&= \sum_{j=1}^{n} \underbrace{\left( \int_{ 0 }^{ 1 } \frac{\partial f_i}{\partial x_j} 
								\left( \gamma _{ab}(t) \right) \cdot
						dt \right)}_{v_j} \underbrace{ (b_j - a_j}_{u_j} \\
						&=  \left< \left( \int_{ 0 }^{ 1 } Df_i ( \gamma_{ab}(t)) dt \right) , (b-a) \right>
	\end{align*}
	Dies zeigt die $ i- $te Zeile der gewünschte Gleichung. 
\end{proof}

\begin{ibox}[]{Halfsatz}{CDefinition}
    Ist $ \left[ \alpha , \beta  \right] \subseteq \mathbb{R}  $ ein Kompaktes Intervall, $ v : \left[ \alpha , \beta 
	\right] \to \mathbb{R}^n  $ stetige Abbildung. So gilt: 
	$$ \left| \left|  \int_{ \alpha }^{ \beta } v(t) dt  \right|  \right| _{2} \leq \int_{ \alpha }^{ \beta } \|v(t)\|_{1} dt $$
\end{ibox}
\begin{proof}
	Sei $ u := \int_{ \alpha }^{ \beta } v(t) dt \in \mathbb{R}^n  $ . Dann ist 
\begin{align*}
	\| u \|_{2}^{2} = \left<u,u \right> &= \sum_{j=1}^{n} \int_{ \alpha }^{ \beta } v_j(t) dt \cdot u_j\\
	&= \int_{ \alpha  }^{ \beta  } \left( \sum_{j = 1}^{n}v_j(t) \cdot u_j dt \right) \\
	&=  \int_{ \alpha  }^{ \beta  } \left<v(t), u \right> dt \\
  & \leq \int_{ \alpha  }^{ \beta } \left| \left<v(t),u\right> \right| dt \\
  & \stackrel{Schwz.}{ \leq } \int_{ \alpha  }^{ \beta  } \left( \| v(t) \|_{2} \cdot \|u \|_{2} \right) dt \\
  &=  \|u \|_{2} \int_{ \alpha  }^{ \beta  } \|v(t) \|_{2}dt
\end{align*}
Für
$$
\|u \|_{2} > 0 \implies \|u \|_{2} \leq  \int_{ a }^{ b } \|v(t) \|_{2} dt
$$
Für
$$
\|u \|_{2} = 0 \implies \|u \|_{2} = 0 \leq  \int_{ a }^{ b } \|v(t) \|_{2} dt
$$
\end{proof}

\smalltitle[]{Korollar}
Seien $ U \subseteq \mathbb{R}^n  $ offen, $ f: u \to \mathbb{R}^m  $ stetig partiell differenzierbar. Seien 
$ a,b \in U $ mit $ \left| \gamma _{ab} \right| \subseteq U$. Dann gilt: 
$$ \|f(b) - f(a) \|_{2} \leq  M \|b-a\|_{2} $$
für $ M := sup \left\{ \, \|Df(x) \| \,| \; x \in \left| \gamma _{ab} \right|  \right\}  $ Dabei ist die Matrix norm 
$ \|A \| $ für $ A \in \mathbb{R}^{m \times n} $ gegeben durch 
$$ \|A \| = \sup_{ v \in \mathbb{R}^n \setminus \{0\}}  \frac{\|Av \|_2}{\|v \|_{2}} 
=\sup_{\substack{v \in \mathbb{R}^n  \\ \|v \|_{2} = 1}} \frac{\|Av \|_{2}}{\|v \|_{2}}  $$

Es gilt: 
$$ M = \sup_{ \substack{ x \in \left| \gamma _{ab} \right|  \\ v \in  \mathbb{R}^n  \\ \|v \|_{2} = 1 }} 
\underbrace{ \frac{\| Df (x) \cdot v \|_{2}}{\|v \|_{2}} }_{G(x,v)} < \infty \text{ ,da } $$
$$ G: \underbrace{ \| \gamma_{ab} \| \times \partial \mathbb{B}(0,1) }_{kompakt} \to \mathbb{R} \text{ stetig }  $$

\begin{proof}
	Es ist : 
	\begin{align*}
		\|f(b) - f(a) \|_{2} &=  \left|\left| \left( \int_{ 0 }^{ 1 } Df \left( \gamma _{ab}(t) \right) dt \right)
		\cdot (b-a)\right|\right|_{2} \\
		& \leq \int_{ 0 }^{ 1 } \left|\left| Df ( \gamma _{ab}(t)) \cdot (b-a) \right|\right|_{2} dt \\
		& \leq \int_{ 0 }^{ 1 } M \cdot \|b-a \|_{2} dt \\
		&=  M \cdot \|b-a \|_{2}
	\end{align*}
	
\end{proof}

\para{8}{Die Taylorformel}

\begin{ibox}[]{Definition}{CDefinition}
    Seien $ U \subset \mathbb{R}^n  $ offen, $ x_0 \in U $ und $ f: U \to \mathbb{R}  $ eine $ k $-mal stetig partiell differenzierbar
	Funktion ( $ c^{k} $ ). Dann bezeichnet man für $ 0 \leq  m \leq k $ das duch
	$$ T_{f, x_0, m} \left(x\right) : \sum_{i=0}^{m} \left( \, \sum_{ \substack{ |\alpha| = L \\ \alpha \in \mathbb{N}_{0}^{n} }}
	\frac{D^{\alpha }f \left(x_0\right) }{\left( x - x_0 \right) ^{\alpha }} \right) $$ 
		definiert Polynom $ T_{f,x_0,m} $ als \textbf{ $ m $-tes Taylorpolynom von $ f \text{ in } x_0 $ }. Hierbei wird mit 
	$ |\alpha| : = \alpha_1 +  \cdots + \alpha_n $ die \textbf{Länge des Multiindex} $ \alpha := \left( \alpha_1, \cdots,  \alpha_{n} \right) \in \mathbb{N}_{0}^{n} $ und mit $ \alpha ! $ das Produkt $ \alpha ! := \alpha_1 ! , \cdots,  \alpha_{n} ! $ bezeichnet. Der 
	\textbf{Differentialoperator} $ D^{\alpha } $ ist für $ \alpha  \in \mathbb{N}_{0}^{n} $ gegeben durch $ D^{\alpha }f = 
	D_1^{\alpha } \cdots  D_n^{\alpha_n} f \in C^{|\alpha |}$ dabei ist $ D_{\gamma }^{\alpha r} := \underbrace{ D_{\gamma } 
	D_{\gamma }, \cdots,  D_{\gamma }}_{\alpha \gamma \text{ -mal } } \; \gamma = 1, \cdots,  n \text{ für } y = \left( y_1
, \cdots,  y_n\right) \in \mathbb{R}^n $	ist mit  $ y^{\alpha } $ die reele Zahl $ y^{\alpha } = y_1^{\alpha_1},
\cdots,  y_n^{\alpha_n} $  gemeint. Wie im Kapitel 7 werde für $ x,x_0 \in \mathbb{R}^n  $ mit $ \upgamma_{x_0x} $ die $ c^{ \infty } $
-Ableitung $ \upgamma_{x_0x}: \mathbb{R}  \to \mathbb{R}^n  $, $ t \mapsto \upgamma_{x_0x} \left(t\right) := x_0 + t(x-x_0)$ und mit
$  \left| \upgamma_{x_0x} \right| = \upgamma_{x_0x} \left( \left[ 0,1 \right]  \right)  $ die Verbindugsstrecke von $ x_0 \text{ und }
x $ bezeichtent. Die Taylorformel für $ C^{k+1} $ Fkten mehrerer veränderlichen lautet dann wie folgt:
\end{ibox}
\begin{ibox}[40]{Taylor}{CTheorem}
	Seien $ U \subset \mathbb{R}^n  $ offen, $ x_0 \in U $ und $ f:U \xrightarrow{c^{k+1}}$ eine $ (k+1) $-mal stetig 
	partiell differenzierbare Funktion. Ist dann $ x \in  U $ win Punkt mit $ \| \upgamma_{x_0x} \| \in U $, so existiert ein 
	$ \xi \in \left| \upgamma_{x_0x} \right|  $ mit 
	$$ f \left(x\right) = \sum_{l=0}^{k} \left( \, \sum_{ \substack{ |\alpha | = l \\ \alpha \in \mathbb{N}_{0}^{n} }} 
	\frac{D^{\alpha }f \left(x_0\right) }{\alpha !} \left( x - x_0 \right) ^{\alpha } \right) + \sum_{ \substack{ |\alpha | = l \\ 
\alpha  \in  \mathbb{N}_{0}^{n}}} \frac{D_{\alpha }f \left(x_0\right) }{\alpha !} \left( x-x_0 \right) ^{\alpha } $$
\end{ibox}
 
\smalltitle[]{Bemerkung:}
Unter den Voraussetzungen von Satz 40 wird die Funktion $ f $ an der Stell x durch das $ k $-te Taylorpolynom von $ f $ in $ x_0 $ 
approximiert. Der Fehler $ R_{k+1} \left(x\right)  := f \left(x\right) - T_{f, x_0, k} \left(x\right)  $ dier Approximation, das 
\textbf{$ k $-te Restglied} in $ x  $ kann in de Form
$$
R_{k+1}\left(x\right)  =\sum_{ \substack{ |\alpha | = k+1 \\ \alpha \in \mathbb{N} _{0}^{n} }} \frac{D_{\alpha }f \left(x_0\right) }{
\alpha !} \left( x -x_0 \right) ^{\alpha }
$$ dargestellt werden, wobei $ \xi \in  \left| \upgamma_{x_0x}  \right|  $  geeignet zu wählen ist.
\begin{proof}
	Wir werden den Taylorschen Satz für Funktionen einer veränderlichen auf die Funktion $ g: \left( f \circ \upgamma_{x_0x} \right) 
	: [0,1] \xrightarrow{c^{k+1}} \mathbb{R} $ an. Danach existiert ein $ t_0 \in  (0,1) $ mit
	$$
	g (1) = g \left(0\right) + \sum_{l=1}^{k} \frac{g' \left(0\right) }{l !} \left(  1 - 0 \right) ^{l} + \frac{g^{k+1} \left(t_0\right) 
	}{\left( k+1 \right) !} \left( 1-0  \right) ^{k+1}.
	$$
Dies ist gleichbedeutend mit $ f \left(x\right)  = f \left(x_0\right) \sum_{l=1}^{n} \frac{g' \left(0\right) }{l !} + \frac{g^{k+1}
\left(t_0\right) }{\left( k+1 \right) !}  $ Zum Beweis der Taylorformel genügt es also zu zeigen: 
\end{proof}
\smalltitle[]{Behauptung:}
$ \forall  t \in  [0,1], \, l = 1 , \cdots,  k +1 :  $ 
\begin{equation}
	g^{\left(l\right) } \left(t\right) = \sum_{ \substack{ |\alpha | = l \\ \alpha  \in \mathbb{N}_{0}^{n}}} l! \frac{D^{\alpha}f 
	\left( \upgamma_{x_0x} \left(t\right)  \right) }{\alpha !} \left( x - x_0 \right) ^{\alpha } 
\end{equation}
Also Satz 40 für $ \xi := \upgamma_{x_0x} \left(t\right)  $ 
\begin{proof}
	Nach der Kettenregel ist $ g' \left(t\right) = \sum_{i=1}^{m}D_i f \left( \upgamma_{x_0x} \left(t\right)  \right) \cdot 
	\left( x_i-x_0 \right) \; \forall t \in [0,1]$ durch Induktion über $ l $ folgt ebenso mit Hilfe der Kettenregel, dass 
	\begin{equation}
		g^{\left(l\right) } \left(t\right) = \sum_{i_1, \cdots,  i_l = 1}^{n} D_{i_{l}} , \cdots,  D_{i_{1}} f \left( \upgamma_{x_0x}
		\left(t\right) \right) \cdot \left( x_{i_1} - x_{0i_1} \right) \cdots \left( x_{i_{l}} - x_{0i_{l}} \right) 
	\end{equation}
	für alle $ t \in [0,1] $ gilt zum Beweis von (1) müssen wir nur noch die Summanden in (2) geeignet zusammenfassen. Dabei nutzen wir:
\smalltitle[]{Hilfssatz}
Seien $ U \in \mathbb{R}^n  $ offen, $ 2 \leq  k \in \mathbb{N}, f \in C^{k} \left(U\right)  \text{ und }  \left( i_1 , \cdots,  i_{n} 
\right) \in \left\{ 1, \cdots,  n \right\} ^{k} $. Dann gilt für jede Permutation $ \pi : \left\{ 1, \cdots,  k \right\} \to 
\left\{ 1, \cdots,  k \right\} : \; D_{i_{n}} D_{i_{n-1}}  \cdots \; D_{i_2} D_{i_1} f = D_{i \pi \left(k\right) } D_{i \pi (k-1)} 
 \cdots D_{i \pi (1)} f  $ 
 \parskip=3pt
 Der Beweis ist leicht durch Induktion über $ k $ zu führen. Für $ k=2 $ geht der Hilfssatz in Satz 29 über, nach Induktionsvoraussetzung
 ist die Vertauschung benachbarter $ D_{\gamma } \cdot D_{\mu}$ erlaubt, der Induktionsschritte kann also vollzogen werden, wenn man 
 berücksichtig, dass sich jede Permutaion als Hintereinanderausführung solcher Vertauschungen beschreiben lässt
\end{proof}

Der Hilfssatz besagt, dass in Gl (2) alle Summanden $ D_{i_{l}} \cdots D_{i_1} f \left( \upgamma_{x_0x} \left(t\right)  \right) 
\cdot \left( x_{i_1} x_{0i_1} \right) \cdots \left( x_{i_{l}} - x_{0i_{l}} \right)   $ in denen jeder index $ \gamma  $ genau 
$ a_{\gamma }$-mal auftritt. $ \left( 1 \leq  \gamma  \leq n , \alpha_1 + \cdots + \alpha_n = l, \alpha = 
\left( \alpha_1 , \cdots,  \alpha_n \right) \in  \mathbb{N} _{0}^{n}\right)  $ folgt aus (2): $ \forall t \in [0,1] $ 

\begin{align*}
	g^{\left(l\right) }\left(t\right)  &= \sum_{i_1 , \cdots,  i_{l} = 1}^{n} D_{i_{l}} \dots D_{i_1} f \left( \upgamma_{x_0x}(t)\right) 
	\cdot \prod_{\alpha  = 1}^{n} \left( x_{\gamma } - x_{0 \gamma } \right) ^{\alpha _{\gamma }} \\
	&= \sum_{ \substack{ |\alpha | = l \\ \alpha \in \mathbb{N}_{0}^{n} }} \frac{l !}{\alpha_1 ! \cdots \alpha_n !} D_{1}^{\alpha_1}
	\cdots D_{n}^{\alpha_n} f \left( \upgamma_{x_0x} \left(t\right)  \right) \cdot \prod_{\alpha = 1}^{n} \left( 
		x_\gamma - x_{0 \gamma }\right)^{\alpha_{\gamma }}  \\
		&= \sum_{ \substack{ |\alpha | = l \\ \alpha \in \mathbb{N}_{0}^{n} }} l ! \frac{D^{\alpha }f \left( \upgamma_{x_0x} 
		\left(t\right) \right) }{\alpha !} \left( x - x_0 \right)^{\alpha }
\end{align*}
Damit ist sowohl (1), als auch der Taylorsche Satz bewiesen.


%\end{document}


%\documentclass[a4paper]{memoir}
%
\usepackage[ngerman]{babel}
\usepackage{bookmark}
\usepackage{amsmath}
\usepackage{amssymb}
\usepackage{amsthm}
\usepackage[T1]{fontenc}
\usepackage{imakeidx}
\usepackage{enumitem}
\usepackage{mathtools}

\usepackage{svg}
\usepackage{parskip}
\usepackage{hyperref} % Make TOC clickable

\usepackage{lipsum}
\usepackage{fancyhdr} % Heading customization
\usepackage{geometry} % Adjust page padding 
\usepackage{adjustbox}

\usepackage{xcolor}
\usepackage[most]{tcolorbox}

\definecolor{White}{HTML}{F8F8F3}
\definecolor{Black}{HTML}{292939}
\definecolor{CTheorem}{HTML}{FFFFFF}
\definecolor{CLemma}{HTML}{FDFFED}
\definecolor{CDefinition}{HTML}{EFF9F0}
\definecolor{DarkGray}{HTML}{6B6969}
\definecolor{ImportantBorder}{HTML}{F14747}


% Set up page layout
\geometry{
    a4paper,
    left=2.5cm,
    right=2.5cm,
    top=2.5cm,
    bottom=2.5cm
}
\linespread{1.1}
\makeindex

\pagestyle{fancy}
\fancyhf{} % Clear headre/footer
\fancyheadoffset[LE,RO]{0pt} % Adjust headsep for page number and title
\fancyhead[RE,LO]{}
\fancyhead[RE,RO]{\rightmark} % Add pape header title
\fancyhead[LE,LO]{\thepage}  % Add page number to the left
\renewcommand{\headrulewidth}{0pt} % Delete the line in the header 

% Define the \para command
\newcommand{\para}[2]{
    \clearpage % Start on a new page
    \thispagestyle{empty}% removes the top left page number and top right chapter name
    \begin{center} % Center the chapter title
        \vspace*{11em}
        \Huge   \bfseries \S #1 \quad  #2 % Chapter title with symbol and counter
    \end{center}
    \vspace{11em}
    \phantomsection
    \addcontentsline{toc}{chapter}{#1.\hspace{0.6em} #2} % Add chapter to table of contents
    \fancyhead[RE,RO]{#2} % Add pape header title
}

\newcommand{\nsec}[2]{
    \section*{\large  #1 \hspace{0.3em} #2}
    \phantomsection
    \addcontentsline{toc}{section}{#1 \hspace{0.3em} #2}
}
\newcommand{\smalltitle}[2][]{
    \phantomsection
    \subsubsection*{ #1 \hspace{0.2em} #2}
}
% Theorem env
%\newenvironment{ibox}[3]{
%    \phantomsection
%    \addcontentsline{toc}{section}{#1 \hspace{0.3em} #2}
%    \vspace{1.5em}
%    \begin{tcolorbox}[
%        enhanced jigsaw,
%        colback=#3 ,
%        colframe=DarkGray,
%        drop shadow, 
%		attach boxed title to top center={yshift=-2mm},
%        before upper={\vspace{0.5em}},
%        after upper={\vspace{0.5em}},
%        \textbf{ \large #1 \hspace{0.1em} #2\hspace{0.4em}}
%	]}{%
%    \end{tcolorbox}
%}
\newtcolorbox{ibox}[3][]{
	enhanced jigsaw,
	colback=#3,
	colframe=DarkGray,
	coltitle=Black,
	drop shadow,
	title={\textbf{ \large  #1\hspace{0.4em}#2}},
	before skip = 2.7em,
	attach title to upper,
	after title={:\quad \vspace{0.5em}\ },
	%toc = {#1 \quad* #2}{section}
	%IfValueTF={#1}{\addcontentsline{toc}{section}{#1 \hspace{0.3em}#2}}{}
	code={\addcontentsline{toc}{section}{#1 \hspace{0.3em}#2}}
}


%\begin{document}
Unsere Ausgangsquader $ Q_0 = Q $ hat die geforderten Eigenschaften. Seine 
$ Q = Q_0 \supset Q_1 \supset Q_2 \supset \cdots$ bereits konstruiert. Es gelte $ Q_{m} = I_1 
\times I_2 \times \cdots I_{n} \; , I_{k} = \left[ a_{k}, b_{k} \right]  \subset \mathbb{R} \; 
\forall k = 1,2,\cdots, n$ . Wir splitten jedes der Intervalle $ I_{k} $ in der Intervalllemitte 
auf und erhalten Unterteilung $ I_{k} = I_{k_{1}} \cup I_{k_{2}} $ mit \\
$ I_{k_{1}} = \left[ a_{k}, \frac{a_{k}+ b_{k}}{2} \right] \; , I_{k_{2}} 
= \left[ \frac{a_{k}+ b_{k}}{2}, b_{k} \right] \; , 1 \leq k \leq n $. Die kartesischen Produckte
$ I_{1_{ \gamma_{1}}} \times I_{2_{ \gamma_{2}}} \times \cdots  I_{n_{ \gamma_{n}}} \; ,
\gamma \in {1,2} \; \forall 1 \leq k \leq n $ aller Intervallhälften der $ I_{k} \; , k = 1,2, \cdots n $ unterteilen den  abgeschlossener Quader $ Q_{m} $  in 
in insgesamt $ 2^{n} $  vielen kleinere abgeschlos. Quader  
Nur können wir folgendes Kriterium formulieren:
\begin{ibox}[25]{Satz von Heine-Borel}{CTheorem}
    Für ein Menge $ K \subset  \mathbb{R}^n  $ sind folgende Ausagen äquivalent:
	\begin{enumerate}[label=\alph*)]
		\item $ K $ ist abgeschlossen und beschränkt
		\item $ K $ ist kompakt
	\end{enumerate}
\end{ibox}

\begin{proof}
	Siehe Satz 21.
\end{proof}

\smalltitle[]{Korollar}
Für jede kompakte Teilmenge $ \emptyset \neq A \subset \mathbb{R}  $ gilt:
$$ sup A \in A, \; inf A \in A $$
\begin{proof}
	Da $ A $ als kompakte Menge insbesondere beschränkt ist, existeren $ sup A, \; inf A \in R $.
	Weil sowohl $ sup A  $ als auch $ inf A $ Häufungspunkte geeigneter Folgen in $ A $ sind, 
	impliziert die Abgeschlossenheit von $ A $ in Verbindung mit Satz 8, dass $ sup A \in  A
	\text{ und } inf A \in A $ gilt.
\end{proof}

\begin{ibox}[26]{Satz}{CTheorem}
    Ist $ U \subset  \mathbb{R}^n  $ offen $ K \subset  U $-kompakt und $ f:U \to \mathbb{R}^m $
	-stetige Abbildung, so ist $ f(k) \subset \mathbb{R}^m $ ebenfalls kompakt.
\end{ibox}

\begin{proof}
	Sei $ \left( U_{i} \right)_{i \in  I} $   eine offene Überdeckung von $ f(k) $. Sei 
	$ \forall i \in  I, \; V_{i}:= f^{-1} \left( U_{i} \right) $, $ f- $stetig $ \implies i \in I$
	$ V_{i}- $offen. Die Familie $ \left( V_{i} \right)_{i \in  I} $ ist dine Überdeckungvon $ K $
	, $ K $-kompakt $ \implies  \exists V_{i_{1}}, V_{i_{2}} , \cdots V_{i_{n}} \text{ s.d. } K 
	\subset \bigcup\limits_{p = 1}^{k} U_{i_{p}} \implies f(k) \subset  
	\bigcup\limits_{p = 1}^{k} U_{i_{p}} \implies  f(k) $-kompakt
\end{proof}

\begin{ibox}[27]{Satz}{CTheorem}
    Ist $ U \subset \mathbb{R}^n  $ offen, $ k \subset  U $ kompakt und $ f: U \to \mathbb{R}  $
	eine stetige Funktion auf $ U $, so gibt es Punkte $ p,q \in K $ mit $ f(p) = sup \left\{ 
f(x): x \in  K \right\}, f(q) = inf \left\{ f(x): x \in K \right\}   $ 
\end{ibox}

\begin{proof}
	Laut Satz 26 ist $ A := f(k) \subset  \mathbb{R}  $ kompakt, also im Hinblick auf dem Korollar
	zu Satz 25: $ sup f(k) \in  f(k), \; inf f(k) \in  f(k) $ . Deshalb können wir $ p,q \in K $ 
	mit gewünschten Eigenschaften finden
\end{proof}

\begin{ibox}[]{Definition}{CDefinition}
    Sei $ U \subset  \mathbb{R}^n  $ eine beliebige Teilmenge und $ f: U \to \mathbb{R}^m  $ eine 
	Abbildung. Dann nennt man $ f $ \textit{gleichmäßig stetig} auf $ U $ wenn gilt:
	$$ \forall \varepsilon > 0 \; \exists \delta > 0 \text{ s.d. } \left| x - y \right| < \delta , 
	 \; x,y \in  U \implies  \left| f(x) - f(y) \right| < \varepsilon  $$
\end{ibox}

\begin{ibox}[28]{Satz}{CTheorem}
    Sei $ R \in  \mathbb{R}^n $ kompakt und $ f: K \to \mathbb{R}^m $ eine stetige Abbildung.
	Dann ist $ f $ gleichmäßig stetig auf $ K $ 
\end{ibox}

\begin{proof}
	Sei $ \varepsilon > 0 $ vergegeben. $ \forall a \in  K \; \exists r(a)>0$  mit 
	$ \left| f(x) - f(a) \right| < \frac{ \varepsilon }{2} \; \forall x \in  B(a, r(a)) \cap K $
	(wegen der Stetigkeit von $ f $  in a) Offenbar ist $ K \subset \bigcup\limits_{a \in K}^{} 
	B(a,\frac{r(a)}{2} ) $. $ K $-kompakt $ \implies a_1, a_2, \cdots , c_{k} $ s.d $   K \subset 
	\bigcup\limits_{p = 1}^{} B \left( a_{p}, \frac{r \left( a_{p} \right) }{2}  \right) $.
	Man setz nun $ \delta := \frac{1}{2} min \left\{ r \left( a_{1} \right), \cdots, 
	\left( a_{k} \right) \right\}   $ Seien dann $ x_{1}, x_{2} \in K $ mit 
	$ \left| x_1-x_2 \right|  < \delta $ beliebig gewählt. Wir fixieren ein
	$ i \in  \left\{1,2,\cdots , k  \right\}  $  mit $ x_1 \in  B \left( a_{i},
	\frac{r \left( a_{i} \right) }{2}  \right)  $; dann gilt auch $ \left| x_2 - a_{i} \right| 
	\leq \left| x_2 - x_1 \right| + \left| x_{1}  + a_{i}\right| 
	< \delta + \frac{r(a_i)}{2} < r(a_i) $
	also liegt mit $ x_1 \text{ auch }  x_2 $ in $ B(a_{i}, r \left( a_{i} \right))
	$, und wir erhalten $ \left| f \left( x_1 \right) - f \left( x_2 \right)  \right| \leq 
	\left| f \left( x_1 \right) - f \left( a_{i} \right)   \right| + \left| f \left( a_{i} \right) 
	- f \left( x_2 \right) \right| $ $ < \frac{ \varepsilon }{2} + \frac{ \varepsilon }{2}  $.
	Damit ist die gleichmäßig Stetigkeit von $ f $ auf $ K $ beweisen.
\end{proof}
\para{5}{Partielle Ableitung}
\begin{ibox}[]{Definition}{CDefinition}
    Seien $ U \subset \mathbb{R}^n  $ offen, ferner 
	$$ f:U \to \mathbb{R}  (x_1,x_2,\cdots,x_{n}) \mapsto f(x_1,x_2, \cdots, x_{n} $$
     eine Funktion auf $ U $ und $ a = (a_1,a_2, \cdots, a_{n}) \in U $ 
	 \begin{enumerate}[label=\alph*)]
	 	\item $ f $ heißt in $ a $ \textit{partiell differenzierbar} nach $ x_i $ oder auch \textit{partiell 
			differenzierbar bezüglich der $ i $-ten Koordinate}, wenn der Grenzwert 
			$$	
				 D_{i}f(a): \frac{\partial f}{\partial x_{i}} (a)_i := \\= 
				\lim_{\substack{t \to a_{i} \\ t \neq  a_i}} \frac{f(a_1,\cdots ,
			a_{i-1}, t, a_{i+1}, \cdots , a_n) - f(a_1,\cdots ,a_{i-1}, a_i, a_{i+1}, \cdots , a_n) }{t-a_i}  
			$$	
			dann $ \frac{\partial f}{\partial x_i} (a) $ die \textit{partille Ableitung von $ f $ nach $ x_i $ an der 
			Stelle $ a $ } 
		\item $ f $ heißt in $ a $ partiall differenzierbar, wenn $ f $ in $ a $ nach \textit{ allen} $ x_i, i = 1,\cdots,n$ 
			partiell differenzierbarist. In diesem Fall nennt man den Vektor $ grad(f(a)) = \left( 
			\frac{\partial f}{\partial x_1}(a), \frac{\partial f}{\partial x_2}(a), \cdots ,
			\frac{\partial f}{\partial x_n}(a) \right)  $ den \textit{Gradient von $ f $ in $ a $ } 
		\item $ f $ heißt auf $ U $ \textit{ partiell differenzierbar }, wenn $ f $ partiell differenzierbar in $ a $ 
			für jedes $ a \in  U $ ist.
		\item $ f $ heißt in $ a $ \textit{stetig partiell differenzierbar} , wenn $ f $ auf $ U $ partiell differenzierbar
			ist und zusätzlich alle Ableitungen $ \frac{\partial f}{\partial x_{i}}  = D_i f_{i} U \to \mathbb{R} , 
			1 \leq i \leq  n$, an der Stelle $ a $ stetig sind.
		\item $ f $ heißt \textit{stetig partiell differenzierbar auf $ U $ }, wenn $ f  $ in jedem $ a \in  U $ stetig 
			partiell differenzierbar ist. Für die Menge aller solcher Funktion führen wir die Bezeichnung 
			$$ C^{1}(U) : = \{f: U \to \mathbb{R} : f \text{ ist stetig partiell differenzierbar auf } U\} $$ ein.
	 \end{enumerate}
\end{ibox}
\smalltitle[]{Beispiel}
... 
\begin{ibox}[]{Definition}{CDefinition}
	Seien $ U \subset \mathbb{R}^n  $ offen, ferner 
	$ f:U \to \mathbb{R}  (x_1,x_2,\cdots,x_{n}) \mapsto f(x_1,x_2, \cdots, x_{n} $
	eine Funktion auf $ U $ und $ a = (a_1,a_2, \cdots, a_{n}) \in U $. Seien $ k \in \mathbb{N}  $ und $ i \in {1,\cdots,n }$
	\begin{enumerate}[label=\alph*)]
		\item $ f $ heißt in $ a $ \textit{ $ k $-mal partiell differenzierbar nach $ x_i $  },wenn $ f \; (k-1) $ -mal partiell
			differenzierbar auf $ U $ ist und alle partiell Ableitungen $ (k-1) $-ter Ordnung von $ f $ an der Stell $ a $ partiell
			differenzierbar nach $ x_i $ sind.
		\item f heißt in $ a $ \textit{$ k $-mal partiell differenzierbar}, wenn $ f $ in $ a \; k$-mal partiell differenzierbar ist.
			Nach $ x_j $ für alle $ j \in  {1,2, \cdots, n} $ ist. Die partielle Ableitungen der Ordnung $ k $ von $ f $ in $ a $ 
			sind dann gegeben durch 
			\begin{align*}
				D_{i_k}\left( D_{i_{k-1}} \left( \cdots D_{i_2} \left( D_{i_1} (f) \right) \cdots \right)  \right)(a) &:= 
			\frac{\partial^{k} f}{\partial x_{i_{k}} \partial x_{i_{k-1}} \cdots \partial x_{i_1}} (a) \\
			&:= D_{i_{k}} D_{i_{k-1}} \cdots D_{i_1} f(a) \\
			&:= f_{x_{i_{k}} x_{i_{k-1}} \cdots x_{i_1}}^{\left( k \right) } (a)
			\end{align*}
			
			 wobei die Indizes $ i_{1}, \cdots, i_{k} $ voneinandere unabhängig die Menge $ {1,2,\cdots,n}  $ durchlaufen. 
		\item $ f $ heißt $ k $- \textit{mal partiell differenzierbar auf $ U $ }, wenn $ f $ in jedem $ a \in U \; k-$mal partiell
			differenzierbar sit. Die Funktionen $ D_{i_{k}} \cdots D_{i_1} f:U \to \mathbb{R} , i_1,i_2, \cdots, i_{k} \in{1,\cdots,n}$
			heißen \textit{partiell Ableitung $ k $-te Ordnung von f}.
		\item $ f $ heißt \textit{$ k $-mal stetig partiell differenzierbar auf $ U $ } wenn $ f k- $mal partiell differenzierbar auf 
			$ U $ ist und alle partiellen Ableitungen der Ordnung $ \leq  k $ auf $ U $ stetig sind. Für die Menge aller solcher 
			Funktionen führen wir die Bezeichnung 
			$$ C^{k}(U) : = \{f: U \to \mathbb{R} : f \text{ ist $ k $-mal stetig partiell differenzierbar auf } U \} $$ ein.
	\end{enumerate}
\end{ibox}
\begin{ibox}[29]{Satz}{CTheorem}
    Sei $ U \subset  \mathbb{R}^n  $ offen und sei $ f: U \to \mathbb{R}  $ zweimal partiell differenzierbar auf $ U $; außerhalb 
	seien alle partiellen Ableitungen 2.Ordnung von $ f $ stetig auf $ U $. Dann gilt :
	$$ \frac{\partial^{2} f }{\partial x_{i}  \partial x_{j} } =  \frac{\partial^{2} f }{\partial x_{j}  \partial x_{i} } 
	\forall  i,j \in {1,\cdots, n} $$
\end{ibox}

\smalltitle[]{Korollar}
Seien $ U \subset  \mathbb{R}^n  $ offen, $ k \geq 2 $ und sei $ f \in  C^{k}(U) $. Dann gilt für jedes $ k $-Tupel von Indizes \\
$ \left( i_1,i_2,\cdots,i_{k} \in  {1,2,\cdots,n}^{k} \right)  $ und für jedes Permutation $ \pi: \{1,\cdots,k\} \to \{1,\cdots,k\}: $ 
$$ D_{i_{k}} D_{i_{k-1}} \cdots D_{i_{1}} f = D_{i_{\pi(k)}} D_{i_{\pi(k-1)}} \cdots D_{i_{\pi(1)}}f  $$
\begin{proof}
	Der Beweis erfolgt durch vollständige Induktion über $ k $. Den Induktions Anfang bildet Satz 29 im Induktionsschluss verwendet 
	man die Tatsache, dass sich jede Permutation als Hintereinanderausführung endliche vieler Vertauchungen benachbarter 
	Glider darstellen lässt.

\end{proof}

%\end{document}

\printindex
\end{document}
