\documentclass[12pt]{article}
\usepackage[a4paper, margin=1in]{geometry}
\usepackage{amsmath}
\usepackage{amsfonts}
\usepackage{amsthm}
\title{Ringe}
\author{Sisam Khanal}
\date{\today}
\begin{document}
\maketitle

\newtheorem{theorem}{Theorem}
\theoremstyle{definition}
\newtheorem{definition}{Definition}
\newcommand{\iu}{{i\mkern1mu}}
\newcommand{\defn}[2]{
	\begin{definition} \textbf{ #1}
  #2
  \end{definition}
}


\defn{Ring}{
	$ \left( R, +, \cdot \right)  $ mit inneren Verknüpffung $ + : R \times R \to R  $ (der Addition) und 
	$ \cdot : R \times R \to R $ (der Multiplikaion heißt \textbf{Ring}, wenn gilt:
	\begin{itemize}
		\item $ (R, +)  $ ist eine ablesche Gruppe,
		\item $ (R,\cdot) $ ist eine Halbgruppe (ohne Identität)
		\item $ a(b+c) = ab +ac \text{ und } (a+b)c = ac+bc $ für alle $ a,b,c \in R $ (Distributivgesetze).
	\end{itemize}
}
%Im Fall $ ab = ba $ gilt für alle $ n \in  \mathbb{N}  $ die binomische Formel.
\defn{Einheitengruppe}{
	Sei $ R $ ein Ring mit 1. Ein Element $ a \in R $ heißt \textit{invertierbar} oder eine \textit{Einheit},
	wenn es ein $ b \in R $ gibt mit $ ba = 1 = ab $. Die Einheiten bilden eine Gruppe 
	$$ R^{\times} = \left\{ a \in R | \;  a \text{ ist invertierbar  } \right\}  $$
}
Beachte, dass $ R^{\times} $ kein Teilring ist.
\subsubsection*{Beispiel}
\begin{itemize}
	\item $ \mathbb{Z}^{\times} = \left\{ \pm 1 \right\}  $
	\item $ \mathbb{R}^{\times} = \mathbb{R} \setminus \{0\} $ 
\end{itemize}
\defn{Nullteiler}{
	Ein Element $ a \neq 0 $ eines Ringes $ R $ heißt \textbf{Nullteiler} von $ R $ genannt, wenn ein 
	$ b \neq 0 $ in R existiert mit $ a b = 0 \text{ oder } b a=0	$. 
}
\defn{Integritätsringe}{
	Ein kommutativer, nullteilerfreier Ring mit 1 heißt \textbf{Integritätsringe} oder \textbf{Integritätsbereich}.
}
	Hiervon folgt für $ a b=0 \implies a = 0 \text{ oder } b = 0$ und für $ ac = bc \text{ mit } c \neq 0 \implies a = b $ 
\subsubsection*{Beispiel}
\begin{itemize}
	\item  $ \mathbb{Z}  $ ist ein Integritätsring
	\item Der Teilring $ \mathbb{Z}[i] := \left\{ a + \iu b \; | \; a, b \in  \mathbb{Z}\right\} $ 
		der \textit{ganze Gauß'schen Zahlen} ist Integritäts-\\bereich.
	\item Der Restklassenring $ \mathbb{Z} / p \mathbb{Z}  $ ist genau dann ein Integritätsring, wenn 
		$ p $ prim ist.
	\item $ R[X] $ ist ein Integritätsring. Es gilt weiterhin $ R[X]^{\times} = R^{\times} $ 
	\item Der Ring aus reellen wertigen Funktionen $ F (\mathbb{R}) $ ist kein Integritätsring, da es Nullteiler gibt. 
\end{itemize}

\defn{Ideale}{
	Eine Untergruppe A von $ \left( R , + \right)  $ heißt \textbf{Ideal} von $ R $, wenn gilt:
	\begin{itemize}
		\item $ a \in A, r \in R \implies r a \in  A \; : R A \subseteq A $  
		\item $ a \in A, r \in R \implies a r \in  A \; : A R \subseteq A $  
	\end{itemize}
	Gilt nur $ R A \subseteq A \text{ bzw. } A R \subseteq A $ für eine Untergruppe $ A \text{ von } 
	\left( R, + \right) $, so nennt man \textbf{Links-} bzw. \textbf{Rechtsideal}. Wenn ein Ring Links- und 
	Rechtsideal ist, dann ist der Ring \textbf{Ideal} von $ R $. Alle Ideale sind Teilring
}

\subsubsection*{Beispiel}
\begin{itemize}
	\item $ \left\{ 0 \right\} \text{ und } R $ sind die trivialen Ideale des Ringes R.
	\item Die Ideale von $ \mathbb{Z}  $ sind genau die Mengen $ n \mathbb{Z}  \text{ mit } n \in \mathbb{N}_{0}$ 
	\item Sei $ 1 \in A $ ein Ideal, dann $ A = R $ 
	\item $ \left\{ \begin{pmatrix}x &0 \\ y &0 \end{pmatrix} | \; x, y \in \mathbb{R} \right\}  $ ist ein 
		\textit{Linksideal} aber kein \textit{Rechtsideal} 
	\item $ \left\{ \begin{pmatrix}x &y \\ 0 &0 \end{pmatrix} | \; x, y \in \mathbb{R} \right\}  $ ist 
		\textit{Rechtsideal} aber kein \textit{Linksideal} 
\end{itemize}

\begin{theorem}
	Für jede Familie $ \left( A_{i} \right)_{i \in I} $ von Idealen $ A_{i} $ von $ R $ ist auch 
	$ A := \bigcap\limits_{ i \in  I}^{ } A_{i}  $ ein Ideal von $ R $ .
\end{theorem}
\subsubsection*{Beispiel}
$ \left( 2 \mathbb{Z}  \right) \cap \left( 3 \mathbb{Z}  \right) \cap \left( 4 \mathbb{Z}  \right) = 12 \mathbb{Z}  $
 
\end{document}
